\appendix
%\chapter{Ap\'endice}

\chapter{Plugins}
A continuaci\'on se muestran las listas de los tipos de m\'odulos implementados a la fecha en {\lpmd}. Adem\'as se indica la calidad del m\'odulo, segun la tabla~\ref{tab:modquality}


\begin{table}[h!]\centering
 \begin{tabular}{|c|l|}\hline\hline
 Calidad & Descripci\'on \\\hline\hline
 S & \textit{Stable} El plugin funciona de manera estable. \\
 T & \textit{Testing} El plugin funciona y se encuentra usable, pero debe ser precavido. \\
 U & \textit{Unstable} El plugin esta en fase de desarrollo, no utilizar para publicaciones. \\
\hline
 \end{tabular}
 \label{tab:modquality}
 \caption{Tabla de calidad de implementaci\'on de m\'odulos, s\'olo se presta soporte para los m\'odulos incluidos en el paquete \textbf{lpmd-plugins}}
\end{table}

\section{Entrada Salida}
M\'odulos para manejar los ficheros de entrada/salida para las configuraci\'ones at\'omicas que se simulan.

\begin{table}[h!]\centering
 \begin{tabular}{|l|c|c|p{10cm}|}\hline
 M\'odulo & Versi\'on & Calidad & Descripci\'on \\
 \hline\hline
 \texttt{dlpoly} & 2.0 & S & Lectura/Escritura de ficheros \texttt{HISTORY} y \texttt{CONFIG} de dl\_poly.\\
 \hline
 \texttt{lpmd} & 2.0 & S & Formato propio de {\lpmd}, soporte lectura/escritura, 3 niveles distintos y tags.\\
 \hline
 \texttt{vasp} & 5.0 & F & Lee ficheros \textbf{POSCAR} de vasp, el cu\'al posee las posiciones at\'omcias de la configuraci\'on.\\
 \hline
 \texttt{xyz} & 1.0 & S & Formato de ficheros \texttt{xyz}, sporta 3 niveles distintos.\\
 \hline
 \texttt{zlp} & 2.0 & S & Formato propio de {\lpmd}, utiliza las zlib, y 3 niveles distintos de manejo, la utilizaci\'on es similar a lpmd pero con ficheros de menor tama\~no.\\
 \hline
 \texttt{mol2} & 1.0 & T & Lectura/Escritura de ficheros \texttt{mol2}, soporte b\'asico.\\
 \hline
 \texttt{pdb} & 1.0 & T & Lectura/Escritura de ficheros \texttt{pdb}, soporte b\'asico.\\
 \hline
 \texttt{rawbinary} & 1.0 & S & Lectura/Escritura en modo binario. Ideal para espacio y velocidad.\\
 \hline
\end{tabular}
\label{tab:modinout}
\caption{Tabla con los m\'odulos de entrada y salida utilizados por {\lpmd} y sus utilitarios.}
\end{table}


\section{Generadores de Celda}
M\'odulos de {\lpmd} que generan celdas at\'omicas de forma autom\'atica.

\begin{table}[h!]\centering
 \begin{tabular}{|l|c|c|p{10cm}|}\hline
 M\'odulo & Versi\'on & Calidad & Descripci\'on \\
 \hline\hline
 \texttt{crystal3d} & 1.0 & S & Generador de celdas tridimensionales.\\
 \hline
 \texttt{crystal2d} & 1.0 & F & Generador de celdas bidimensionales.\\
 \hline
\texttt{voronoi} & 1.0 & F & Generador de Nanoestructuras utilizando m\'etodo de voronoi.\\
 \hline
 \texttt{skewstart} & 1.0 & F & Generador de celdas con m\'etodo skewstart, desarrollado por \textit{K. Refson}, para el programa de din\'amica molecular \textbf{moldy}.\\
 \hline
 \end{tabular}
\label{tab:cellgen}
\caption{M\'odulos generadores de celda usados por {\lpmd} y sus utilitarios.}
\end{table}

\section{Manejadores de Celda}
Determinan como es la forma de interactuar entre los \'atomos pertenecientes a la celda de simulaci\'on.

\begin{table}[h!]\centering
 \begin{tabular}{|l|c|c|p{10cm}|}\hline
 M\'odulo & Versi\'on & Calidad & Descripci\'on \\
 \hline\hline
 \texttt{linkedcell} & 2.0 & S & M\'odulo que utiliza lpmd, para manejar las listas de interacci\'on, utilizando el m\'etodo \textit{Linked Cell}.\\
 \hline
 \texttt{minimumimage} & 2.0 & S & M\'odulo que utiliza lpmd, para manejar las listas de interacci\'on, utilizando el m\'etodo de \textit{m\'inima im\'agen}.\\
 \hline
 \texttt{lcbinary} & 1.0 & S & M\'odulo que utiliza lpmd, para manejar las listas de interacci\'on, utilizando el m\'etodo de \textit{Linked Cell} con \'un \'atomo por celda.\\
 \hline
 \texttt{verletlist} & 1.0 & U & M\'odulo que utiliza lpmd, para manejar las listas de interacci\'on, utilizando el m\'etodo de \textit{Verlet List}.\\
 \hline
 \end{tabular}
\label{tab:modmanager}
\caption{Tabla con los m\'odulos que manejan las interacciones at\'omicas en la din\'amica molecular.}
\end{table}

\section{Filtros}
Act\'uan sobre la celda de simulaci\'on y son capaces de seleccionar \'atomos de la celda en distinta forma.

\begin{table}[h!]\centering
 \begin{tabular}{|l|c|c|p{10cm}|}\hline
 M\'odulo & Versi\'on & Calidad & Descripci\'on \\
 \hline\hline
 \texttt{box} & 1.0 & S & Selecciona \'atomos fuera o dentro de una caja de largos espec\'ificos.\\
 \hline
 \texttt{element} & 1.0 & S & M\'odulo que utiliza lpmd, para seleccionar \'atomos seg\'un su s\'imbolo at\'omico.\\
 \hline
 \texttt{index} & 1.0 & S & M\'odulo que utiliza lpmd, para seleccionar \'atomos seg\'un su \'indice.\\
 \hline
 \texttt{sphere} & 1.0 & S & Selecciona \'atomos fuera y dentro de una esfera.\\
 \hline
 \texttt{tag} & 1.0 & S & M\'odulo que utiliza lpmd, para seleccionar \'atomos seg\'un su tag.\\
 \hline
 \end{tabular}
\label{tab:filtros}
\caption{Tabla con los m\'odulos que manejan las interacciones at\'omicas en la din\'amica molecular.}
\end{table}


\section{Modificadores}
Son los m\'odulos que alteran propiedades de la celda, tales como tama\~no, forma, o bien modifican los \'atomos que se encuentran dentro de ella.

\begin{table}[h!]\centering
 \begin{tabular}{|l|c|c|p{10cm}|}\hline
 M\'odulo & Versi\'on & Calidad & Descripci\'on \\
 \hline\hline
 \texttt{berendsen} & 2.0 & S & Escalamiento de temperatura usando algoritmo de berendsen.\\
 \hline
 \texttt{cellscaling} & 2.0 & S & Escalamiento din\'amico de Celda.\\
 \hline
 \texttt{displace} & 2.0 & S & Desplazamiento de \'atomos.\\
 \hline
 \texttt{moleculecm} & 2.0 & S & Crea mol\'eculas diat\'omicas a partir de \'atomos enlazados.\\
 \hline
 \texttt{propertycolor} & 1.0 & S & Colorea \'atomos seg\'un propiedad.\\
 \hline
 \texttt{quenchedmd} & 2.0 & S & Modificaci\'on estructural usando \textit{Quenched MD}.\\
 \hline
 \texttt{randomatom} & 2.0 & S & Eliminaci\'on/Selecci\'on de \'atomos al azar.\\
 \hline
 \texttt{replicate} & 2.0 & S & Replica celda original.\\
 \hline
 \texttt{rotate} & 2.0 & S & Rotaci\'on de \'atomos.\\
 \hline
 \texttt{setcolor} & 2.0 & S & Setea color de los \'atomos.\\
 \hline
 \texttt{settag} & 2.0 & S & Setea tag de los \'atomos.\\
 \hline
 \texttt{setvelocity} & 2.0 & S & Setea velocidad de los \'atomos.\\
 \hline
 \texttt{shear} & 2.0 & S & Modifica la celda realizando cizalle.\\
 \hline
 \texttt{temperature} & 2.0 & S & Asignaci\'on de temperatura a grupos de \'atomos.\\
 \hline
 \texttt{tempscaling} & 2.0 & S & Escalamiento de temperatura.\\
 \hline
 \texttt{thermalneedle} & 2.0 & S & Aguja t\'ermica - obsoleto.\\
 \hline
 \texttt{undopbc} & 2.0 & S & Deshacer periodicidad en los ejes.\\
 \hline
 \end{tabular}
\label{tab:modmodify}
\caption{Tabla con los m\'odulos modificadores del sistema utilizado por {\lpmd}.}
\end{table}

\section{Propiedades Instant\'aneas}
Calculan caracteristicas propias del sistema at\'omico en estudio, son porpiedades que pueden ser calculadas en cada instante de tiempo, además de la posibilidad de ser promediadas sobre intervalos. Estas propiedades pueden ser calculadas durante la simulaci\'on o bien ser calculadas en forma independiente a posteriori.

\begin{table}[h!]\centering
 \begin{tabular}{|l|c|c|p{10cm}|}\hline
 M\'odulo & Versi\'on & Calidad & Descripci\'on \\
 \hline
 \texttt{angdist} & 2.0 & S & Calcula la distribuci\'on angular de la muestra.\\
 \hline
 \texttt{atomtrail} & 1.0 & S & .\\
 \hline
 \texttt{cna} & 2.0 & S & Realiza un \textit{Common Nieghbor Analysis} de la muestra.\\
 \hline
 \texttt{cordnumfunc} & 2.0 & S & Calcula la \textit{funci\'on n\'umero de coordinaci\'on} de la muestra.\\
 \hline
 \texttt{cordnum} & 2.0 & S & Calcula el n\'umero de coordinaci\'on en forma de histograma.\\
 \hline
 \texttt{densityprofile} & 2.0 & S & Genera un perfil de la densidad de la muestra.\\
 \hline
 \texttt{gdr} & 2.0 & S & Calcula la funci\'on de distribuci\'on de pares de la muestra.\\
 \hline
 \texttt{localpressure} & 2.0 & T & Genera un perfil de presiones locales.\\
 \hline
 \texttt{overlap} & 2.0 & S & .\\
 \hline
 \texttt{pairdistances} & 2.0 & S & Busca la distancias entre los pares de una muestra.\\
 \hline
 \texttt{rvcorr} & 2.0 & S & .\\
 \hline
 \texttt{sitecoord} & 2.0 & S & .\\
 \hline
 \texttt{tempprofile} & 2.0 & S & Perfil de temperaturas de la muestra.\\
 \hline
 \texttt{veldist} & 2.0 & S & Dsitribuci\'on de velocidades de la muestra.\\
 \hline
 \end{tabular}
\label{tab:modproper}
\caption{Tabla con los m\'odulos generales utilizados por lpmd.}
\end{table}

\section{Propiedades Temporales}
Calculan caracteristicas temporales del sistema, estas propiedades no pueden ser calculadas \textit{durante} la simulaci\'on, pero si pueden calcularse utilizando \verb|lpmd-analyzer| para los archivos de salida de las simulaciones previas.

\begin{table}[h!]\centering
 \begin{tabular}{|l|c|c|p{10cm}|}\hline
 M\'odulo & Versi\'on & Calidad & Descripci\'on \\
 \hline
 \texttt{dispvol} & 2.0 & S & Calcula el volumen desplazado de los \'atomos.\\
 \hline
 \texttt{mobility} & 2.0 & S & Calcula la mobilidad at\'omica.\\
 \hline
 \texttt{msd} & 2.0 & S & Calcula el desplazamiento cuadr\'atico medio.\\
 \hline
 \texttt{vacf} & 2.0 & S & Calcula la funci\'on de autocorrelaci\'on de velocidades.\\
 \hline
 \end{tabular}
\label{tab:modtempproper}
\caption{Tabla con los m\'odulos generales utilizados por lpmd.}
\end{table}

\section{Integradores}
Resuelven las ecuaciones de movimiento del sistema.

\begin{table}[h!]\centering
 \begin{tabular}{|l|c|c|p{10cm}|}\hline
 M\'odulo & Versi\'on & Calidad & Descripci\'on \\
 \hline\hline
 \texttt{beeman} & 2.0 & S & Integrador de Beeman.\\
 \hline
 \texttt{euler} & 2.0 & S & Integrador de euler.\\
 \hline
 \texttt{hardspheres} & 1.0 & T & M\'etodo de esferas duras para \textit{mover} los \'atomos.\\
 \hline
 \texttt{leapfrog} & 2.0 & S & Integrador de leapfrog.\\
 \hline
 \texttt{metropolis} & 2.0 & T & M\'etodo de metropolis, usado en minimizaci\'on de estructuras.\\
 \hline
 \texttt{nosehoover} & 2.0 & T & Integrador de nosehoover para ensambles NPT.\\
 \hline
 \texttt{nullintegrator} & 2.0 & S & No mueve los \'atomos.\\
 \hline
 \texttt{velocityverlet} & 2.0 & S & Integrador de VelocityVerlet.\\
 \hline
 \texttt{verlet} & 2.0 & S & Integrador de Verlet.\\
 \hline
 \end{tabular}
\label{tab:modinteg}
\caption{Tabla con los m\'odulos generales utilizados por lpmd.}
\end{table}

\section{Potenciales de Pares}
Plugins especializados en la interacci\'on de pares que llevan a cabo los \'atomos involucrados.

\begin{table}[h!]\centering
 \begin{tabular}{|l|c|c|p{10cm}|}\hline
 M\'odulo & Versi\'on & Calidad & Descripci\'on \\
 \hline\hline
 \texttt{buckingham} & 2.0 & S & Interacci\'on at\'omica con potencial de Buckingham.\\
 \hline
 \texttt{harmonic} & 2.0 & S & Interacci\'on at\'omica con potencial Arm\'onico.\\
 \hline
 \texttt{lennardjones} & 2.0 & S & Interacci\'on at\'omica con potencial de Lennard-Jones.\\
 \hline
 \texttt{morse} & 2.0 & S & Interacci\'on at\'omica con potencial de Morse.\\
 \hline
 \texttt{nullpairpotential} & 2.0 & S & Interacci\'on at\'omica nula.\\
 \hline
\texttt{tabulatedpair} & 1.0 & U & Interacci\'on at\'omica leida desde una tabla de datos.\\
 \hline
 \end{tabular}
\label{tab:modpotentials}
\caption{Tabla con los Potenciales interat\'omicos con los que cuenta {\lpmd}.}
\end{table}

\section{Potenciales Metalicos}
Son los que determinan como interactuan los \'atomos durante la simulaci\'on.

\begin{table}[h!]\centering
 \begin{tabular}{|l|c|c|p{10cm}|}\hline
 M\'odulo & Versi\'on & Calidad & Descripci\'on \\
 \hline\hline
 \texttt{finnissinclair} & 2.0 & T & Interacci\'on at\'omica con potencial de Finnis-Sinclair.\\
 \hline
 \texttt{gupta} & 2.0 & T & Interacci\'on at\'omica con potencial de Gupta.\\
 \hline
 \texttt{nullmetalpotential} & 2.0 & S & Interacci\'on at\'omica nula.\\
 \hline
 \texttt{suttonchen} & 2.0 & S & Interacci\'on at\'omica con potencial de Sutton-Chen.\\
 \hline
 \end{tabular}
\label{tab:modmetalpotentials}
\caption{Tabla con los Potenciales interat\'omicos con los que cuenta {\lpmd}.}
\end{table}


\section{Visualizadores}
Utilizados para obtener imagenes de la simulaci\'on.

\begin{table}[h!]\centering
 \begin{tabular}{|l|c|c|p{10cm}|}\hline
 M\'odulo & Versi\'on & Calidad & Descripci\'on \\
 \hline\hline
 \texttt{average} & 2.0 & S & Visualizador de promedios de datos de simulaci\'on.\\
 \hline
 \texttt{lpvisual} & 2.0 & S & Visualizador de archivos de din\'amica molecular basado en openGL.\\
 \hline
 \texttt{monitor} & 2.0 & S & Visualizador de datos instant\'aneos de la simulaci\'on.\\
 \hline
 \texttt{printatoms} & 2.0 & S & .\\
 \hline
 \end{tabular}
\label{tab:modgvisual}
\caption{Tabla con los m\'odulos visualizadores de lpmd.}
\end{table}



\chapter{API - liblpmd}
\label{ap:API}
La \textbf{API} (Ap. Programming Interface) es una herramienta de programaci\'on que puede ser utilizada por cualquier usuario/programador que se vea beneficiado por sus caracter\'isticas.

Consideramos que la mejor forma de comprender el funcionamiento de esta \textbf{API}, es directamente con c\'odigos de ejemplo que pueden escribir los desarrolladores. A continuaci\'on se muestran 3 ejemplos de utilizaci\'on de la \textbf{API}, el primero se enmarca en un ``nano-programa'' de \textbf{DM}, el segundo es la evaluaci\'on de una propiedad est\'atica de una celda del tipo \texttt{.xyz} y la \'ultima una propiedad din\'amica de una celda.

%%%%%%%%%%%%%%%%%%%%%%%%%%%%%%%%%%%%%%%%%%%%%%%%%%%%%%%%%%%%%%%%%
%%%%%%%%%%%%%%%%%%%%%%%%%%%%%%%%%%%%%%%%%%%%%%%%%%%%%%%%%%%%%%%%%
\section{Din\'amica Molecular B\'asica}

A continuaci\'on un c\'odigo que utilza todas las caracter\'isticas de la \textbf{API}, para realizar din\'amica molecular.

\begin{verbatim}
 /*
 * Ejemplo simple de dinamica molecular usando el API de liblpmd
 */

#include <lpmd/api.h>
#include <iostream>

using namespace lpmd;

int main()
{
 MD md;            // define md como un objeto de dinamica molecular
 PluginManager pm; // define pm como un manejador de plugins

 SimulationCell cell(1, 1, 1, true, true, true); // cell es la celda de simulacion
 cell.SetVector(0, Vector(17.1191, 0.0, 0.0));   // define los vectores de la celda
 cell.SetVector(1, Vector(0.0, 17.1191, 0.0));
 cell.SetVector(2, Vector(0.0, 0.0, 17.1191));
 md.SetCell(cell);                    // asigna la celda de simulacion al objeto MD 

 // Carga de plugins con sus parametros
 pm.LoadPlugin("minimumimage", "");
 pm.LoadPlugin("crystalfcc", "symbol Ar nx 3 ny 3 nz 3");
 pm.LoadPlugin("lennardjones", "sigma 3.41 epsilon 0.0138");
 pm.LoadPlugin("velocityverlet", "dt 1.0");
 pm.LoadPlugin("temperature", "t 600.0");
 pm.LoadPlugin("energy", "");

 CellManager & cm = CastModule<CellManager>(pm["minimumimage"]);
 cell.SetCellManager(cm);            // asigna el manejador de celda

 CellGenerator & cg = CastModule<CellGenerator>(pm["crystalfcc"]);
 cg.Generate(cell);

 Potential & pot = CastModule<Potential>(pm["lennardjones"]);
 PotentialArray & potarray = md.GetPotentialArray();
 potarray.Set("Ar", "Ar", pot); // asigna lennardjones al arreglo de potenciales de MD

 Integrator & integ = CastModule<Integrator>(pm["velocityverlet"]);
 md.SetIntegrator(integ);

 InstantProperty & energ = CastModule<InstantProperty>(pm["energy"]);
 
 SystemModifier & therm = CastModule<SystemModifier>(pm["temperature"]);
 therm.Apply(cell);  // aplica el termalizador "temperature" a la celda de simulacion

 // Loop principal de la simulacion, hace 500 pasos
 md.Initialize(); 
 std::cout << "# Pasos   Temperatura" << '\n';
 for (long i=0;i<500;++i)
 {
  md.DoStep();                       // avanza el sistema un paso de simulacion
  energ.Evaluate(cell, pot);         // evalua las propiedades en el plugin energy
  double T;
  T = pm["energy"].GetProperty("temperature"); // pide valor de temp al plugin energy
  std::cout << i << "         " << T << '\n';
 }
 return 0;
}
\end{verbatim}

Para generar el ejecutable,

\control{g++ -o nanodm main.cc -llpmd -ldl -lm}

y listo, tendremos entonces un ejecutable llamado \verb|nanodm| que realizar\'a una simple corrida de din\'amica molecular.

%%%%%%%%%%%%%%%%%%%%%%%%%%%%%%%%%%%%%%%%%%%%%%%%%%%%%%%%%%%%%%%%%
%%%%%%%%%%%%%%%%%%%%%%%%%%%%%%%%%%%%%%%%%%%%%%%%%%%%%%%%%%%%%%%%%
\section{Calculo de Propiedad est\'atica}

Consideremos que tenemos una celda de simulaci\'on y queremos utiliar las ventajas de la \textbf{API} para calcular una propiedad, que sabemos existe en un m\'odulo, por ejemplo \textbf{gdr}. El c\'odigo para el c\'alculo de \textbf{gdr} de la celda nos quea as\'i,

\begin{verbatim}
 /*
 *
 *
 *
 */

#include <lpmd/api.h>

using namespace lpmd;

int main(int argc, char *argv[])
{
 if (argc < 2) 
 {
  std::cerr << "testgdr <file.xyz>" << '\n';
  exit(1);
 }
 PluginManager pm;
 pm.LoadPlugin("xyz", "file="+std::string(argv[1]));
 pm.LoadPlugin("gdr", "rcut 8.0 bins 300 average true");
 pm.LoadPlugin("nullpotential", "");
 pm.LoadPlugin("linkedcell", "nx 7 ny 7 nz 7 cutoff 8.0");

 CellReader & cread = dynamic_cast<CellReader &>(pm["xyz"]);
 InstantProperty & gdr = dynamic_cast<InstantProperty &>(pm["gdr"]); 
 ScalarTable & gdrvalue = dynamic_cast<ScalarTable &>(pm["gdr"]);
 CellManager & cm = dynamic_cast<CellManager &>(pm["linkedcell"]);
 Potential & dummy = dynamic_cast<Potential &>(pm["nullpotential"]);

 pm["gdr"].Show();

 std::vector<SimulationCell> configs;
 cread.ReadMany(std::string(argv[1]), configs);

 Cell cell(13.16, 13.16, 21.39, M_PI/2, M_PI/2, M_PI*120.0/180.0);
 Vector v1 = cell.GetVector(0);
 v1 = Vector(v1.Get(1), v1.Get(0), v1.Get(2));
 Vector v2 = cell.GetVector(1);
 v2 = Vector(v2.Get(1), v2.Get(0), v2.Get(2));
 cell.SetVector(0, v2);
 cell.SetVector(1, v1);
 for (int i=0;i<3;++i) std::cerr << cell.GetVector(i) << std::endl;

 std::cerr << "Read " << configs.size() << " configurations." << '\n';
 std::cerr << "Configuration 0 has " << configs[0].Size() << " atoms\n";

 for (unsigned long i=0;i<configs.size();++i)
 {
  configs[i].SetCell(cell);
  configs[i].SetCellManager(cm);
  gdr.Evaluate(configs[i], dummy);
  gdrvalue.AddToAverage();
 }

 std::cout << gdrvalue << '\n';

 return 0;
}
\end{verbatim}

Esto, lo compilamos de manera similar al caso anterior, obteniendo un ejecutable para calcular una propiedad est\'atica, en este caso \verb|gdr| para la celda de simulaci\'on.
