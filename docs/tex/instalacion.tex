\chapter{Instalaci\'on}
\label{chap:inst}

{\lpmd} ha sido probado en distintas arquitecturas y compiladores. Sin embargo a partir de la version 0.6.0 s\'olo hemos mantenido pruebas sobre dos arquitecturas principales :

\begin{itemize}
 \item Linux I686/AMD64
 \item OS/X
\end{itemize}

Adem\'as actualmente {\lpmd} cuenta con \verb|lpmd-installer|, un programa enfocado a la instalaci\'on sencilla de lpmd en cualquier maquina, que est\'a escrito en python. Nosotros recomendamos a los usuarios no expertos a que utilizen \verb|lpmd-installer| para la instalaci\'on de {\lpmd} en sus sistemas.

\section{lpmd-installer}

Utilidad enfocada para una instalación amigable de {\lpmd}, llamada \verb|lpmd-installer|. Este programa escrito en python se encarga de instalar de forma automatica los 3 paquetes principales que forman parte de lpmd. Estos son :

\begin{itemize}
 \item liblpmd
 \item plugins
 \item lpmd
\end{itemize}

Si no se desea utilizar la versi\'on autom\'atica de instalaci\'on basada en \verb|lpmd-installer| contin\'ue a la siguiente secci\'on.

\subsection{Instalando lpmd-installer}

El programa \verb|lpmd-installer| es un script escrito en python por lo que las formas de instalaci\'on son bastantes simples, para funcionar correctamente \verb|lpmd-installer| necesita :

\begin{itemize}
\item Python 2.3 o superior.
\item Conecci\'on a internet.
\end{itemize}

Para instalar \verb|lpmd-installer| en su sistema existen dos formas, la primera es para distribuciones basadas en debian en donde contamos con un repositorio especial para ello, o bien en otro tipo de distribuciones para lo cual es necesario descargar el programa e instalarlo manualmente.

\subsubsection{Distribuciones basadas en Debian}

Se ha probado en distribuciones como Debian/Ubuntu, para proceder, en primer lugar como administrador, edite el fichero \verb|\etc\apt\sources.list| y a\~nada :

\begin{verbatim}
 #GNM Repositories
 deb http://www.gnm.cl/repos ./
\end{verbatim}

Luego como administrador nuevamente ejecute :

\begin{verbatim}
 apt-get update
 apt-get install lpmd-installer
\end{verbatim}

Ahora verifique que tenga el comando
\begin{verbatim}
 username@machine:~$ lpmd-installer 
 lpmd-installer [ -i <branch or version> | -u ] [-v] [-p <prefix>] [-P <package>] 
                [-s <server>] [-S <suffix>] [ -d <sources dir>] [-t]
 username@machine:~$
\end{verbatim}

Listo ya esta en condiciones de poder instalar {\lpmd} facilmente.

\subsubsection{Otras Distribuciones}

Para instalar \verb|lpmd-installer| en distribuciones no basadas en Debian, es necesario descargar el programa directamente desde internet, puede encontrarlo en esta direcci\'on :

\begin{verbatim}
 http://www.gnm.cl/lpmd/uploads/Main/lpmd-installer
\end{verbatim}

Descarge el archivo y asignele permisos de ejecuci\'on :

\begin{verbatim}
 chmod 755 lpmd-installer
\end{verbatim}

Luego de eso, se puede copiar o mover a algun lugar donde se tenga \verb|PATH| del sistema. En la mayor\'ia de los sisemas *nix puede copiarse o moverse a :

\begin{verbatim}
 cp lpmd-installer /usr/local/bin/
\end{verbatim}

Y listo, ahora verifique que tiene el comando \verb|lpmd-installer| disponible en su sistema.
\begin{verbatim}
 username@machine:~$ lpmd-installer 
 lpmd-installer [ -i <branch or version> | -u ] [-v] [-p <prefix>] [-P <package>] 
                [-s <server>] [-S <suffix>] [ -d <sources dir>] [-t]
 username@machine:~$
\end{verbatim}

\subsection{Probando lpmd-installer}
El programa cuenta con todo lo necesario para instalar {\lpmd} en un sistema as\'i como tambi\'en en cuentas personales, as\'i como tambi\'en facilidades para instalar \textit{plugins} externos, tales como \verb|lpvisual|. Entre las opciones principales de \verb|lpmd-installer| se cuenta con :

\begin{itemize}
 \item \verb|-l| : Lista las versiones disponibles de instalar.
 \item \verb|-p| : Especifica donde se instalar\'a {\lpmd}.
 \item \verb|-P| : Especif\'ica que paquete se instalar\'a.
 \item \verb|-d| : Especif\'ica donde quedar\'an los sources, sino se especif\'ica son elmiminados.
\end{itemize}

A continuaci\'on veremos como se instala {\lpmd} utilizando \verb|lpmd-installer|.

\subsection{Instalando lpmd en el sistema}

Antes de ejecutar la instalaci\'on directa de {\lpmd} en el sistema es conveniente ver la disponibilidad de las versiones disponibles en el servidor principal, para ello ejecute \verb|lpmd-installer -l|, y obtendra una lista como la siguiente :

\begin{verbatim}
username@machine:~$ lpmd-installer -l
These are the available versions of package lpmd:

From branch 0.5: 
    0.5.4-delta2
    0.5.4-delta2-openmp
    0.5.4-delta1
    0.5.4
    0.5.3

From branch 0.6: 
    0.6.1
    0.6.0
username@machine:~$
\end{verbatim}

Elegimos de la lista la version que se desea instalar, en este caso la version 0.6.1, si intentamos instalar esta version en el sistema sin ser \textbf{administrador} obtendremos un mensaje como  :

\begin{verbatim}
username@machine:~$ lpmd-installer -i 0.6.1
Preparing to install lpmd 0.6.1
[Error] You do not have permission to install on /usr/local
username@machine:~$
\end{verbatim}

Esto ocurre porque por defecto lpmd se intenta instalar en \verb|/usr/local| y un usuario normal no tiene permisos de escritura en este directorio. La manera correcta de proceder es, instalarlo como \textbf{administrador}. Para ello :

\begin{verbatim}
username@machine:~$ sudo -s
[sudo] password for username: 
root@machine:~# lpmd-installer -i 0.6.1
\end{verbatim}

Ahora si, el proceso comenzar\'a de forma autom\'atica a descargar desde el servidor principal los paquetes necesarios para la instalaci\'on. Recuerde que se requiere al menos un compilador de \verb|C++| para llevar a cabo la instalaci\'on.

\subsection{Instalando lpvisual utilizando lpmd-installer}

Con \verb|lpmd-installer| tambi\'en podremos realizar la instalaci\'on de \verb|lpvisual| un plugin independiente basado en OpenGL para la visualizaci\'on de configuraciones at\'omicas de distintos tipos.

Para instalar facilmente lpvisual en el sistema podemos utilizar las mismas opciones de \verb|lpmd-installer| utilizadas previamente :

\begin{verbatim}
username@machine:~$ lpmd-installer -P lpvisual -l
These are the available versions of package lpvisual:

From branch 2.0: 
    6.0
username@machine:~$  
\end{verbatim}

Entonces elegimos la version de \verb|lpvisual| que deseamos instalar y la instalamos facilmente con :

\begin{verbatim}
username@machine:~$ lpmd-installer -P lpvisual -i 6.0
....
\end{verbatim}

Con esto entonces hemos instalado un \textbf{plugin adicional} utilizando lpmd-installer.

\subsection{Instalando lpmd en una cuenta personal}

Si no somos administradores o no tenemos acceso a \'el es posible que nuestra intenci\'on sea instalar {\lpmd} en una cuenta personal, para ello \verb|lpmd-installer| cuenta con flags opcionales para especificar  esto. Consideremos que deseamos instalar {\lpmd} en un directorio \verb|~/local/| para ello es necesario entonces especificar el directorio con el flag \verb|-p| como se muestra a continuaci\'on :

\begin{verbatim}
username@machine:~$ lpmd-installer -i 0.6.1 -p /home/username/local
....
username@machine:~$
\end{verbatim}

con este comando hemos instalado entonces {\lpmd} en nuestra cuenta personal, adem\'mas podemos elegir mantener el c\'odigo fuente en algun directorio en especial :

\begin{verbatim}
username@machine:~$ lpmd-installer -i 0.6.1 -p /home/username/local -d /home/username/sources
....
username@machine:~$
\end{verbatim}

Para m\'as soporte y ayuda puede enviar e-mail a los desarrolladores o visitar la web principal del proyecto \verb|http://www.gnm.cl|.

\section{Descarga}

Si ha decidido no instalar {\lpmd} utilizando \verb|lpmd-installer| entonces debe descargar los 3 paquetes principales para la posterior instalaci\'on, pare ello existen dos versiones disponibles, la version estable y la versi\'on de pruebas.

Para el correcto funcionamiento de {\lpmd} es necesario instalar previamente una librer\'ia (API) y un set de plugins. Por lo que se requieren descargar 3 paquetes principales. La divisi\'on del c\'odigo en este set de paquetes se debe a la reutilizaci\'on de la \textbf{API} y los \textbf{plugins} para nuevos programas de utilidades o bien para un propio usuario interesado en la programaci\'on. Lo que adem\'as lleva una simplificaci\'on a aquellas personas que desean programar sus propios \textbf{plugins}.

%%%%%%%%%%%%%%%%%%%%%%%%%%%%%%%%%%%%%%%%%%%%%%%%%%%%%%%%%%%%%%%%%
%%%%%%%%%%%%%%%%%%%%%%%%%%%%%%%%%%%%%%%%%%%%%%%%%%%%%%%%%%%%%%%%%
\subsection{Descarga de versi\'on estable}

La \'ultima versi\'on estable de los paquetes es :

\begin{itemize}
 \item liblpmd : ver. 0.2.1
 \item plugins : ver. 0.2.0
 \item lpmd    : ver. 0.6.1
\end{itemize}

%Los n\'umeros de los plugins indican la compatibilidad con las versiones de {\lpmd} y de la \textbf{API} liblpmd, de esta forma se mantiene un \textbf{orden} seg\'un la versi\'on que conocemos.

Estos pueden ser descargados directamente en la pagina web \texttt{http://www.gnm.cl/lpmd}, o bien solicitados por e-mail.

%%%%%%%%%%%%%%%%%%%%%%%%%%%%%%%%%%%%%%%%%%%%%%%%%%%%%%%%%%%%%%%%%
%%%%%%%%%%%%%%%%%%%%%%%%%%%%%%%%%%%%%%%%%%%%%%%%%%%%%%%%%%%%%%%%%
\subsection{Descarga de versi\'on en desarrollo}

Para los interesados en el desarrollo de {\lpmd} y sus complementos principales(API, plugins, utilitarios, etc.), la version de pruebas ``\textit{testing}'' se puede descargar con subversion:

\begin{center}
 \begin{verbatim}
  svn co svn://www.gnm.cl/lpmd/liblpmd/testing liblpmd-uns
  svn co svn://www.gnm.cl/lpmd/plugins/testing plugins-uns
  svn co svn://www.gnm.cl/lpmd/lpmd/testing lpmd-uns
 \end{verbatim}
\end{center}

Est\'a disponible adem\'as una rama \verb|unstable|; sin embargo, no recomendamos utilizarla para c\'alculos de Din\'amica Molecular. S\'olo utilizable para investigaci\'on y prueba de c\'odigos.

\section{Instalaci\'on}
Antes de comenzar con la instalaci\'on de {\lpmd} es necesario instalar 2 paquetes previos, \textbf{liblpmd} y \textbf{plugins}. A continuaci\'on se muestra una descripci\'on de c\'omo instalar cada uno de los paquetes.\\

\fb{
\begin{minipage}[l]{10cm}
Nota : Los que poseen la versi\'on testing/unstable cuentan con el fichero \textbf{autogen.sh} para poder generar los Makefiles. \\ Se necesita instalar automake y libtool para poder ejecutar \textbf{autogen.sh}.
\end{minipage}
}

%%%%%%%%%%%%%%%%%%%%%%%%%%%%%%%%%%%%%%%%%%%%%%%%%%%%%%%%%%%%%%%%%
%%%%%%%%%%%%%%%%%%%%%%%%%%%%%%%%%%%%%%%%%%%%%%%%%%%%%%%%%%%%%%%%%
\subsection{Instalando liblpmd}

% \cajatx{Nota : La versi\'on inestable requiere que tenga instalando autmake y libtool. Para ejecutar autogen.sh}
% 
% En primer lugar debe tener instalado automake y libtool, si no lo tiene puede hacerlo como administrador de la siguiente manera:
% 
% \control{apt-get install automake libtool}

En primer lugar descomprima el paquete de la \textbf{liblpmd},

\control{tar -xvzf liblpmd-X.X.X.tar.gz}

lo que le generar\'a un nuevo directorio; para instalar esta librer\'ia con todos los requerimientos necesarios para el funcionamiento de {\lpmd} y la implementaci\'on de plugins ejecute :

\control{./configure \\ make}

y como administrador,


\control{make install}


Por \textit{default} el directorio de instalaci\'on de la API programa es \verb|/usr/local/|, en caso de requerir una ubicaci\'on distinta revise las opciones con \verb|./configure --help|. Y si desea instalarlo en un directorio personal refi\'erase a la secci\'on~\ref{subsub:personaldir}.

En caso de cualquier error en el proceso de instalaci\'on env\'ie un e-mail a alguno de los desarrolladores principales o en su defecto a \verb|gnm@gnm.cl|.

%%%%%%%%%%%%%%%%%%%%%%%%%%%%%%%%%%%%%%%%%%%%%%%%%%%%%%%%%%%%%%%%%
%%%%%%%%%%%%%%%%%%%%%%%%%%%%%%%%%%%%%%%%%%%%%%%%%%%%%%%%%%%%%%%%%
\subsection{Instalando plugins}

Uno de los requerimientos b\'asicos de {\lpmd} es tener bien especificada la ubicaci\'on de los plugins que {\lpmd} requiere, es por eso que se debe indicar la ubicaci\'on de la instalaci\'on de la librer\'ia \textbf{liblpmd}. Normalmente deber\'ia correr sin ningun requerimiento especial si instal\'o la \textbf{liblpmd} en el lugar por \textit{default} (/usr/local), de no ser as\'i especifique el lugar con \verb|--with-lpmd=/ubicacion/|.

\control{./configure  \\ make}

y proceder la instalaci\'on como administrador:

\control{make install}

Esto ubicar\'a todos los plugins incluidos en el paquete \verb|plugins| en el directorio \verb|/usr/local/lib/lpmd|. De esta forma ya estamos listos para comenzar la instalaci\'on de lpmd y realizar las primeras pruebas.

%%%%%%%%%%%%%%%%%%%%%%%%%%%%%%%%%%%%%%%%%%%%%%%%%%%%%%%%%%%%%%%%%
%%%%%%%%%%%%%%%%%%%%%%%%%%%%%%%%%%%%%%%%%%%%%%%%%%%%%%%%%%%%%%%%%
\subsection{Instalando lpmd}

Es uno de los paquetes m\'as peque\~nos y r\'apidos de instalar; para proceder, se hace de manera similar que los anteriores, ejecutando:

\control{./configure \\ make}

y proceder a instalar como administrador:

\control{make install}

Esto generar\'a un set de ejecutables \verb|lpmd|, \verb|lpmd-analyzer| y \verb|lpmd-converter| en \verb|/usr/local/bin/|, que puede ser ejecutado desde cualquier sitio (si se presenta alg\'un problema, corriga su \verb|PATH|).

Puede correr lpmd con

\begin{verbatim}
username@machine:~$ lpmd
...
LPMD version 0.5.2
Using liblpmd version 1.0.2

Usage: lpmd [--verbose | -v ] [--lengths | -L <a,b,c>] [--angles | 
-A <alpha,beta,gamma>] [--vector | -V <ax,ay,az,bx,by,bz,cx,cy,cz>] 
[--scale | -S <value>] [--option | -O <option=value,option=value,...>] 
[--input | -i plugin:opt1,opt2,...] [--output | -o plugin:opt1,opt2,...] 
[--use | -u plugin:opt1,opt2,...] [--replace-cell | -r] [file.control]
       lpmd [--pluginhelp | -p <pluginname>]
       lpmd [--help | -h]
\end{verbatim}

\subsection{Problemas t\'ipicos post-instalaci\'on}

\subsubsection{Error cargando librer\'ia}

Es uno de los errores m\'as comunes luego de la instlaci\'on de {\lpmd}. Ocurre que al ejecutar {\lpmd} no muestra nada m\'as que un error referencial a la librer\'ia liblpmd que no puede ser encontrada.

La forma de corregir el problema es,

\begin{itemize}
 \item Editar /etc/ld.so.conf
 \item A\~nadir al archivo la l\'inea /usr/local/lib (o donde se haya instalado liblpmd)
 \item Ejecutar como admininistrador el comando: ldconfig
\end{itemize}

Ahora deber\'ia ejecutar el comando sin problemas.

\subsection{Instalaci\'on de lpmd en directorio personal}
\label{subsub:personaldir}

Podemos instalar cada uno de los paquetes (\textbf{liblpmd}, \textbf{plugins} y \textbf{lpmd}) en un directorio personal, para eso consideremos un ejemplo, en el cual deseamos instalar estos paquetes en el directorio \verb|local| ubicado en el \textit{home} del usuario, el procedimiento ser\'ia:

\begin{itemize}
 \item Para liblpmd
 \begin{verbatim}
 ./configure --prefix=/home/user/local
 make
 make install
 \end{verbatim}
 \item Para plugins
 \begin{verbatim}
 ./configure --prefix=/home/user/local --with-lpmd=/home/user/local
 make
 make install
 \end{verbatim}
 \item Para lpmd, es necesario indicar con variables de ambiente para la compilaci\'on, note que \verb|\\| indica que es una sola l\'inea que contin\'ua.
 \begin{verbatim}
 LDFLAGS="-Wl,--rpath -Wl,/home/user/local/lib" \\
 ./configure --prefix=/home/user/local --with-lpmd=/home/user/local
 make
 make install
 \end{verbatim}
\end{itemize}

De esta forma se generar\'an los esqueletos en \verb|/home/user/local| con \verb|bin|, \verb|lib|, etc.


%%%%%%%%%%%%%%%%%%%%%%%%%%%%%%%%%%%%%%%%%%%%%%%%%%%%%%%%%%%%%%%%%
%%%%%%%%%%%%%%%%%%%%%%%%%%%%%%%%%%%%%%%%%%%%%%%%%%%%%%%%%%%%%%%%%
%\subsection{Actualizando lpmd}

% {\lpmd} tiene m\'as de una forma de actualizarse, para eso en primer lugar debemos tener claro qu\'e versi\'on de la API posee {\lpmd} ya que las versiones nuevas de {\lpmd} o de los plugins dependen completamente de la API que tengamos.
% 
% Para ver la versi\'on que utiliza {\lpmd} ejecute : \verb|lpmd -h| donde mostrar\'a una l\'inea que contendr\'a el siguiente texto \textbf{Using liblmpd version X.X.X}.
% 
% Actualmente la API se encuentra en la versi\'on \textbf{1.0.0} esperamos que la pr\'oxima versi\'on soporte MPI para muchas fases de la din\'amica molecular.

