\documentclass[a4paper,10pt]{scrbook}

\usepackage{graphicx,fancybox,wrapfig}
\usepackage{fullpage}

%%%%%%%%%%%%%%%%%%%%%%%%%%%%%%%%%%%%%%%%%%%%%%%%%%%%%%%%%%%%%%%%%%%%%%%%%%%%%%%%%%%%%%%%%%%%%%%%%%%%%%%%%%%%%%%
%%NEWCOMMANDS DEFINITIONS%%%%%%%%%%%%%%%%%%%%%%%%%%%%%%%%%%%%%%%%%%%%%%%%%%%%%%%%%%%%%%%%%%%%%%%%%%%%%%%%%%%%%%
%%%%%%%%%%%%%%%%%%%%%%%%%%%%%%%%%%%%%%%%%%%%%%%%%%%%%%%%%%%%%%%%%%%%%%%%%%%%%%%%%%%%%%%%%%%%%%%%%%%%%%%%%%%%%%%
\newcommand{\lpmd}{\textbf{lpmd }}
\newcommand{\fumfun}[4]{\begin{center}\begin{tabular}{|c|c|c|c|}\hline
 Flag & Work & Test & Mand\\\hline
#1 & #2 & #3 & #4 \\\hline
\end{tabular}
\end{center}}
\newcommand{\D}[2]{\frac{\partial #2}{\partial #1}}
\newcommand{\DD}[2]{\frac{\partial^2 #2}{\partial #1^2}}
\newcommand{\DC}[2]{\frac{D #2}{D #1}} %derivada conectiva.
%\newcommand{\caja}[1]{\begin{center}\fbox{#1}\end{center}}
\newcommand{\cajatx}[1]{\begin{center}\setlength{\fboxsep}{0.2cm}\fbox{\parbox[l]{10cm}{#1}}\end{center}}
%\newcommand{\cvb}[1]{\begin{center}\begin{verbatim} #1 \end{verbatim}\end{center}}
\newcommand{\cajaeq}[1]{\begin{center}\setlength{\fboxsep}{0.1cm}\fbox{\parbox[b]{10cm}{#1}}\end{center}}
\newcommand{\cajafi}[3]{\begin{figure}\centering\shadowbox{\begin{minipage}{3.5 in}\centering\includegraphics[totalheight=2in]{#1}\caption{#2}\label{fig: #3}\end{minipage}}\end{figure}}
\newcommand{\control}[1]{\begin{center}\begin{minipage}{10cm}\texttt{#1}\end{minipage}\end{center}}


\newcommand {\foto}[4]{\begin{figure}
%   \begin{framed}
      \begin{center}
      \includegraphics[scale=#2]{#1}
      \end{center}

      \caption{\emph{#3}}

      \label{#4}
%   \end{framed}
   \end{figure}
   }

\begin{document}
\author{http://www.gnm.cl}
\title{``Las Palmeras'' Molecular Dynamics - \textbf{LPMD}.}
\maketitle

\tableofcontents
%%%%%%%%%%%%%%%%%%%%%%%%%%%%%%%%%%%%%%%%%%%%%%%%%%%%%%%%%%%%%%%%%%%%%%%%%%%%%%%%%%%%%%%%%%%%%%%%%%%
%%%%%%%%%%%%%%%%%%%%%%%%%%%%%%%%%%%%%%%%%%%%%%%%%%%%%%%%%%%%%%%%%%%%%%%%%%%%%%%%%%%%%%%%%%%%%%%%%%%
\chapter{El Programa}
\label{chap:lpmd}

\section{Origenes.}

El c\'odigo \lpmd comenz\'o a desarrollarse durante junio del a\~no 2007 como una idea de desarrollar e implementar din\'amica molecular de forma modular, tratando de ser lo m\'as amigable para el dise\~no de plugins y de actualizaciones. Los programadores principales de la primera etapa del c\'odigo fueron, Sergio Davis, Claudia Loyola y Joaqu\'in Peralta, todos ellos integrantes del \textit{Grupo de NanoMateriales} (\textbf{http://www.gnm.cl}). Actualmente existen m\'as colaboradores en el desarollo del c\'odigo (v\'ease \textbf{gente}).

Las versiones que se han lanzado en formato estable son:

\begin{itemize}
 \item Version 0.5.2 : Lanzada 30 de Junio 2008.
 \item Version 0.5.1 : Lanzada 30 de Abril 2008.
 \item Version 0.5.0 : Lanzada 4 de Abril 2008.
 \item Version 0.4.0 : Lanzada Julio 2007.
\end{itemize}

El sistema de actualizaci\'on va siendo administrado compleamente por los desarrolladores. A partir de la versi\'on 0.5 de \lpmd se cuenta con tres paquetes principales de desarrollo.

\begin{itemize}
 \item liblpmd \\
\textbf{API} principal de programaci\'on desarrollada por el grupo. Esta \textbf{API} es la m\'edula de todas las caracter\'isticas que presenta \lpmd. Por favor refierase a m\'as caracter\'isticas en el ap\'endice ~\ref{ap:API}
 \item lpmd-plugins \\
Es un set de plugins que han sido implementados, que cuentan con caracter\'isticas provenientes de la \textbf{API}, son en general potenciales intert\'omicos, integradores, analizadores de celdas de simulaci\'on, as\'i como tambi\'en manejadores de archivos de din\'amica molecular.
 \item lpmd \\
\lpmd es un software de din\'amica molecular que utiliza plugins implementados por usuarios para realizar todo el trabajo de simulaci\'on, tambi\'en incluye herramientas como \textbf{lpmd-analyzer} y \textbf{lpmd-converter} que son utilizados para la generaci\'on de configuraciones as\'i como tambi\'en el an\'alisis a partir de ficheros de simulaci\'on de din\'amica molecular.
\end{itemize}

\section{Idea Principal}

El esquema principal de \lpmd, es la utilizaci\'on de un archivo de control (\verb|fichero.control|) el c\'ual es cargado directamente por el ejecutable \verb|lpmd|.

Por ejemplo consideremos un sistema de din\'amica molecular, en el cual todas las opciones, tales como temperatura inicial, pasos de tiempo a simular, integrador, potencial interat\'omico, posiciones at\'omicas, etc. est\'an especificadas en un fichero llamado \verb|simulacion.control|, entonces lpmd puede ejecutarse simplemente con,

\begin{center}
 \texttt{lpmd simulacion.control > simulacion.output}
\end{center}

o bien, tambien podemos utilizar,

\begin{center}
 \texttt{lpmd < simulacion.control > simulacion.output}
\end{center}

De esta forma, \lpmd carga toda la informaci\'on necesaria ubicada en el fichero \verb|simulacion.control| y toda la informaci\'on de salida es guardada en el fichero \verb|simulacion.output|. Es importante destacar que \lpmd adem\'as interact\'ua y genera archivos de salida seg\'un los m\'odulos que hagan referencia a esos archivos.

\section{Otras Caracter\'isticas}

Otras caracter\'isticas de \lpmd es que cuenta con flags para el manejo de variables internas en un fichero de control, por ejemplo supongamos que queremos correr muchos sistemas con distintas temperaturas iniciales. De eta forma entonces consideremos, un fichero de control con una sintaxis como esta,

\begin{verbatim}
 ...
 prepare temperature t=TEMP
 ...
\end{verbatim}

entonces, podemos ejecutar \lpmd de la siguiente forma para asignar a \verb|TEMP| un valor real, dado directamente por linea de comandos, por ejemplo :

\begin{center}
 \texttt{lpmd -o TEMP=500 simulacion.control > simulacion.output}
\end{center}

de esta forma el sistema comienza su simulaci\'on con una temperatura inicial de 500K. Una de las ventajas de una caracter\'istica como \'esta es que reduce mucho la dificultad de implementar scripts para distintas simulaci\'ones o simulaciones consecutivas de din\'amica molecular. Por ejemplo con un solo comando podemos correr un set de simulaciones con un valor distinto asociado a una variable del archivo de control,

\begin{center}
 \texttt{for i in 300 400 500 ; \\do lpmd -o TEMP=\$i simulacion.control > simulacion-\$i.output ; done}
\end{center}

es decir tres simulaciones seguidas con diferentes temperaturas iniciales con un solo fichero de control.


Otro de los flags importantes de \lpmd es \verb|-p| que brinda informaci\'on acerca de los plugins que se encuentran instalados en el sistema y que identifica directamente \lpmd. Pruebe por ejemplo algo como :

\begin{center}
 \texttt{lpmd -p angdist}
\end{center}

donde \verb|angdist| corresponde al plugin que calcula la distribuci\'on angular de una celda de simulaci\'on. Para m\'as opciones vea \verb|lpmd -h| o \verb|man lpmd|. En los cap\'itulos siguientes se dar\'a una descripci\'on mucho m\'as detallada de \lpmd y sus funcionalidades.
%%%%%%%%%%%%%%%%%%%%%%%%%%%%%%%%%%%%%%%%%%%%%%%%%%%%%%%%%%%%%%%%%%%%%%%%%%%%%%%%%%%%%%%%%%%%%%%%%%%%%%%%%
%%%%%%%%%%%%%%%%%%%%%%%%%%%%%%%%%%%%%%%%%%%%%%%%%%%%%%%%%%%%%%%%%%%%%%%%%%%%%%%%%%%%%%%%%%%%%%%%%%%%%%%%%
%CAPITULO 2%%%%%%%%%%%%%%%%%%%%%%%%%%%%%%%%%%%%%%%%%%%%%%%%%%%%%%%%%%%%%%%%%%%%%%%%%%%%%%%%%%%%%%%%%%%%%%
%%%%%%%%%%%%%%%%%%%%%%%%%%%%%%%%%%%%%%%%%%%%%%%%%%%%%%%%%%%%%%%%%%%%%%%%%%%%%%%%%%%%%%%%%%%%%%%%%%%%%%%%%
%%%%%%%%%%%%%%%%%%%%%%%%%%%%%%%%%%%%%%%%%%%%%%%%%%%%%%%%%%%%%%%%%%%%%%%%%%%%%%%%%%%%%%%%%%%%%%%%%%%%%%%%%
\chapter{Instalaci\'on}
\label{chap:inst}

\lpmd ha sido probado en distintas arquitecturas, y compiladores, hasta ahora podemos \textbf{dar fe} que se ha logrado compilar el c\'odigo fuente en las siguientes arquitecturas.

\begin{itemize}
 \item Linux I686/AMD64
 \item OS/X
 \item Solaris
 \item Inclusive en Windows
\end{itemize}

\section{Descarga}

Para el correcto funcionamiento de \lpmd es necesario instalar previamente una librer\'ia y un set de plugins. Por lo que se requieren descargar 3 paquetes principales. La divisi\'on del c\'odigo en este set de paquetes se debe a la reutilizaci\'on de la \textbf{API} y los \textbf{plugins} para nuevos programas de utilidades o bien para un propio usuario interesado en la programaci\'on. Lo que adem\'as lleva una simplificaci\'on a aquellas personas que desean programar sus propios \textbf{plugins}.

%%%%%%%%%%%%%%%%%%%%%%%%%%%%%%%%%%%%%%%%%%%%%%%%%%%%%%%%%%%%%%%%%
%%%%%%%%%%%%%%%%%%%%%%%%%%%%%%%%%%%%%%%%%%%%%%%%%%%%%%%%%%%%%%%%%
\subsection{Descarga de versi\'on estable}

La \'ultima versi\'on estable de los paquetes es :

\begin{itemize}
 \item liblpmd : ver. 0.1.2
 \item plugins : ver. 0.5.2
 \item lpmd    : ver. 0.5.2
\end{itemize}

Los n\'umeros de los plugins indican la compatibilidad con las versiones de \lpmd y de la \textbf{API} liblpmd, de esta forma se mantiene un \textbf{orden} seg\'un la versi\'on que conocemos.

%%%%%%%%%%%%%%%%%%%%%%%%%%%%%%%%%%%%%%%%%%%%%%%%%%%%%%%%%%%%%%%%%
%%%%%%%%%%%%%%%%%%%%%%%%%%%%%%%%%%%%%%%%%%%%%%%%%%%%%%%%%%%%%%%%%
\subsection{Descarga de versi\'on en desarrollo}

Para los interesados en el desarrollo actual de \lpmd y sus dependencias principales, las pueden descargar con:

\begin{center}
 \begin{verbatim}
  svn co svn://www.gnm.cl/lpmd/liblpmd/testing liblpmd-uns
  svn co svn://www.gnm.cl/lpmd/plugins/testing plugins-uns
  svn co svn://www.gnm.cl/lpmd/lpmd/testing lpmd-uns
 \end{verbatim}
\end{center}

Est\'a disponible adem\'as una rama \verb|unstable|, sin embargo no recomendamos utilizarla para c\'alculos de Din\'amica Molecular, s\'olo para investigaci\'on y pruebas en c\'odigos.

\section{Instalaci\'on}
Antes de comenzar con la instalaci\'on de \lpmd es necesario instalar los 2 paquetes previos, \textbf{liblpmd} y \textbf{plugins}, a continuaci\'on se muestra una descripci\'on de c\'omo instalar cada uno de los paquetes.

\cajatx{Nota : Los que poseen la versi\'on testing/unstable cuentan con el fichero \textbf{autogen.sh} para poder generar los Makefiles. \\ Se necesita instalar automake y libtool para poder ejecutar \textbf{autogen.sh}.}

%%%%%%%%%%%%%%%%%%%%%%%%%%%%%%%%%%%%%%%%%%%%%%%%%%%%%%%%%%%%%%%%%
%%%%%%%%%%%%%%%%%%%%%%%%%%%%%%%%%%%%%%%%%%%%%%%%%%%%%%%%%%%%%%%%%
\subsection{Instalando liblpmd}

% \cajatx{Nota : La versi\'on inestable requiere que tenga instalando autmake y libtool. Para ejecutar autogen.sh}
% 
% En primer lugar debe tener instalado automake y libtool, si no lo tiene puede hacerlo como administrador de la siguiente manera:
% 
% \control{apt-get install automake libtool}

En primer lugar descomprima el paquete de la \textbf{liblpmd},

\control{tar -xvzf liblpmd-X.X.X.tar.gz}

lo que le generar\'a un nuevo directorio, para instalar esta librer\'ia con todos los m\'etodos necesarios para el funcionamiento de \lpmd y la implementaci\'on de plugins ejecute :

\control{./configure \\ make}

y como administrador,


\control{make install}


Por \textit{default} el directorio de instalaci\'on de la API programa es \verb|/usr/local/|, en caso de requerir una ubicaci\'on distinta revise las opciones con \verb|./configure --help|. Y si desea instalarlo en un directorio personal refierase a la secci\'on~\cite{subsub:personaldir}.

En caso de cualquier error env\'ie un e-mail a alguno de los desarrolladores principales o en su defecto a \verb|gnm@gnm.cl|.

%%%%%%%%%%%%%%%%%%%%%%%%%%%%%%%%%%%%%%%%%%%%%%%%%%%%%%%%%%%%%%%%%
%%%%%%%%%%%%%%%%%%%%%%%%%%%%%%%%%%%%%%%%%%%%%%%%%%%%%%%%%%%%%%%%%
\subsection{Instalando plugins}

Uno de los requerimientos b\'asicos de \lpmd es tener bien configurada la ubicaci\'on de los plugins que \lpmd requiere, es por eso que se debe utilizar la ubicaci\'on de la instalaci\'on de la librer\'ia lpmd \textbf{liblpmd}. Usualmente deber\'ia correrse sin ningun requerimiento especial si instal\'o la \textbf{liblpmd} en el lugar por \textit{default} (/usr/local), de no ser as\'i especifique el lugar con \verb|--with-lpmd=/ubicacion/|.

\control{./configure  \\ make}

y proceder la instalaci\'on como administrador:

\control{make install}

Esto ubicar\'a todos los plugins incluidos en el paquete \verb|plugins| en el directorio \verb|/usr/local/lib/lpmd|. De esta forma ya estamos listos para comenzar la instalaci\'on de lpmd y realizar las primeras pruebas.

%%%%%%%%%%%%%%%%%%%%%%%%%%%%%%%%%%%%%%%%%%%%%%%%%%%%%%%%%%%%%%%%%
%%%%%%%%%%%%%%%%%%%%%%%%%%%%%%%%%%%%%%%%%%%%%%%%%%%%%%%%%%%%%%%%%
\subsection{Instalando lpmd}

Es uno de los paquetes m\'as peque\~nos y r\'apidos de instalar, para proceder, se hace de manera similar que los anteriores, ejecutando :

\control{./configure \\ make}

y proceder a instalar como administrador:

\control{make install}

Esto generar\'a un set de ejecutables \verb|lpmd|, \verb|lpmd-analyzer| y \verb|lpmd-converter| en \verb|/usr/local/bin/|, que puede ser ejecutado desde cualquier sitio (si se presenta alg\'un problema, corriga su \verb|PATH|).

Puede correr lpmd con

\begin{verbatim}
username@machine:~$ lpmd
...
LPMD version 0.5.0
Using liblpmd version 1.0.0

Usage:lpmd [--verbose|-v] [--option|-o <option=value,option=value,...>] <file.control>
      lpmd [--pluginhelp | -p <pluginname>]
      lpmd [--help | -h ]
username@machine:~$
\end{verbatim}

\subsection{Problemas t\'ipicos post-instalaci\'on}

\subsubsection{Error cargando librer\'ia}

Es uno de los errores m\'as comunes luego de la instlaci\'on de \lpmd. Ocurre que al ejecutar \lpmd no muestra nada m\'as que un error referencial a la librer\'ia liblpmd que no puede ser encontrada.

La forma de corregir el problema es,

\begin{itemize}
 \item Editar /etc/ld.so.conf
 \item A\~nadir al archivo la linea /usr/local/lib (o donde se haya instalado liblpmd)
 \item Ejecutar como admininistrador el comando : ldconfig
\end{itemize}

Ahora deber\'ia ejecutar el comando sin problemas.

\subsection{Instalaci\'on de lpmd en directorio personal}
\label{subsub:personaldir}

Podemos istalar cada uno de los paquetes (\textbf{liblpmd}, \textbf{plugins} y \textbf{lpmd}) en un directorio personal, para eso consideremos un ejemplo, en el cu\'al deseamos instalar estos paquetes en el directorio \verb|loca;| ubicado dentro de el \textit{home} del usuario, el procedimiento ser\'ia.

\begin{itemize}
 \item Para liblpmd
 \begin{verbatim}
 ./configure --prefix=/home/user/local
 make
 make install
 \end{verbatim}
 \item Para plugins
 \begin{verbatim}
 ./configure --prefix=/home/user/local --with-lpmd=/home/user/local
 make
 make install
 \end{verbatim}
 \item Para lpmd, es necesario indicar con variables de ambiente para la compilaci\'on, note que \verb|\\| indica que es una sola l\'inea que contin\'ua.
 \begin{verbatim}
 LDFLAGS="-Wl,--rpath -Wl,/home/user/local/lib" \\
 ./configure --prefix=/home/user/local --with-lpmd=/home/user/local
 make
 make install
 \end{verbatim}
\end{itemize}

De esta forma se generaran los esqueletos en \verb|/home/user/local| con \verb|bin|, \verb|lib|, etc.


%%%%%%%%%%%%%%%%%%%%%%%%%%%%%%%%%%%%%%%%%%%%%%%%%%%%%%%%%%%%%%%%%
%%%%%%%%%%%%%%%%%%%%%%%%%%%%%%%%%%%%%%%%%%%%%%%%%%%%%%%%%%%%%%%%%
%\subsection{Actualizando lpmd}

% \lpmd tiene m\'as de una forma de actualizarse, para eso en primer lugar debemos tener claro qu\'e versi\'on de la API posee \lpmd ya que las versiones nuevas de \lpmd o de los plugins dependen completamente de la API que tengamos.
% 
% Para ver la versi\'on que utiliza \lpmd ejecute : \verb|lpmd -h| donde mostrar\'a una l\'inea que contendr\'a el siguiente texto \textbf{Using liblmpd version X.X.X}.
% 
% Actualmente la API se encuentra en la versi\'on \textbf{1.0.0} esperamos que la pr\'oxima versi\'on soporte MPI para muchas fases de la din\'amica molecular.


%%%%%%%%%%%%%%%%%%%%%%%%%%%%%%%%%%%%%%%%%%%%%%%%%%%%%%%%%%%%%%%%%%%%%%%%%%%%%%%%%%%%%%%%%%%%%%%%%%%%%%%%%
%%%%%%%%%%%%%%%%%%%%%%%%%%%%%%%%%%%%%%%%%%%%%%%%%%%%%%%%%%%%%%%%%%%%%%%%%%%%%%%%%%%%%%%%%%%%%%%%%%%%%%%%%
%CAPITULO 3%%%%%%%%%%%%%%%%%%%%%%%%%%%%%%%%%%%%%%%%%%%%%%%%%%%%%%%%%%%%%%%%%%%%%%%%%%%%%%%%%%%%%%%%%%%%%%
%%%%%%%%%%%%%%%%%%%%%%%%%%%%%%%%%%%%%%%%%%%%%%%%%%%%%%%%%%%%%%%%%%%%%%%%%%%%%%%%%%%%%%%%%%%%%%%%%%%%%%%%%
%%%%%%%%%%%%%%%%%%%%%%%%%%%%%%%%%%%%%%%%%%%%%%%%%%%%%%%%%%%%%%%%%%%%%%%%%%%%%%%%%%%%%%%%%%%%%%%%%%%%%%%%%
\chapter{El Fichero de Control}
\label{chap:input}

Una de las piezas fundamentales en \lpmd para la corrida de una simulaci\'on molecular son los ficheros iniciales de configuraci\'on del sistema. Para el correcto funcionamiento se necesita un fichero de sistema, este fichero espec\'ifica casi el 100\% de los requerimientos de la simulaci\'on y en ocasiones el 100\%. Aunque en la mayor\'ia de los casos se requiere un fichero adicional en donde se encuentran las posiciones at\'omicas de los \'atomos pertenecientes a la celda de simulaci\'on. Es por eso que en primer lugar revisaremos los archivos de posiciones at\'omicas antes de revisar el fichero de sistema.

\section{Fichero con Posiciones At\'omicas}

\lpmd puede manejar los tipos de fichero de posiciones at\'omicas, seg\'un los m\'odulos que dispongamos, actualmente \textbf{lpmd-plugins} cuenta con varios formatos de ficheros para especificar las posiciones at\'omicas, entre ellos \textbf{xyz} y \textbf{lpmd}.

Esperamos que los usuarios, seg\'un la necesidad, ayuden en implementar o requerir tipos espec\'ificos de sistemas de posici\'ones at\'omicas para din\'amica molecular. Veamos brevemente en que consisten algunos de ellos. Para informaci\'on espec\'ifica de cada m\'odulo, vea la secci\'on~\cite{chap:modulos:entradasalida}

%%%%%%%%%%%%%%%%%%%%%%%%%%%%%%%%%%%%%%%%%%%%%%%%%%%%%%%%%%%%%%%%%
%%%%%%%%%%%%%%%%%%%%%%%%%%%%%%%%%%%%%%%%%%%%%%%%%%%%%%%%%%%%%%%%%
\subsection{Fichero .xyz}

Es el est\'andar de ficheros \verb|xyz| utilizado en muchos c\'odigos de simulaci\'on computacional, es un archivo simple que cuenta con la informaci\'on de las posiciones at\'omicas del sistema en cordenadas cartesianas y sus unidades en \AA. La estructura de un fichero es :
\begin{center}
\begin{tabular}{l|l}
 \verb|N| & Especifica el n\'umero de \'atomos en la celda \\
 \verb|comment| & Una l\'inea adicional de comentario, t\'itulo etc. \\
 \verb|Sym X Y Z| & S\'imbolo at\'omico y las posiciones en coordenadas cartesianas. \\
\end{tabular}
\end{center}

%%%%%%%%%%%%%%%%%%%%%%%%%%%%%%%%%%%%%%%%%%%%%%%%%%%%%%%%%%%%%%%%%
%%%%%%%%%%%%%%%%%%%%%%%%%%%%%%%%%%%%%%%%%%%%%%%%%%%%%%%%%%%%%%%%%
\subsection{Fichero .lpmd}

Es un fichero con las posiciones escaladas de los \'atomos que forman la celda. Es de tipo ASCII y su estructura principal est\'a dada por:

\begin{center}
 \begin{tabular}{l|l}
 \verb|LPMD VERSION X.X | & Especifica la versi\'on del fichero \verb|.lpmd| \\
 \verb|cell properties | & Propiedades de la celda, pueden ser 3 vectores o largos y \'angulos. \\
 \verb|Sym sx sy sz| & S\'imbolo at\'omico y las posiciones escaladas en cada eje [0,1].\\
\end{tabular}
\end{center}

%%%%%%%%%%%%%%%%%%%%%%%%%%%%%%%%%%%%%%%%%%%%%%%%%%%%%%%%%%%%%%%%%
%%%%%%%%%%%%%%%%%%%%%%%%%%%%%%%%%%%%%%%%%%%%%%%%%%%%%%%%%%%%%%%%%
\subsection{Fichero .zlp}

Es un fichero con las posiciones escaladas de los \'atomos que forman la celda. A diferencia del formato \verb|lpmd| este formato es comprimido, utilizando zlib. Recomendamos lo utilize para obtenci\'on de datos muy largas o bien que cuentan con una cantidad de \'atomos muy grande.

%%%%%%%%%%%%%%%%%%%%%%%%%%%%%%%%%%%%%%%%%%%%%%%%%%%%%%%%%%%%%%%%%
%%%%%%%%%%%%%%%%%%%%%%%%%%%%%%%%%%%%%%%%%%%%%%%%%%%%%%%%%%%%%%%%%
\subsection{Otros Formatos Soportados}

Pese a que los formatos principales de entrada recomendados son \textbf{xyz} y \textbf{lpmd}, existen actualmente otros formatos que son soportados (de forma b\'asica) para lectura/escritura de configuraciones at\'omicas, algunos de ellos son:

\begin{tabular}{lcl}\\
 mol2 &:& M\'odulo para Leer/Escribir Configuraciones en formato \textbf{mol2} \\
 pdb  &:& M\'odulo para Leer/Escribir Configuraciones en formato \textbf{pdb} \\
\end{tabular}



\section{El Fichero de configuraci\'on .control}

Es el fichero principal para realizar la simulaci\'on computacional. Es por eso que en primer lugar se har\'a una descripci\'on general y luego veremos cada una de sus secciones principales.

Entre las cosas a considerar en un fichero de sistema, est\'an:

\begin{itemize}
 \item \# Es una l\'inea de comentario.
 \item Pese a que puede ser aleatorio el orden de los flags, se recomienda llevar un orden.
 \item Recomendamos definir los m\'odulos antes de utilizarlos. Tampoco es obligaci\'on.
 \item Existen secciones que \textbf{no} pueden ser omitidas, como por ejemplo, un \textbf{cellmanager}.
\end{itemize}

%%%%%%%%%%%%%%%%%%%%%%%%%%%%%%%%%%%%%%%%%%%%%%%%%%%%%%%%%%%%%%%%%
%%%%%%%%%%%%%%%%%%%%%%%%%%%%%%%%%%%%%%%%%%%%%%%%%%%%%%%%%%%%%%%%%
\subsection{Celda de Simulaci\'on.}

Son todas las propiedades que describen la celda de simulaci\'on. La mayor\'ia de las opciones del c\'odigo en esta parte est\'an descritas en \textbf{Propiedades Generales}. A continuaci\'on se describir\'a la propiedad que siempre debe estar presente en el fichero .control y va al comienzo de \'este.

\subsubsection{cell}

El flag cell es utilizado para describir la celda de simulaci\'on y asignar las propiedades de \'esta. Generalmente una celda de simulaci\'on puede venir descrita ya en el formato del fichero de entrada, sin embargo hay formatos, como el \textbf{xyz}, que no poseen la descripci\'on de la celda, es por eso que es necesario utilizar siempre esta opci\'on. Si despu\'es de dar la opci\'on \textbf{cell} se utiliza un formato de entrada descriptivo, como \textbf{lpmd}, este \'ultimo es el valor que toma la celda.

\begin{itemize} 
\item{Forma 1}

Se utilizan los largos y \'angulos de la celda, como sigue a continuaci\'on

\control{cell a=10 b=5 c=5 alpha=45 beta=90 gamma=90}

donde,

\cajatx{ 
\begin{tabular}{lcl}
 a & = & indica el largo de la celda en \textbf{a}.\\
 b & = & indica el largo de la celda en \textbf{b}.\\
 c & = & indica el largo de la celda en \textbf{c}.\\
 alpha & = & indica el \'angulo $\alpha$.\\
 beta & = & indica el \'angulo $\beta$.\\
 gamma & = & indica el \'angulo $\gamma$.\\
\end{tabular}
}

\item{Forma 2}

Se utilizan los 3 vectores bases, poni\'endolos de la siguiente manera en el fichero, note que \verb|//| indican la continuaci\'on de una l\'inea \'unica.

\control{cell ax=1.0 ay=0.0 az=0.0 bx=0.0 by=1.0 bz=0.0 // \\cx=0.0 cy=0.0 cz=1.0}
donde,
\cajatx{ 
 a${i}$, b${i}$ y c${i}$ con $i$={$x$,$y$,$z$} son las coordenadas $x, y$ y $z$ de los vectores bases.
}

Las posibles formas de ingresar una descripcion de la llamada \verb|cell| pueden ser entregadas como argumentos en la ejecuci\'on de \lpmd y no necesitan estar dentro del fichero de \textbf{control}, lo que ayuda a la creaci\'on de scrtips.

\cajatx{\texttt{lpmd -L a,b,c -A alpha,beta,gamma archivo.control} \\ \texttt{lpmd -V ax,ay,az,bx,by,bz,cx,cy,cz archivo.control}}

\end{itemize}

%%%%%%%%%%%%%%%%%%%%%%%%%%%%%%%%%%%%%%%%%%%%%%%%%%%%%%%%%%%%%%%%%
%%%%%%%%%%%%%%%%%%%%%%%%%%%%%%%%%%%%%%%%%%%%%%%%%%%%%%%%%%%%%%%%%
\subsection{Entrada - Salida}

\subsubsection{input}

Existen actualmente dos formas de ingreso de un sistema de entrada para la configuraci\'on at\'omica que se requiere simular, estas son, el ingreso de las posiciones at\'omicas de la celda a trav\'es de un archivo (por ejemplo \verb|.xyz| o \verb|.lpmd|) y el otro es mediante m\'odulos que generan automaticamente celdas at\'omicas con ciertas propiedades, por ejemplo celdas \textbf{bcc}, \textbf{fcc}, etc.

Veamos a continuaci\'on brevemente cada uno de ellos,

\begin{itemize}
 \item{Con Fichero}

Para cargar un fichero con configuraciones at\'omicas, es necesario la existencia del m\'odulo que reconoce el tipo de ficheros, por ejemplo para cargarun fichero del tipo \verb|.xyz|, es necesario cargar el m\'odulo \verb|xyz| para poder leer sin problemas el archivo. 
  \item{Generadores de celda}

A diferencia con el m\'etodo anterior, este m\'etodo no requiere de un fichero con posiciones at\'omicas, en lugar de ello se requiere un m\'odulo que genera automaticamente una celda con atomos, seg\'un los requerimientos propios del m\'odulo. Por ejemplo existen m\'odulos actualmente para genrar celdas del tipo \textbf{sc}, \textbf{bcc}, \textbf{fcc} y el m\'etodo \textbf{skewstart}.

\end{itemize}

La forma general de la orden \verb|input| requiere de argumentos para un funcionamiento adecuado. Para ver m\'as informaci\'on sobre el m\'odulo revise la secci\'on ~\ref{chap:modulos:entradasalida}. Estos son ejemplos de algunos argumentos:

\cajatx{ 
\begin{tabular}{lcl}
 module & = & indica el m\'odulo con el que cargar la celda.\\ 
 file & = & indica el fichero con posiciones at\'omicas.\\
 level & = & indica el nivel del fichero.\\ 
\end{tabular}
}

Estos son los argumntos m\'as standard ya que cada m\'odulo posee sus propios argumentos, por lo que se hace necesario ver cada uno seg\'un los inter\'eses propios.

Veamos algunas formas de uso para la orden \verb|input| :

\begin{itemize}
\item Carga desde un fichero XYZ.
\control{input module=xyz file=fichero.xyz level=0}
\item Inicializa una celda del tipo fcc
\control{input module=fcc a=1 nx=3 ny=3 nz=3}
\item Inicializa con metodo skewstart
\control{input module=skewstart atoms=108 symbol=Ar}
\end{itemize}

Esta es una lista de los m\'odulos soportados a la fecha para lectura/generaci\'on de configuraciones,

\begin{enumerate}
 \item \verb|xyz|
 \item \verb|lpmd|
 \item \verb|zlp|
 \item \verb|hcp| - \verb|fcc| - \verb|bcc| - \verb|sc|
 \item \verb|skewstart|
 \item \verb|mol2|
 \item \verb|pdb|
\end{enumerate}


\subsubsection{output}

Con el par\'ametro \verb|output| se especifican las opciones de salida de las configuraciones at\'omicas de nuestra simulaci\'on, los formatos de salidas son complementamente modulares y pueden ser implementados por los usuarios, sin embargo en la version 0.5.2 del set de plugins \verb|lpmd-plugins| ya se encuentran disponibles muchos m\'odulos, sin embargo es \textit{importante} notar que cada m\'odulo posee configuraciones independientes, por ejemplo \verb|level| es utilizado por m\'odulos como \textbf{xyz} o \textbf{zlp}, sin embargo no es requerido para \textbf{mol2}, para m\'as informaci\'on refierase a la secci\'on ~\ref{chap:modulos:entradasalida}. Los argumentos requeridos por el par\'ametro \verb|output| son:

\cajatx{ 
\begin{tabular}{lcl}
 module & = & indica el m\'odulo (formato) de \\
&&salida de la simulaci\'on.\\
 file & = & indica el fichero en el que graba.\\
 level & = & indica el nivel del modulo de salida.\\
 each & = & indica cada cuantos pasos la celda \\
&&es gabada en el fichero.\\
\end{tabular}
}

Al igual que antes, existen m\'as par\'ametros que son independientes de cada m\'odulo.

Algunas formas de uso,

\begin{itemize}
 \item Grabando la simulaci\'on en un fichero XYZ (nivel 0), cada 20 steps.
\control{output module=xyz file=fichero.xyz level=0 each=20}
 \item Grabando la simulaci\'on en fichero LPMD (nivel 1), cada 1 step.
\control{output module=lpmd file=fichero.xyz level=1 each=1}
\end{itemize}

Esta es una lista de los m\'odulos soportados a la fecha para escritura de configuraciones,

\begin{enumerate}
 \item \verb|xyz|
 \item \verb|lpmd|
 \item \verb|zlp|
 \item \verb|mol2|
 \item \verb|pdb|
\end{enumerate}


\subsubsection{restore}

Es utilizado para restaurar una simulaci\'on a partir de un punto en que se produjo un corte de luz o cualquier otro tipo de falla f\'isica en un centro de c\'alculo. El punto de restauraci\'on es a partir de el \'ultimo dumping realizado por la simulacion, dado por la orden ``dumping''.

Actualmente no se han realizado pruebas exhaustivas de este punto, pero deber\'ia funcionar sin problemas, porfavor si encuentra alg\'un bug, reportelo.

%%%%%%%%%%%%%%%%%%%%%%%%%%%%%%%%%%%%%%%%%%%%%%%%%%%%%%%%%%%%%%%%%
%%%%%%%%%%%%%%%%%%%%%%%%%%%%%%%%%%%%%%%%%%%%%%%%%%%%%%%%%%%%%%%%%
\subsection{Propiedades Generales}
\subsubsection{prepare}
Esta opci\'on es utilizada para \textit{setear} valores y caracter\'isticas de la simulaci\'on, que son brindadas a trav\'es de plugins, tenemos por ejemplo :

\begin{itemize}
 \item \textbf{temperature}
Para dar una temperatura inicial al sistema, se prepara la celda con :
\control{prepare temperature T=300}
de esta forma el sistema asigna velocidades iniciales a las part\'iculas para que la temperatura de nuestro sistema corresponda a 300K.
 \item \textbf{replicate}
Para replicar nuestra celda en las distintas direcciones de los vectores bases, la forma de hacerlo para una celda antes de comenzar la simulaci\'on es:
\control{prepare replicate nx=2 ny=2 nz=2}
De esta froma, la celda que se ley\'o en \verb|input| es replicada 2 veces por cada eje, alcanzando 8 veces el n\'umero inicial de part\'iculas.
\end{itemize}

\subsubsection{set}
Utilizado para setear valores de la simulaci\'on, utilizado principalmente para algunas variables globales del sistema. A continuaci\'on los m\'as utilizados, ya que la mayor\'ia de las variables pueden ser modificadas utilizando \verb|set|.

\begin{itemize}
 \item Evitando que aparezca la celda inicial en pantalla
\control{set showcoords false}
 \item Evitando que se muestren los modulos no utilizados durante la salida.
\control{set showunused false}
 \item Utilizando el cache de distancias en la simulaci\'on
\control{set distancecache true}
\end{itemize}

\subsubsection{charge}
Set de las cargas en eV para las especies at\'omicas. Estos valores de las cargas, son seteados principalmente para utilizaci\'on de potenciales interatomicos en los cuales se utilizan las cargas de los atomos involucrados.


Forma de uso

\begin{itemize}
 \item Seteando las cargas de los atomos de O y Ge.
\control{charge O=XX \\ charge Ge=XX}
\end{itemize}

\subsubsection{periodic}
Indica la periodicidad de la celda, en cada eje. Al bloquear la periodicidad en un eje, este se ve ``modificado'' en ambos lados de la celda, revise con cuidado estas opciones.

\control{periodic false false true}

En \'este caso s\'olo tenemos periodicidad en el eje \verb|z|.

\subsubsection{steps}
N\'umero de pasos de la simulaci\'on de DM.

\control{steps 10000}

Ac\'a se indica que la siulaci\'on se realizar\'a con 10000 pasos.

\subsubsection{dumping}
Genera una salida global del sistema para poder restaurar a partir de ese punto.

\control{dumping file=rescue.dump each=10000}

Generamos un fichero de volcado cada 10000 pasos de la simulaci\'on, en \'el se graba toda la informaci\'on necesaria, para reiniciar una corrida.

\subsubsection{monitor}
Cada cuantos pasos la simulaci\'on muestra las propiedades globales. Estas propiedades, pueden ser asignadas por el mismo usuario, haciendo la salida lo m\'as configurables seg\'un los requerimientos propios.

Entre las opciones de monitor, cuentan :

\cajatx{
\begin{tabular}{lcl}
 start & = & indica el valor de epsilon.\\
 end & = & indica el valor de sigma.\\
 each & = & indica el cutoff del potencial.\\
 properties & = & indica que valores se desea monitorear. \\
 output & = & archivo de salida para guardar los valores, \\
 & & si no, el \textit{standard output} es utilziado.\\
\end{tabular}
}

Si queremos ir chequeando, los valores de la energ\'ia durante la simulacion, cada 10 pasos, utilizamos la l\'inea, (note que \verb|\\| indica que la l\'inea contin\'ua)

\begin{verbatim}
monitor start=0 end=1000 each=10 properties=kinetik-energy, \\
        potential-energy,total-energy output=salida.out
\end{verbatim}

en \'este caso, no es necesario cargar los modulos \verb|energy| y \verb|cell| ya que son cargados por \textit{default}, sin embargo para cuando necesitemos ver la presi\'on durante la simulaci\'on, es necesario ingresar en el fichero de control el uso del m\'odulo \verb|pressure| ya que \'el es el encargado de mostrar y calcular la presi\'on.

\begin{verbatim}
use pressure
enduse
monitor start=0 end=1000 each=10 properties=kinetik-energy, \\
        potential-energy,total-energy,total-pressure output=salida.out
\end{verbatim}

Algunas de las opciones soportadas por \textbf{properties} son:

\begin{itemize}
 \item kinetic-energy : Muestra la energ\'ia cin\'etica.
 \item potential-energy : Muestra la energ\'ia potencial.
 \item total-energy : Muestra la energ\'ia total.
 \item temperature : Muestra la temperatura del sistema.
 \item virial-pressure : Aporte del t\'ermino del virial a la presi\'on.
 \item kinetic-pressure : T\'ermino de la presi\'on asociado.
 \item pressure : Presi\'on total del sistema.
 \item volume : Volumen de la celda de simulaci\'on.
 \item cell-x : x=a,b,c son los largos de la celda en cada eje.
 \item sij : i,j=x,y,z entrega los valores para el tensor de stress.
\end{itemize}

Una ventaja muy considerable es la utilizaci\'onde multiples \verb|monitor| para as\'i ir almacenando la informaci\'on \textit{a medida}. Para revisar todas las opciones de salida, refirase a la secci\'on~\ref{chap:modulos}.

\subsection{Carga de M\'odulos}

Ac\'a mostraremos c\'omo se utilizan en general la carga de m\'odulos dentro de un fichero de control. Los m\'odulos o plugins pueden ser cargados en cualquier secci\'on del fichero de control, sin embargo recomendamos hacerlo de forma ordenada como veremos en los ejemplos posteriores.

\subsubsection{C\'omo cargar un m\'odulo}

Los m\'odulos son de distintos tipos en general, y los de un tipo en com\'un comparten ``ciertas'' caracter\'isticas ya que cada uno posee sus propias ventajas. Una visi\'on general de c\'omo se han distribuido los m\'odulos es: 

\begin{tabular}{lcl}
 Generadores de celda & : & Tales como \verb|xyz|, \verb|pdb|, \verb|crystalfcc|, etc. \\
 Manejadores de celda & : &\verb|minimumimage|, \verb|linkedcell|, etc. \\
 Calculadores de Propiedades & : & \verb|gdr|, \verb|angdist|, \verb|msd|, etc. \\
 Integradores & : & \verb|verlet|, \verb|euler|, etc. \\
 Potenciales & : & \verb|LennardJones|, \verb|SuttonChen|, \verb|Morse|, etc. \\
 Informativos & : & \verb|cell|, \verb|pressure|, \verb|Energy|, etc.
\end{tabular}

En general, siempre a un m\'odulo se le puede asignar un \textbf{alias} para su posterior llamado, por ejemplo.

\control{use MODULO as ALIAS \\ ... \\enduse}

De esta forma el m\'odulo \texttt{MODULO} puede ser llamado en forma posterior con el nombre \texttt{ALIAS}, lo que da una ventaja para reutilizar, combinar y simplificar la utilizaci\'on de m\'odulos.

Entonces dentro de un archivo de \textbf{control}, debemos cargar los m\'odulos necesarios para un posterior llamado.

\subsection{Aplicac\'on de m\'odulos}

Los m\'odulos son llamados en la parte final de nuestro fichero de \textbf{control}, como existen distintas ``especies'' de m\'odulos, estos deben ser llamados de diferentes maneras, aunque su forma es muy general.

\begin{itemize}
 \item \textbf{Calculadores de Propiedades}
  Estos siempre son llamados a ser evaluados cada cierto tiempo entre un rango de intervalos, 
  \control{property module-alias start=0 end=1000 each=10}
 \item \textbf{Integradores}
  Estos son llamados facilmente con
  \control{integrator module-alias}
 \item \textbf{Potenciales}
  Estos definen una interaccion entre dos \'atomos (por el momento no hay de tres cuerpos). Se crea un potencial para cada interacci\'on especificando los valores y llamando luego con
  \control{potential module-alias Pt Au}
 \item \textbf{Manejadores de celda}
  Llaman al manejador de celda que se utilizara en la simulaci\'on, en nuestro caso hay dsiponibilidad de dos manejadores, \verb|linkdecell| y \verb|minimumimage|.
  \control{cellmanager linkedcell}
\end{itemize}




\section{Ficheros de Salida}

%%%%%%%%%%%%%%%%%%%%%%%%%%%%%%%%%%%%%%%%%%%%%%%%%%%%%%%%%%%%%%%%%
%%%%%%%%%%%%%%%%%%%%%%%%%%%%%%%%%%%%%%%%%%%%%%%%%%%%%%%%%%%%%%%%%
\subsection{Ficheros de salida}
\lpmd Tiene dos tipos de fichero de salida, uno que es generado usualmente por la opci\'on \verb|output| dentro del fichero de control y otro es la salida standard, que por defecto va a \verb|cout|, pero que nosotros recomendamos enviar (o redireccionar) a un fichero.

%%%%%%%%%%%%%%%%%%%%%%%%%%%%%%%%%%%%%%%%%%%%%%%%%%%%%%%%%%%%%%%%%
%%%%%%%%%%%%%%%%%%%%%%%%%%%%%%%%%%%%%%%%%%%%%%%%%%%%%%%%%%%%%%%%%
\subsubsection{Salida standard}
Esta es la salida que muestra en pantalla lpmd, la forma de enviar esta salida a un fichero, durante la ejecuci\'on de lpmd, es

\control{lpmd fichero.control > salida.out}

o si desea verla en pantalla y enviar a un fichero:

\control{lpmd fichero.control | tee salida.out}

tambi\'en puede redireccionar una salida standard y los mensajes de error de forma independiente utilizando:

\control{lpmd fichero.control 1\&> salida.out 2\&> salida.err}

el fichero \verb|salida.out| tiene toda la informaci\'on que deb\'ia salir a pantalla utilizando lpmd, entre ella se encuentran,

\begin{itemize}
 \item Descripci\'on completa de la celda
 \item Informacion de \verb|startinfo|
 \item Informaci\'on de m\'odulos utilizados y variables de c/u.
 \item Energ\'ias, Temperatura, Presi\'on y Vol\'umen, seg\'un \verb|monitor|. En caso de que monitor no utilize \verb|output| propio.
\end{itemize}


%%%%%%%%%%%%%%%%%%%%%%%%%%%%%%%%%%%%%%%%%%%%%%%%%%%%%%%%%%%%%%%%%
%%%%%%%%%%%%%%%%%%%%%%%%%%%%%%%%%%%%%%%%%%%%%%%%%%%%%%%%%%%%%%%%%
\subsubsection{Fichero generado por output}
Este fichero se gener\'o con el nombre entregado en el fichero de control a la linea \textbf{output}, en \'el se encuentran las configuraciones atomicas de la \textbf{DM} y suelen ser los ficheros a partir de los cuales suelen crearse animaciones y an\'alisis detallados de la simulaci\'on.

Muchos an\'alisis pueden llevarse a cabo \textit{\textbf{durante la simulaci\'on misma}}, sin embargo \lpmd tambi\'en cuenta con herramientas propias de an\'alisis como se puede ver en la secci\'on~\ref{chap:utilidades}

Es importante definir un buen formato de salida ya que es la clave para simplificar o dificultar los an\'alisis posteriores. Por ejemplo si se adquiere un archivo \verb|xyz| con \verb|level=0| no se podr\'a calcular un perfil de temperaturas ya que no se cuenta con la informaci\'on suficiente (posiciones y velocidades).


%%%%%%%%%%%%%%%%%%%%%%%%%%%%%%%%%%%%%%%%%%%%%%%%%%%%%%%%%%%%%%%%%
%%%%%%%%%%%%%%%%%%%%%%%%%%%%%%%%%%%%%%%%%%%%%%%%%%%%%%%%%%%%%%%%%
\subsection{Salida de errores}
La versi\'on actual no redrecciona informacion a la salida de errores, salvo los que realmente corresponden a errores de ejecuci\'on de \lpmd.

Para aislar estos errores, ejecute \lpmd como ya se indico:

\control{lpmd fichero.control 1\&> salida.out 2\&> salida.err}


%%%%%%%%%%%%%%%%%%%%%%%%%%%%%%%%%%%%%%%%%%%%%%%%%%%%%%%%%%%%%%%%%%%%%%%%%%%%%%%%%%%%%%%%%%%%%%%%%%%%%%%%%
%%%%%%%%%%%%%%%%%%%%%%%%%%%%%%%%%%%%%%%%%%%%%%%%%%%%%%%%%%%%%%%%%%%%%%%%%%%%%%%%%%%%%%%%%%%%%%%%%%%%%%%%%
%CAPITULO 4%%%%%%%%%%%%%%%%%%%%%%%%%%%%%%%%%%%%%%%%%%%%%%%%%%%%%%%%%%%%%%%%%%%%%%%%%%%%%%%%%%%%%%%%%%%%%%
%%%%%%%%%%%%%%%%%%%%%%%%%%%%%%%%%%%%%%%%%%%%%%%%%%%%%%%%%%%%%%%%%%%%%%%%%%%%%%%%%%%%%%%%%%%%%%%%%%%%%%%%%
%%%%%%%%%%%%%%%%%%%%%%%%%%%%%%%%%%%%%%%%%%%%%%%%%%%%%%%%%%%%%%%%%%%%%%%%%%%%%%%%%%%%%%%%%%%%%%%%%%%%%%%%%
\chapter{M\'odulos}
\label{chap:modulos}

%%%%%%%%%%%%%%%%%%%%%%%%%%%%%%%%%%%%%%%%%%%%%%%%%%%%%%%%%%%%%%%%%
%%%%%%%%%%%%%%%%%%%%%%%%%%%%%%%%%%%%%%%%%%%%%%%%%%%%%%%%%%%%%%%%%
\section{Propiedades del sistema}
Estos m\'odulos corresponden principalmente a \textbf{informaci\'on} del set de part\'iculas con el cu\'al se esta trabajando, as\'i como tambi\'en a aquellos m\'odulos que tienen la capacidad de \textbf{modificar} y \textbf{manejar} este set, o alguna propiedad de ellas.
\subsection{cell}
Devuelve informaci\'on del sistema, se utiliza principalmente en \verb|monitor| para entregar esta informaci\'on. \textbf{NO} es necesario cargar este m\'odulo ya que se carga por \textit{default} al correr \lpmd.

\cajatx{
\begin{tabular}{lcl}
 volume & = & Retorna el volumen de la celda.\\
 density & = & Retorna la densidad de la celda.\\
 cell-n & = & n=a,b,c Retorna el largo de cada eje.\\
 particledensity & = & Retorna la densidad por particula. \\
 volumeperatom & = & Retornael volumen por \'atomo. \\
\end{tabular}
}

\subsection{energy}
Entrega informaci\'on sobre la energ\'ia en el sistema, al igual que \verb|cell|, \textbf{no} es necesario cargar el m\'odulo \verb|energy| ya que es cargao autom\'aticamente al correr \lpmd. La informaci\'on que entregua el m\'odulo a trav\'es de \verb|monitor| es:

\cajatx{
\begin{tabular}{lcl}
 kinetic-energy & = & Retorna el valor de la Energ\'ia cin\'atica [eV].\\
 potential-energy & = & Retorna la energ\'ia potencial [eV].\\
 total-energy & = & Retorna la energ\'ia total [eV].\\
 momentum & = & Retorna el momentum [?]. \\
 p-n & = & n=x,y,z Retorna el momentum por eje. \\
 temperature & = & Retorna el valor de la temperatura [K]
\end{tabular}
}

\subsection{pressure}

Entrega informaci\'on sobre la presi\'on en el sistema y el stress. Este m\'odulo \textbf{SI} es necesario cargarlo antes de llamar a la informaci\'on requerida. Basta llamarlo con,

\control{use pressure \\enduse}

para luego utilizar en \verb|monitor| los llamados posibles de pressure. Es recomendable que primero ``cargue'' el m\'odulo \verb|pressure| y luego especif\'ique el monitor que har\'a menci\'on a alguna de sus propiedades.

\cajatx{
\begin{tabular}{lcl}
 pressure & = & Retorna el valor de la Presi\'on [MPa].\\
 virial-pressure & = & Retorna la contribuci\'on virial a la presi\'on [MPa].\\
 kinetic-pressure & = & Retorna la contribuci\'on cin\'etica a la presi\'on [MPa].\\
 sij & = & i,j=x,y,z Retorna los valores del tensor de stress[?]. \\
\end{tabular}
}

%%%%%%%%%%%%%%%%%%%%%%%%%%%%%%%%%%%%%%%%%%%%%%%%%%%%%%%%%%%%%%%%%
%%%%%%%%%%%%%%%%%%%%%%%%%%%%%%%%%%%%%%%%%%%%%%%%%%%%%%%%%%%%%%%%%
\section{M\'odulos Entrada/Salida}
\label{chap:modulos:entradasalida}
\subsection{crystalfcc}
Genera una celda cristalina del tipo fcc.
\subsection{crystalsc}
Genera una celda cstalino del tipo sc.
\subsection{lpmd}
M\'odulo lectura/escritura de formato lpmd.
\subsection{xyz}
M\'odulo lectura/escrtura de formato xyz.
\subsection{skewstart}
Genera una celda con m\'etodo SkewStart (Refson).
%%%%%%%%%%%%%%%%%%%%%%%%%%%%%%%%%%%%%%%%%%%%%%%%%%%%%%%%%%%%%%%%%
%%%%%%%%%%%%%%%%%%%%%%%%%%%%%%%%%%%%%%%%%%%%%%%%%%%%%%%%%%%%%%%%%
\section{Modificadores}
\subsection{tempscaling}

Utilizado para escalar la temperatura de la muestra, utilizando rescalamiento de velocidades en las part\'iculas, 

\subsection{berendsen}
Termostato de berendsen, mucho m\'as suave que tempscaling.
\subsection{cellscaling}
Escala la celda cierto porcentaje, puede ser por eje o total.

%%%%%%%%%%%%%%%%%%%%%%%%%%%%%%%%%%%%%%%%%%%%%%%%%%%%%%%%%%%%%%%%%
%%%%%%%%%%%%%%%%%%%%%%%%%%%%%%%%%%%%%%%%%%%%%%%%%%%%%%%%%%%%%%%%%
\section{Manejadores de Celda}
Especifica el manejador de celda, actualmente hay dos m\'odulos manejadores de celda, minima imagen y listas linkeadas. cada m\'odulo debe declararse previemente con los parametros requeridos.
\subsection{minimumimage}
\subsection{linkedcell}
%%%%%%%%%%%%%%%%%%%%%%%%%%%%%%%%%%%%%%%%%%%%%%%%%%%%%%%%%%%%%%%%%
%%%%%%%%%%%%%%%%%%%%%%%%%%%%%%%%%%%%%%%%%%%%%%%%%%%%%%%%%%%%%%%%%
\section{Potenciales Interat\'omicos de pares}
\subsection{lennardjones}
El m\'odulo \textbf{lennardjones} hace referencia al potencial de Lennard-Jones, que es de la forma,
$$U(r_{ij}) = 4\epsilon\left(\left(\frac{\sigma}{r_{ij}}\right)^{12}-\left(\frac{\sigma}{r_{ij}}\right)^6\right)$$
En donde $r_{ij}$ es la distancia interat\'omica de los \'atomos $i$ y $j$. La implementaci\'on del m\'etodo virtual, queda entonces como :
\begin{verbatim}
double LennardJones::pairEnergy(const double & r) const
{
 double rtmp=sigma/r;
 double r6 = rtmp*rtmp*rtmp*rtmp*rtmp*rtmp;
 double r12 = r6*r6;
 return 4.0e0*epsilon*(r12 - r6);
}
\end{verbatim}

Para el c\'alculo de Fuerzas, la forma del potencial que nos interesa, es aquella fuerza que siente el \'atomo $i$ producida por el \'atomo $j$, la que debe ser implementada en el plugin, para potentiales de pares, para el caso del potencial de Lennard Jones, la fuerza esta dada por,

$$F_{ij} = \frac{-48.0\epsilon}{r_{ij}^2}\left( \left(\frac{\sigma}{r_{ij}}\right)^{12} + \frac{1}{2}\left(\frac{\sigma}{r_{ij}}\right)^6 \right) \vec{r_{ij}}$$

en donde $\vec{r_{ij}}$ es el vector distancia entre los \'atomos $i$ y $j$, y $r_{ij}$ es la distancia entre ellos. La implementaci\'on de la fuerza de pares requerida por la API es :

\begin{verbatim}
Vector LennardJones::pairForce(const Vector & r) const
{
 double rr2 = r.Mod2();
 double r6 = pow(sigma*sigma / rr2, 3.0e0);
 double r12 = r6*r6;
 double ff = -48.0e0*(epsilon/rr2)*(r12 - 0.50e0*r6);
 Vector fv = r;
 fv.Scale(ff);
 return fv;
}
\end{verbatim}

Las palbras reservadas por el plugin \textbf{lennardjones}, son :

\cajatx{
\begin{tabular}{lcl}
 epsilon & = & indica el valor de epsilon.\\
 sigma & = & indica el valor de sigma.\\
 cutoff & = & indica el cutoff del potencial.\\
\end{tabular}
}

Las unidades en que deben ser ingresados, las constantes, deben ser basadas en que las distancias estan en [\AA] y la energ\'ia debe ser adquirida en [eV].

\cajatx{A continuaci\'on, los otros \textbf{potenciales interat\'omicos de pares} son explicados brevemente, pero el trasfondo es similar al ya planteado hasta ahora.}

\subsection{fastlj}
Utiliza Potencial de LJ tabulado. De forma similar al m\'odulo anterior pero porcentualmente m\'as r\'apido. Es recomendable utilizarlo para sistemas con grn n\'umero de part\'iculas.

\subsection{morse}
Utiliza Potencial de Morse para la interacci\'on de las especies at\'omicas. La energ\'ia que siente una part\'icula $i$ a causa de la precensia de otra part\'icula $j$, si ambas interactuan con un potencial de este estilo, esta dada por :

$$E(\vec{r}_{ij}) = D_e\left(1-\exp(-a(\vec{r}_{ij}-\vec{r}_e))\right)^2$$

en donde $D_e$ es la profundidad del pozo, $a$ es el ancho del pozi y $r_e$ es la distancia en equilibrio. Y entonces, la fuerza que siente un \'atomo $i$ producto de otro \'atomo $j$ est\'a dada por,

$$\vec{F}_{ij} ( \vec{r}_{ij}) = 2aD_e\exp(-a(\vec{r}_{ij}-\vec{r}_e))\left(1-\exp(-a(\vec{r}_{ij}-\vec{r}_e))\right)\frac{\vec{r}}{|\vec{r}|}$$

Las palabras reservadas por el plugin \textbf{morse}, son :

\cajatx{
\begin{tabular}{lcl}
 depth & = & indica el valor de profundidad del pozo.\\
 a & = & indica el valor del ancho del pozo.\\
 re & = & indica el largo de enlaze en equilibrio.\\
 cutoff & = & indica el cutoff del potencial.\\
\end{tabular}
}


\subsection{constantforce}
Mantiene una fuerza constante sobre atomos de cierta especie, se utiliza principalmente para aplicarles fuerzas a especies at\'omicas o bien a atomos seleccionados de alguna forma en especial. Este \textit{potencial}, no retorna una energ\'ia (cero) y solo tiene capacidad de asginar una fuerza constante a un set de atomos.

Por ejemplo si queremos que algunos \'atomos se vean afectados por la gravedad, podr\'iamos tener algo como :

\control{use constantforce as CF \\   forcevector 0.0 0.0 -9.8 \\enduse}

Las palabras reservadas por el plugin \textbf{constantforce}, son :

\cajatx{
\begin{tabular}{lcl}
 forcevector & = & indica de la fuerza constante a aplicarse.\\
\end{tabular}
}

\subsection{harmonic}
Potencial arm\'onico entre especies at\'omicas. De manera similar a un potencial de morse, tenemos que la energ\'ia que sienten las part\'iculas $i$ y $j$ a causa de la interacci\'on a trav\'es de \'este potencial es,

$$E(\vec{r}_{ij}) = \frac{1}{2}k\left(|\vec{r}_{ij}|-a\right)^2$$

En donde $k$ es la constante de elasticidad y $a$ la separaci\'on de equilibrio. Con esto la fuerza para el potencial arm\'onico esta dada por,

$$\vec{F}(\vec{r}_{ij}) = \frac{k}{|\vec{r}_{ij}|}\left(|\vec{r}_{ij}|-a\right)$$

Las palabras reservadas por el plugin \textbf{harmonic}, son :

\cajatx{
\begin{tabular}{lcl}
 k & = & indica el valor de la constante de elasticidad.\\
 a & = & indica el valor de el largo de equilibrio.\\
 cutoff & = & indica el cutoff del potencial.\\
\end{tabular}
}

\subsection{buckingham}
\cajatx{Buckingham, no incluye directamente parte coulombiana, para ello es necesario a\~nadir como un potencial adicional a ewald u otro similar que a\~nada la parte coulombiana.}

Est\'e m\'odulo especif\'ica la interacci\'on de buckingham entre los \'atomos, de esta forma, la energ\'ia producida por la interacci\'on de dos part\'iculas $i$ y $j$, queda

$$E(\vec{r}_{ij}) = B1 \exp\left(-\frac{|\vec{r}_{ij}|}{\rho}\right) - \frac{B2}{(|\vec{r}_{ij}|)^6}$$

y la fuerza,

$$\vec{F}(\vec{r}_{ij}) = -\frac{B1\exp\left(-\frac{|\vec{r}_{ij}|}{\rho}\right)}{|\vec{r}_{ij}|\rho}\vec{r}_{ij} + \frac{6B2}{|\vec{r}_{ij}|^8}\vec{r}_{ij}$$

Las palabras reservadas por el plugin \textbf{buckingham}, son :

\cajatx{
\begin{tabular}{lcl}
 B1 & = & indica el valor de la constante B1.\\
 B2 & = & indica el valor de la constante B2.\\
 Ro & = & el valor de rho para el potencial.\\
 cutoff & = & indica el cutoff del potencial.\\
\end{tabular}
}

%%%%%%%%%%%%%%%%%%%%%%%%%%%%%%%%%%%%%%%%%%%%%%%%%%%%%%%%%%%%%%%%%
%%%%%%%%%%%%%%%%%%%%%%%%%%%%%%%%%%%%%%%%%%%%%%%%%%%%%%%%%%%%%%%%%
\section{Potenciales Interat\'omicos Met\'alicos}
\subsection{suttonchen}
Este potencial, se utiliza para interacciones de atomos met\'alicos, es por eso que el plugin \textbf{suttonchen} implementa los m\'etodos virtuales de \verb|metalpotential|, que cuentan con una parte de pares y otro t\'ermino de muchos cuerpos. La parte asociada al t\'ermino de pares, est\'a dado por,

$$U(r_{ij}) = \left(\frac{a}{r_{ij}}\right)^n$$

en donde $r_ij$ es la distancia entre un atomo $i$ y otro atomo $j$ del sistema. El t\'ermino de mcuhos cuerpos esta dado por

$$F(\rho_{i}) = -c\epsilon\sqrt{\rho_i}$$

en donde,

$$\rho_i = \sum_{j\neq i} \left(\frac{a}{r_{ij}}\right)^m$$

lo que corresponde a una densidad local del atomo $i$, que depende de todos los atomos $j$ cercanos a \'el, \'esta densidad local sin embargo, debe ser corregida para el caso de suttonchen (note que no todos los potenciales asociados a los metales requieren de esta correcci\'on, pero \verb|metalpotential| lo requiere, as\'i que en ocaciones debe ser cero).

Para el potencial de SuttonChen, la correcci\'on de la densidad esta dada por,

$$\delta\rho_i=\frac{4\pi\overline{\rho}a^3}{m-3}\left(\frac{a}{r_{met}}\right)^{(m-3)}$$

Esta correcci\'on de la densidad debe ser aplicada inmediatamente luego de ser calculada la densidad local. La correcci\'on de la energ\'ia para Sutton Chen, se obtiene de esta forma con :

$$\delta U_1 = \frac{2\pi N\overline{\rho}\epsilon a^3}{n-3}\left(\frac{a}{r_{met}}\right)^{n-3}$$

Hay que notar que $\delta U_2$ no es requerido si $\rho_i$ ya fue corregido, con $\delta U_2$ de la forma

$$\delta U_2 = -\frac{4\pi\overline{\rho}a^3}{m-3}\left(\frac{a}{r_{met}}\right)^{n-3}\left<\frac{Nc\epsilon}{2\sqrt{\rho_i^0}}\right>$$

La implementaci\'on de esta energ\'ia para el potencial met\'alico, 

\begin{verbatim}
double SuttonChen::pairEnergy(const double &r) const
{
 return e*pow((a/r),n);
}

double SuttonChen::rhoij(const double &r) const
{
 return pow((a/r),m);
}

double SuttonChen::F(const double &rhoi) const
{
 return -c*e*sqrt(rhoi);
}

double SuttonChen::deltarhoi(const double &rhobar) const
{
 return (4*M_PI*rhobar*a*a*a/(m-3))*pow(a/rcut,m-3);
}

double SuttonChen::deltaU1(const double &rhobar, const int &N) const
{
 double f = 2*M_PI*N*rhobar*e*a*a*a/(n-3);
 return f*pow(a/rcut,n-3);
}
\end{verbatim}

y la fuerza asociada al potencial de suttonchen que aplica para un par de atomos $i$ y $j$ est\'a dada por,

$$\vec{F}(\vec{r}_{ij}) = -\epsilon\left[n\left(\frac{a}{\vec{r}_{ij}}\right)^n - \frac{Cm}{2}(\rho_j^{(-1/2)}+\rho_i^{(-1/2)})\left(\frac{a}{\vec{r}_{ij}}\right)^m\right]\left(\frac{1}{\vec{r}_{ij}^2}\right)\vec{r}_{ij}$$

Donde la implementaci\'on de la fuerza como m\'etodo virtual de los potenciales met\'alicos queda de la forma,

\begin{verbatim}
 Vector SuttonChen::PairForce(const Vector &rij) const
{
 Vector norm = rij;
 double rmod = rij.Mod();
 norm.Norm();
 return -n*e*pow(a/rmod,n)*(norm/rmod);
}

Vector SuttonChen::ManyBodies(const Vector &rij, const double &rhoi, \\
const double &rhoj) const
{
 double tmp;
 double rmod = rij.Mod();
 tmp=(m/2)*c*e*((1/sqrt(rhoi))+(1/sqrt(rhoj)))*pow(a/rmod,m)*(1.0/rmod);
 Vector ff = rij;
 ff.Norm();
 return tmp*ff;
}
\end{verbatim}


\cajatx{A continuaci\'on, los otros \textbf{potenciales interat\'omicos Met\'alicos} son explicados brevemente, pero el trasfondo es similar al ya planteado hasta ahora.}

%%%%%%%%%%%%%%%%%%%%%%%%%%%%%%%%%%%%%%%%%%%%%%%%%%%%%%%%%%%%%%%%%
%%%%%%%%%%%%%%%%%%%%%%%%%%%%%%%%%%%%%%%%%%%%%%%%%%%%%%%%%%%%%%%%%
\section{Potenciales Interat\'omicos Nulos}
\subsection{nullpairpotential}
Potencial de pares nulo entre especies. Se utiliza en caso de que se desea imponer una interaccion nula entre un par de especies especies at\'omicas. 
\subsection{nullpotential}
Potencial nulo entre especies. Utilizado para anular la interacion entre especies, retorna inmediatamente \verb|NULL| sin necesidad de verificar paridad.

%%%%%%%%%%%%%%%%%%%%%%%%%%%%%%%%%%%%%%%%%%%%%%%%%%%%%%%%%%%%%%%%%
%%%%%%%%%%%%%%%%%%%%%%%%%%%%%%%%%%%%%%%%%%%%%%%%%%%%%%%%%%%%%%%%%
\section{Integradores}
\subsection{beeman}
Integrador de beeman.
\subsection{euler}
Integrador de Euler.
\subsection{nullintegrator}
Integrador nulo.
\subsection{velocityverlet}
Utiliza metodo velocity verlet para integrar.
\subsection{verlet}
Utiliza metodo verlet para integrar.

%%%%%%%%%%%%%%%%%%%%%%%%%%%%%%%%%%%%%%%%%%%%%%%%%%%%%%%%%%%%%%%%%
%%%%%%%%%%%%%%%%%%%%%%%%%%%%%%%%%%%%%%%%%%%%%%%%%%%%%%%%%%%%%%%%%
\section{CellManager}
Estos m\'odulos son los encargados de generar las \textbf{lista inteligentes} para poder realizar simulaciones o calculos de Din\'amica Molecular, los dos plugins implementados a la fecha son:

\subsection{minimumimage}
Uiliza el m\'etodo de m\'inima imagen para realizar los procesos.

\subsection{linkedcell}
Utiliza el m\'etodo de listas linkeadas, \'este m\'etodo es mucho m\'as rapido que el m\'etodo de m\'inima imagen, y de por s\'i es el m\'as utilizado en la mayor\'ia de los c\'odigos de \textbf{DM}

%%%%%%%%%%%%%%%%%%%%%%%%%%%%%%%%%%%%%%%%%%%%%%%%%%%%%%%%%%%%%%%%%
%%%%%%%%%%%%%%%%%%%%%%%%%%%%%%%%%%%%%%%%%%%%%%%%%%%%%%%%%%%%%%%%%
\section{Visualizadores}
\subsection{povray}
Genera diretorios con ficheros pov para visualizar el sistema. Este m\'odulo genera un set de archivos \verb|pov| los cuales son ubicados dentro de un directorio, para un posterior \textit{rendering} para el dise\'no de peliculas o fotograf\'ias de la simulaci\'on.

Los argumentos requeridos (no todos) por el m\'odulo povray, son los siguientes,

\cajatx{
\begin{tabular}{lcl}
 header & = & Es un nombre previo al nombre \\
&&de los ficheros \textbf{pov} que ser\'an generados.\\
 direct & = & Nombre del directorio que se crear\'a.\\
 text & = & Orden para poner texto en \\
&&diferentes posiciones.\\
 background & = & Color del fondo de la imagen. \\
 rotate & = & Orientaci\'on de la c\'amara.\\
 logo   & = & Si desea anadir una imagen. \\
 box  & = & Muestra o no la \textbf{celda} (True/False).\\
 camera & = & Posici\'on de la c\'amara \\
 &&(recomendamos default).\\
\end{tabular}
}

Dentro de cada uno de \'estos argumentos, el que m\'as cabe detallar es \textbf{text}, el formato de ingreso de textos para la vsualizaci\'on, es el siguiente,

\control{text "Titulo" <pos> <color> [size] (extra)}

Ac\'a las opciones son en el orden requerido. El t\'itulo puede ser cualquier texto, si se utiliza el s\'imbolo ``\% '' entonces el valor de \verb|extra| ser\'a reemplazado (Actualmente : Temp, Step). Las opciones \verb|<pos>| y \verb|<color>| son vectores que deben ingresarse en el formato \verb|<x,y,z>| y corresponden a la posici\'on del texto y los colores, exiten valores por defecto, tales como \verb|<green>|, \verb|<red>| o posiciones como \verb|<dl>| (abajo a la izquierda) que pueden ser utilizadas. Finalmente [size] es el tama\~no escalado del texto.

A continuaci\'on un ejemplo t\'ipico de uso de \verb|povray| en una simulaci\'on.

\begin{verbatim}
use povray
    header shoot-
    direct movie
    text "Modelacion de Ar" <dl> <green> [1] ()
    text "Step = %" <3,3,3> <red> [1] (Step)
    text "Temperatura : % [K]" <uc> <blue> [1] (Temp)
    text "http://www.gnm.cl/" <dr> <green> [0.5] ()
    background <0.2,0.1,0.4>
    rotate <0,0,0>
    logo "logo-v2.gif" 1.5 <cr>
enduse
\end{verbatim}

Luego de que los archivos son creados en el directorio ``movie'', es importante que coloque en ese directorio el logo al que los archivos hacen referencia \verb|logo-v2.gif|.

%%%%%%%%%%%%%%%%%%%%%%%%%%%%%%%%%%%%%%%%%%%%%%%%%%%%%%%%%%%%%%%%%
%%%%%%%%%%%%%%%%%%%%%%%%%%%%%%%%%%%%%%%%%%%%%%%%%%%%%%%%%%%%%%%%%
\section{Propiedades Est\'aticas}
\subsection{angdist}
Calcula la distribucion angular de la celda
\subsection{cordnum}
Calcula el n\'umero de cordinaci\'on de la celda.
\subsection{cordnumfunc}
Calcula el n\'umero de cordinaci\'on de la celda.
\subsection{gdr}
Caulcula la funcion de distribucion de pares de la celda.

%%%%%%%%%%%%%%%%%%%%%%%%%%%%%%%%%%%%%%%%%%%%%%%%%%%%%%%%%%%%%%%%%
%%%%%%%%%%%%%%%%%%%%%%%%%%%%%%%%%%%%%%%%%%%%%%%%%%%%%%%%%%%%%%%%%
\section{Propiedades Din\'amicas}
\subsection{vacf}
Calcula la funci\'on de autocorrelaci\'on de velocidades de la celda.
\subsection{msd}
Calcula el desplazamiento cuadratico medio del sistema.

%%%%%%%%%%%%%%%%%%%%%%%%%%%%%%%%%%%%%%%%%%%%%%%%%%%%%%%%%%%%%%%%%%%%%%%%%%%%%%%%%%%%%%%%%%%%%%%%%%%%%%%%%
%%%%%%%%%%%%%%%%%%%%%%%%%%%%%%%%%%%%%%%%%%%%%%%%%%%%%%%%%%%%%%%%%%%%%%%%%%%%%%%%%%%%%%%%%%%%%%%%%%%%%%%%%
%CAPITULO 5%%%%%%%%%%%%%%%%%%%%%%%%%%%%%%%%%%%%%%%%%%%%%%%%%%%%%%%%%%%%%%%%%%%%%%%%%%%%%%%%%%%%%%%%%%%%%%
%%%%%%%%%%%%%%%%%%%%%%%%%%%%%%%%%%%%%%%%%%%%%%%%%%%%%%%%%%%%%%%%%%%%%%%%%%%%%%%%%%%%%%%%%%%%%%%%%%%%%%%%%
%%%%%%%%%%%%%%%%%%%%%%%%%%%%%%%%%%%%%%%%%%%%%%%%%%%%%%%%%%%%%%%%%%%%%%%%%%%%%%%%%%%%%%%%%%%%%%%%%%%%%%%%%
\chapter{Utilidades Derivadas de lpmd}
\label{chap:utilidades}
\section{lpmd-analyzer}
\section{lpmd-convert}
\section{lpmd-mixer}

%%%%%%%%%%%%%%%%%%%%%%%%%%%%%%%%%%%%%%%%%%%%%%%%%%%%%%%%%%%%%%%%%%%%%%%%%%%%%%%%%%%%%%%%%%%%%%%%%%%%%%%%%
%%%%%%%%%%%%%%%%%%%%%%%%%%%%%%%%%%%%%%%%%%%%%%%%%%%%%%%%%%%%%%%%%%%%%%%%%%%%%%%%%%%%%%%%%%%%%%%%%%%%%%%%%
%CAPITULO 6%%%%%%%%%%%%%%%%%%%%%%%%%%%%%%%%%%%%%%%%%%%%%%%%%%%%%%%%%%%%%%%%%%%%%%%%%%%%%%%%%%%%%%%%%%%%%%
%%%%%%%%%%%%%%%%%%%%%%%%%%%%%%%%%%%%%%%%%%%%%%%%%%%%%%%%%%%%%%%%%%%%%%%%%%%%%%%%%%%%%%%%%%%%%%%%%%%%%%%%%
%%%%%%%%%%%%%%%%%%%%%%%%%%%%%%%%%%%%%%%%%%%%%%%%%%%%%%%%%%%%%%%%%%%%%%%%%%%%%%%%%%%%%%%%%%%%%%%%%%%%%%%%%
\chapter{Ejemplos.}
\label{chap:exa}

Ac\'a encontrar\'a algunos ejemplos de simulaciones realizadas con lpmd, en su ultima versi\'on estable.

\section{Ejemplos B\'asicos}

\subsection{Sistemas Cristalinos 3D}

\subsubsection{Arg\'on}

A continuaci\'on una simulaci\'on de Ar con 108 \'atomos, en la cu\'al se realizan distintos escalamientos de temperatura. El ejemplo puede descargarlo completamente de,

\cajatx{http://wwww.gnm.cl/software/lpmd/examples/ar108-1.tgz}

Veamos algunas partes fundamentales del fichero de control.


\begin{verbatim}

################################################################
# System file of Ar gas 
# using LPMD
################################################################

####################
#CELL PROPERTIES####
####################
cell crystal a=17.1191 b=17.1191 c=17.1191 alpha=90.0 beta=90.0 gamma=90.0
input module=lpmd file=argon108.lpmd level=0
output module=xyz file=output.xyz each=20 level=0
charge Ar 0.0

###################
#MD PROPERTIES#####
###################
repeat 1 1 1 
periodic true true true
steps 50000
dumping 10000 ljargon.dump
monitor 10
temperature 84

###################
##MODULES DEF######
###################

use lennardjones as lj_Ar
    sigma 3.41
    epsilon 0.0103408
    cutoff 8.5
enduse

use beeman
    dt 10.0
enduse

use tempscaling as ts0
    from 84
    to 84
enduse

use tempscaling as ts1
    from 84
    to 300
enduse

use tempscaling as ts2
    from 300
    to 300
enduse

use tempscaling as ts3
    from 300
    to 1
enduse

use tempscaling as ts4
    from 1
    to 1
enduse

use povray2
    header shoot-
    direct movie
    text "Modelacion de Ar" <dl> <green> [1] ()
    text "Step = %" <3,3,3> <red> [1] (Step)
    text "Temperatura : % [K]" <uc> <blue> [1] (Temp)
    text "http://www.gnm.cl/" <dr> <green> [0.5] ()
    background <0.2,0.1,0.4>
    rotate <0,0,0>
    logo "logo-v2.gif" 1.5 <cr>
enduse

###################
#MOD APPLICATION###
###################
potential lj_Ar Ar Ar
integrator beeman
apply ts0 start=1 end=10000 each=10
apply ts1 start=10000 end=20000 each=10
apply ts2 start=20000 end=30000 each=10
apply ts3 start=30000 end=40000 each=10
apply ts4 start=40000 end=50000 each=10
visualize povray2 start=1 end=50000 each=100
\end{verbatim}


\subsection{Ejemplos m\'as Complejos}


%%%%%%%%%%%%%%%%%%%%%%%%%%%%%%%%%%%%%%%%%%%%%%%%%%%%%%%%%%%%%%%%%%%%%%%%%%%%%%%%%%%%%%%%%%%%%%%%%%%%%%%%%
%%%%%%%%%%%%%%%%%%%%%%%%%%%%%%%%%%%%%%%%%%%%%%%%%%%%%%%%%%%%%%%%%%%%%%%%%%%%%%%%%%%%%%%%%%%%%%%%%%%%%%%%%
%CAPITULO 7%%%%%%%%%%%%%%%%%%%%%%%%%%%%%%%%%%%%%%%%%%%%%%%%%%%%%%%%%%%%%%%%%%%%%%%%%%%%%%%%%%%%%%%%%%%%%%
%%%%%%%%%%%%%%%%%%%%%%%%%%%%%%%%%%%%%%%%%%%%%%%%%%%%%%%%%%%%%%%%%%%%%%%%%%%%%%%%%%%%%%%%%%%%%%%%%%%%%%%%%
%%%%%%%%%%%%%%%%%%%%%%%%%%%%%%%%%%%%%%%%%%%%%%%%%%%%%%%%%%%%%%%%%%%%%%%%%%%%%%%%%%%%%%%%%%%%%%%%%%%%%%%%%
\chapter{Desarrollando M\'odulos}
\label{chap:own}

\section{Idea Principal}

Una de las caracter\'isticas princpales de \lpmd con respecto a otros c\'odigos de Din\'amica Molecular es su gran \textit{modularidad} lo que hace que muchas propiedades de un ciclo regular de din\'amica molecular sean modificables facilmentes, por ejemplo un cilco de din\'amica molecular consta de muchas \textbf{piezas} constantes, tales como los potenciales, integradores o bien una propiedad que puede ser calculada de forma instantanea o que requiere una correlaci\'on temporal del sistema.

Consideremos por ejemplo :

Se puede observar claramente que existen \textbf{bloques} en donde la caracter\'istica principal de cada uno de ellos en \lpmd es que son modificables por diferentes tipos de \textbf{m\'odulos} que \textit{encajan} perfectamente en estos bloques, estos m\'odulos pueden ser din\'amicos lo que da una ventaja significativa a la hora de desarrollar el c\'odigo necesario para trabajar con \'el.

En este cap\'itulo se exponen las distintas piezas \textbf{modificables} de \lpmd que har\'an de este un c\'odigo mucho m\'as \'util para el desarrollo de distintas investigaciones con una misma herramienta.

\section{Desarrollando un Potencial}

Una de las piezas fundamentales en la din\'amica molecular, es la integraci\'on de un potencial interat\'omico entre las particulas que componen el sistema, es por eso que \lpmd facilita 

\section{Desarrollando una Propiedad}

Durante una simulaci\'on de din\'amica molecular una de las herramientas m\'as utilizadas  son las propiedades f\'isicas del sistema, las que son, een ocaciones, comparables con resultados experimentales provenientes del laboratorio. Sin embargo estas propiedades, no siempre pueden ser evaluadas ya que los programas no cuentan con ellas, o bien deben implementarse para resolver este problema, aprendiendo a tomar configuraciones de salida de otros programas, para nuestros fines.

Para resolver esta situaci\'on \lpmd calcula propiedades de un sistema atomico, de forma modular, es decir cada uno de nosotros puede \textbf{programar} la propidad que necesesita para su evaluacion, instantanea, o en ocaciones temporal.

\lpmd separa las propiedades de una celda de simulaci\'on en 2 tipos :

\begin{itemize}
 \item Propiedades Instantaneas.
 \item Propiedades Temporales.
\end{itemize}

En donde, las instantaneas corresponden a las propiedades que pueden calcularse en un instante de tiempo y no dependen de configuraciones previas del sistema (como funci\'on de distribuci\'on de pares), en cambio las temporales son aquellas que dependen de configuraciones previas del sistema, por ejemplo la funci\'on de autocorrelaci\'on de velocidades.

A continuaci\'on se mostrar\'a la estructura b\'asica necesaria para implementar propiedades instantaneas y temporales en el programa \lpmd y as\'i utilizarlas durante la ejecuci\'on de \lpmd o bien para trabajar con nuevas utilidades.

%%%%%%%%%%%%%%%%%%%%%%%%%%%%%%%%%%%%%%%%%%%%%%%%%%%%%%%%%%%%%%%%%
%%%%%%%%%%%%%%%%%%%%%%%%%%%%%%%%%%%%%%%%%%%%%%%%%%%%%%%%%%%%%%%%%
\subsection{Instant\'anea}

Las propiedades m\'as simples para comenzar a implementar son las instantaneas, dentro de este tipo de propiedades tenemos aquellas que retornan por valor un solo n\'umero real (energ\'ia, temperatura, etc.) y otras que retornan una matriz de numeros reales (como g(r) o distribuci\'on angular, etc.), para esto es necesario ubicarse dentro del directorio \verb|lib| de \lpmd y generar dos nuevos archivos que constan con informacion b\'asica de la propiedad.

Consideremos por ejemplo la funci\'on de distribuci\'on de pares (\verb|g(r)|) (\textbf{nota : esto ya existe en el directorio, ac\'a se muestra a modo de ejemplo.}), para ello generamos dos nuevos ficheros dentro de \verb|lib| :

\begin{center}
 \verb|touch gdr.cc gdr.h|
\end{center}

%%%%%%%%%%%%%%%%%%%%%%%%%%%%%%%%%%%%%%%%%%%%%%%%%%%%%%%%%%%%%%%%%
%%%%%%%%%%%%%%%%%%%%%%%%%%%%%%%%%%%%%%%%%%%%%%%%%%%%%%%%%%%%%%%%%
\subsection{Temporal}

Las propiedades temporales n est\'an dise\~nadas para ser evaluadas duratne a siulaci\'on, sin embargo es facil su implementacion en la API, lo que puede llevar a utilziarlas en otros c\'odigos, tales como fumody.

La idea es utilizar los archivos de configuraci\'on de salida de lpmd.

\section{Desarrollando Integrador}

Un integrador cumple la funcion de ...

\section{Desarrollando Utilidades}

La API (liblpmd) es la principal herramienta que deja lpmd, que puede ser utilizada no solo por \'el sino que por utilidades que nosotros deseamos dise\~nar.


%%%%%%%%%%%%%%%%%%%%%%%%%%%%%%%%%%%%%%%%%%%%%%%%%%%%%%%%%%%%%%%%%%%%%%%%%%%%%%%%%%%%%%%%%%%%%%%%%%%%%%%%%
%%%%%%%%%%%%%%%%%%%%%%%%%%%%%%%%%%%%%%%%%%%%%%%%%%%%%%%%%%%%%%%%%%%%%%%%%%%%%%%%%%%%%%%%%%%%%%%%%%%%%%%%%
%CAPITULO 8%%%%%%%%%%%%%%%%%%%%%%%%%%%%%%%%%%%%%%%%%%%%%%%%%%%%%%%%%%%%%%%%%%%%%%%%%%%%%%%%%%%%%%%%%%%%%%
%%%%%%%%%%%%%%%%%%%%%%%%%%%%%%%%%%%%%%%%%%%%%%%%%%%%%%%%%%%%%%%%%%%%%%%%%%%%%%%%%%%%%%%%%%%%%%%%%%%%%%%%%
%%%%%%%%%%%%%%%%%%%%%%%%%%%%%%%%%%%%%%%%%%%%%%%%%%%%%%%%%%%%%%%%%%%%%%%%%%%%%%%%%%%%%%%%%%%%%%%%%%%%%%%%%
\chapter{Paralelizaci\'on}

Por qu\'e paralelizar, desde donde comenzar. Actualmente, se espera una versi\'on de \lpmd paralela para la versi\'on 0.6 o 0.7 del c\'odigo, sin embargo el nucleo principal de paralelizaci\'on no s encuentra en lpmd, sino que en la API \textbf{liblpmd}, por lo que la evoluci\'on de \'esta es el primer paso en la paralelizaci\'on final del c\'odigo.


%%%%%%%%%%%%%%%%%%%%%%%%%%%%%%%%%%%%%%%%%%%%%%%%%%%%%%%%%%%%%%%%%%%%%%%%%%%%%%%%%%%%%%%%%%%%%%%%%%%%%%%%%
%%%%%%%%%%%%%%%%%%%%%%%%%%%%%%%%%%%%%%%%%%%%%%%%%%%%%%%%%%%%%%%%%%%%%%%%%%%%%%%%%%%%%%%%%%%%%%%%%%%%%%%%%
%CAPITULO 9%%%%%%%%%%%%%%%%%%%%%%%%%%%%%%%%%%%%%%%%%%%%%%%%%%%%%%%%%%%%%%%%%%%%%%%%%%%%%%%%%%%%%%%%%%%%%%
%%%%%%%%%%%%%%%%%%%%%%%%%%%%%%%%%%%%%%%%%%%%%%%%%%%%%%%%%%%%%%%%%%%%%%%%%%%%%%%%%%%%%%%%%%%%%%%%%%%%%%%%%
%%%%%%%%%%%%%%%%%%%%%%%%%%%%%%%%%%%%%%%%%%%%%%%%%%%%%%%%%%%%%%%%%%%%%%%%%%%%%%%%%%%%%%%%%%%%%%%%%%%%%%%%%
\chapter{Gente}
\label{chap:auth}

Desarrollador principal \textbf{Sergio Davis} en colaboraci\'on con \textbf{Joaqu\'in Peralta} y \textbf{Claudia Loyola}, sin embargo el objetivo de lpmd es lograr atraer la atenci\'on de la gente y motivar a que sean colaboradores activos en el desarrollo del proyecto, ya sea desarrollando plugins o bien realizando sugerencias, que con mucho gusto ser\'an acogidas.

Programadores Principales:

\begin{itemize}
 \item Sergio Davis, KTH
 \item Claudia Loyola, UCHILE
 \item Joaqu\'in Peralta, UCHILE
\end{itemize}

Programadores Adicionales:

\begin{itemize}
 \item Nicolas Viaux, UCHILE
 \item Nicolas Perez, UCHILE
 \item NN, UCHILE
 \item 
\end{itemize}

Colaboradores :

\begin{itemize}
 \item Gonz\'alo Guti\'errez
 \item Eduardo Men\'endez
\end{itemize}


\begin{itemize}
 \item ... : ...
\end{itemize}

%%%%%%%%%%%%%%%%%%%%%%%%%%%%%%%%%%%%%%%%%%%%%%%%%%%%%%%%%%%%%%%%%%%%%%%%%%%%%%%%%%%%%%%%%%%%%%%%%%%%%%%%%
%%%%%%%%%%%%%%%%%%%%%%%%%%%%%%%%%%%%%%%%%%%%%%%%%%%%%%%%%%%%%%%%%%%%%%%%%%%%%%%%%%%%%%%%%%%%%%%%%%%%%%%%%
%CAPITULO 10%%%%%%%%%%%%%%%%%%%%%%%%%%%%%%%%%%%%%%%%%%%%%%%%%%%%%%%%%%%%%%%%%%%%%%%%%%%%%%%%%%%%%%%%%%%%%%
%%%%%%%%%%%%%%%%%%%%%%%%%%%%%%%%%%%%%%%%%%%%%%%%%%%%%%%%%%%%%%%%%%%%%%%%%%%%%%%%%%%%%%%%%%%%%%%%%%%%%%%%%
%%%%%%%%%%%%%%%%%%%%%%%%%%%%%%%%%%%%%%%%%%%%%%%%%%%%%%%%%%%%%%%%%%%%%%%%%%%%%%%%%%%%%%%%%%%%%%%%%%%%%%%%%
\chapter{Ap\'endice}

\section{API - liblpmd}
\label{ap:API}
La \textbf{API} (Ap. Programming Interface) es una herramienta de programaci\'on que puede ser utilizada por cualquier usuario/programador que se vea beneficiado por sus caracter\'isticas.

Consideramos que la mejor forma de comprender el funcionamiento de esta \textbf{API}, es directamente con c\'odigos de ejemplo que pueden escribir los desarrolladores. A continuaci\'on se muestran 3 ejemplos de utilizaci\'on de la \textbf{API}, el primero se enmarca en un ``nano-programa'' de \textbf{DM}, el segundo es la evaluaci\'on de una propiedad est\'atica de una celda del tipo \texttt{.xyz} y la \'ultima una propiedad din\'amica de una celda.

%%%%%%%%%%%%%%%%%%%%%%%%%%%%%%%%%%%%%%%%%%%%%%%%%%%%%%%%%%%%%%%%%
%%%%%%%%%%%%%%%%%%%%%%%%%%%%%%%%%%%%%%%%%%%%%%%%%%%%%%%%%%%%%%%%%
\subsection{Din\'amica Molecular B\'asica}

A continuaci\'on un c\'odigo que utilza todas las caracter\'isticas de la \textbf{API}, para realizar din\'amica molecular.

\begin{verbatim}
 /*
 * Ejemplo simple de dinamica molecular usando el API de liblpmd
 */

#include <lpmd/api.h>
#include <iostream>

using namespace lpmd;

int main()
{
 MD md;            // define md como un objeto de dinamica molecular
 PluginManager pm; // define pm como un manejador de plugins

 SimulationCell cell(1, 1, 1, true, true, true); // cell es la celda de simulacion
 cell.SetVector(0, Vector(17.1191, 0.0, 0.0));   // define los vectores de la celda
 cell.SetVector(1, Vector(0.0, 17.1191, 0.0));
 cell.SetVector(2, Vector(0.0, 0.0, 17.1191));
 md.SetCell(cell);                    // asigna la celda de simulacion al objeto MD 

 // Carga de plugins con sus parametros
 pm.LoadPlugin("minimumimage", "");
 pm.LoadPlugin("crystalfcc", "symbol Ar nx 3 ny 3 nz 3");
 pm.LoadPlugin("lennardjones", "sigma 3.41 epsilon 0.0138");
 pm.LoadPlugin("velocityverlet", "dt 1.0");
 pm.LoadPlugin("temperature", "t 600.0");
 pm.LoadPlugin("energy", "");

 CellManager & cm = CastModule<CellManager>(pm["minimumimage"]);
 cell.SetCellManager(cm);            // asigna el manejador de celda

 CellGenerator & cg = CastModule<CellGenerator>(pm["crystalfcc"]);
 cg.Generate(cell);

 Potential & pot = CastModule<Potential>(pm["lennardjones"]);
 PotentialArray & potarray = md.GetPotentialArray();
 potarray.Set("Ar", "Ar", pot); // asigna lennardjones al arreglo de potenciales de MD

 Integrator & integ = CastModule<Integrator>(pm["velocityverlet"]);
 md.SetIntegrator(integ);

 InstantProperty & energ = CastModule<InstantProperty>(pm["energy"]);
 
 SystemModifier & therm = CastModule<SystemModifier>(pm["temperature"]);
 therm.Apply(cell);  // aplica el termalizador "temperature" a la celda de simulacion

 // Loop principal de la simulacion, hace 500 pasos
 md.Initialize(); 
 std::cout << "# Pasos   Temperatura" << '\n';
 for (long i=0;i<500;++i)
 {
  md.DoStep();                       // avanza el sistema un paso de simulacion
  energ.Evaluate(cell, pot);         // evalua las propiedades en el plugin energy
  double T;
  T = pm["energy"].GetProperty("temperature"); // pide valor de temp al plugin energy
  std::cout << i << "         " << T << '\n';
 }
 return 0;
}
\end{verbatim}

Para generar el ejecutable,

\control{g++ -o nanodm main.cc -llpmd -ldl -lm}

y listo, tendremos entonces un ejecutable llamado \verb|nanodm| que realizar\'a una simple corrida de din\'amica molecular.

%%%%%%%%%%%%%%%%%%%%%%%%%%%%%%%%%%%%%%%%%%%%%%%%%%%%%%%%%%%%%%%%%
%%%%%%%%%%%%%%%%%%%%%%%%%%%%%%%%%%%%%%%%%%%%%%%%%%%%%%%%%%%%%%%%%
\subsection{Calculo de Propiedad est\'atica}

Consideremos que tenemos una celda de simulaci\'on y queremos utiliar las ventajas de la \textbf{API} para calcular una propiedad, que sabemos existe en un m\'odulo, por ejemplo \textbf{gdr}. El c\'odigo para el c\'alculo de \textbf{gdr} de la celda nos quea as\'i,

\begin{verbatim}
 /*
 *
 *
 *
 */

#include <lpmd/api.h>

using namespace lpmd;

int main(int argc, char *argv[])
{
 if (argc < 2) 
 {
  std::cerr << "testgdr <file.xyz>" << '\n';
  exit(1);
 }
 PluginManager pm;
 pm.LoadPlugin("xyz", "file="+std::string(argv[1]));
 pm.LoadPlugin("gdr", "rcut 8.0 bins 300 average true");
 pm.LoadPlugin("nullpotential", "");
 pm.LoadPlugin("linkedcell", "nx 7 ny 7 nz 7 cutoff 8.0");

 CellReader & cread = dynamic_cast<CellReader &>(pm["xyz"]);
 InstantProperty & gdr = dynamic_cast<InstantProperty &>(pm["gdr"]); 
 ScalarTable & gdrvalue = dynamic_cast<ScalarTable &>(pm["gdr"]);
 CellManager & cm = dynamic_cast<CellManager &>(pm["linkedcell"]);
 Potential & dummy = dynamic_cast<Potential &>(pm["nullpotential"]);

 pm["gdr"].Show();

 std::vector<SimulationCell> configs;
 cread.ReadMany(std::string(argv[1]), configs);

 Cell cell(13.16, 13.16, 21.39, M_PI/2, M_PI/2, M_PI*120.0/180.0);
 Vector v1 = cell.GetVector(0);
 v1 = Vector(v1.Get(1), v1.Get(0), v1.Get(2));
 Vector v2 = cell.GetVector(1);
 v2 = Vector(v2.Get(1), v2.Get(0), v2.Get(2));
 cell.SetVector(0, v2);
 cell.SetVector(1, v1);
 for (int i=0;i<3;++i) std::cerr << cell.GetVector(i) << std::endl;

 std::cerr << "Read " << configs.size() << " configurations." << '\n';
 std::cerr << "Configuration 0 has " << configs[0].Size() << " atoms\n";

 for (unsigned long i=0;i<configs.size();++i)
 {
  configs[i].SetCell(cell);
  configs[i].SetCellManager(cm);
  gdr.Evaluate(configs[i], dummy);
  gdrvalue.AddToAverage();
 }

 std::cout << gdrvalue << '\n';

 return 0;
}
\end{verbatim}

Esto, lo compilamos de manera similar al caso anterior, obteniendo un ejecutable para calcular una propiedad est\'atica, en este caso \verb|gdr| para la celda de simulaci\'on.

\end{document}
