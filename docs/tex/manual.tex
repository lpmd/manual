\documentclass[a4paper,10pt]{scrbook}

\usepackage{graphicx,fancybox,wrapfig}
\usepackage{fullpage}

%%%%%%%%%%%%%%%%%%%%%%%%%%%%%%%%%%%%%%%%%%%%%%%%%%%%%%%%%%%%%%%%%%%%%%%%%%%%%%%%%%%%%%%%%%%%%%%%%%%%%%%%%%%%%%%
%%NEWCOMMANDS DEFINITIONS%%%%%%%%%%%%%%%%%%%%%%%%%%%%%%%%%%%%%%%%%%%%%%%%%%%%%%%%%%%%%%%%%%%%%%%%%%%%%%%%%%%%%%
%%%%%%%%%%%%%%%%%%%%%%%%%%%%%%%%%%%%%%%%%%%%%%%%%%%%%%%%%%%%%%%%%%%%%%%%%%%%%%%%%%%%%%%%%%%%%%%%%%%%%%%%%%%%%%%
\newcommand{\lpmd}{\textbf{lpmd }}
\newcommand{\fumfun}[4]{\begin{center}\begin{tabular}{|c|c|c|c|}\hline
 Flag & Work & Test & Mand\\\hline
#1 & #2 & #3 & #4 \\\hline
\end{tabular}
\end{center}}
\newcommand{\D}[2]{\frac{\partial #2}{\partial #1}}
\newcommand{\DD}[2]{\frac{\partial^2 #2}{\partial #1^2}}
\newcommand{\DC}[2]{\frac{D #2}{D #1}} %derivada conectiva.
%\newcommand{\caja}[1]{\begin{center}\fbox{#1}\end{center}}
\newcommand{\cajatx}[1]{\begin{center}\setlength{\fboxsep}{0.2cm}\fbox{\parbox[l]{10cm}{#1}}\end{center}}
%\newcommand{\cvb}[1]{\begin{center}\begin{verbatim} #1 \end{verbatim}\end{center}}
\newcommand{\cajaeq}[1]{\begin{center}\setlength{\fboxsep}{0.1cm}\fbox{\parbox[b]{10cm}{#1}}\end{center}}
\newcommand{\cajafi}[3]{\begin{figure}\centering\shadowbox{\begin{minipage}{3.5 in}\centering\includegraphics[totalheight=2in]{#1}\caption{#2}\label{fig: #3}\end{minipage}}\end{figure}}
\newcommand{\control}[1]{\begin{center}\begin{minipage}{10cm}\texttt{#1}\end{minipage}\end{center}}


\newcommand {\foto}[4]{\begin{figure}
%   \begin{framed}
      \begin{center}
      \includegraphics[scale=#2]{#1}
      \end{center}

      \caption{\emph{#3}}

      \label{#4}
%   \end{framed}
   \end{figure}
   }

\begin{document}
\author{http://www.gnm.cl}
\title{``Las Palmeras'' Molecular Dynamics - \textbf{LPMD}.}
\maketitle

\tableofcontents
%%%%%%%%%%%%%%%%%%%%%%%%%%%%%%%%%%%%%%%%%%%%%%%%%%%%%%%%%%%%%%%%%%%%%%%%%%%%%%%%%%%%%%%%%%%%%%%%%%%
%%%%%%%%%%%%%%%%%%%%%%%%%%%%%%%%%%%%%%%%%%%%%%%%%%%%%%%%%%%%%%%%%%%%%%%%%%%%%%%%%%%%%%%%%%%%%%%%%%%
\chapter{El Programa}
\label{chap:lpmd}

\section{Origenes.}

El c\'odigo \lpmd comenz\'o a desarrollarse durante junio del a\~no 2007 como una idea de desarrollar e implementar din\'amica molecular de forma modular, tratando de ser lo m\'as amigable para el dise\~no de plugins y de actualizaciones. Los programadores principales de la primera etapa del c\'odigo fueron, Sergio Davis, Claudia Loyola y Joaqu\'in Peralta, todos ellos integrantes del \textit{Grupo de NanoMateriales} \textbf{http://www.gnm.cl}, actualmente existen m\'as colaboradores en le desarollo del c\'odigo (v\'ease \textbf{gente}).

Pruebas de Texto y comandos.

\begin{center}
\texttt{cellmanager minimumimage}
\end{center}

\control{cellmanager minimumimage \\ other line \\ another line more}


\begin{minipage}[c]{10cm}
 \texttt{cellmanager minimumimage}
\end{minipage}

\begin{center}
 \begin{minipage}[c]{5cm}
 \texttt{cellmanager minimumimage \\ other line \\ another line more}
\end{minipage}
\end{center}


\section{Caracter\'isticas Globales.}

Dentro de las caracter\'isticas principales de \lpmd tenemos:
\begin{itemize}
 \item Modularidad
 \item Plugins simples
 \item Easy input file.
\end{itemize}

%%%%%%%%%%%%%%%%%%%%%%%%%%%%%%%%%%%%%%%%%%%%%%%%%%%%%%%%%%%%%%%%%%%%%%%%%%%%%%%%%%%%%%%%%%%%%%%%%%%%%%%%%
%%%%%%%%%%%%%%%%%%%%%%%%%%%%%%%%%%%%%%%%%%%%%%%%%%%%%%%%%%%%%%%%%%%%%%%%%%%%%%%%%%%%%%%%%%%%%%%%%%%%%%%%%
%CAPITULO 1%%%%%%%%%%%%%%%%%%%%%%%%%%%%%%%%%%%%%%%%%%%%%%%%%%%%%%%%%%%%%%%%%%%%%%%%%%%%%%%%%%%%%%%%%%%%%%
%%%%%%%%%%%%%%%%%%%%%%%%%%%%%%%%%%%%%%%%%%%%%%%%%%%%%%%%%%%%%%%%%%%%%%%%%%%%%%%%%%%%%%%%%%%%%%%%%%%%%%%%%
%%%%%%%%%%%%%%%%%%%%%%%%%%%%%%%%%%%%%%%%%%%%%%%%%%%%%%%%%%%%%%%%%%%%%%%%%%%%%%%%%%%%%%%%%%%%%%%%%%%%%%%%%
\chapter{Instalaci\'on}
\label{chap:inst}

\lpmd ha sido probado en distintas arquitecturas, y compiladores, hasta ahora podemos \textbf{dar fe} que se ha logrado compilar el c\'odigo fuente en las siguientes arquitecturas.

\begin{itemize}
 \item Linux I686/AMD64
 \item OS/X
 \item Solaris
 \item Inclusive en Windows
\end{itemize}

\section{Descarga}

Para el correcto funcionamiento de \lpmd es necesario instalar previamente una libreria y un set de plugins. Por lo que se requieren descargar 3 paquetes principales. La divisi\'on del c\'odigo en este set de paquetes se debe a la reutilizaci\'on de la \textbf{API} y los \textbf{plugins} para nuevos programas de utilidades o bien para un propio usuario interesado en la programaci\'on.

%%%%%%%%%%%%%%%%%%%%%%%%%%%%%%%%%%%%%%%%%%%%%%%%%%%%%%%%%%%%%%%%%
%%%%%%%%%%%%%%%%%%%%%%%%%%%%%%%%%%%%%%%%%%%%%%%%%%%%%%%%%%%%%%%%%
\subsection{Descarga de versi\'on estable}

La \'ultima versi\'on estable de los paquetes es :

\begin{itemize}
 \item liblpmd : ver. 0.1.0
 \item plugins : ver. 0.5.1
 \item lpmd    : ver. 0.5
\end{itemize}

\begin{description}
 \item [liblpmd] Version 0.1.0
 \item [plugins] Version 0.5.1
 \item [lpmd] Version 0.5
\end{description}


Los n\'umeros de los plugins indican la compatibilidad con las versiones de \lpmd y de la \textbf{API} liblpmd, de \'esta forma se mantiene un \textbf{orden} segun la version que conocemos.

%%%%%%%%%%%%%%%%%%%%%%%%%%%%%%%%%%%%%%%%%%%%%%%%%%%%%%%%%%%%%%%%%
%%%%%%%%%%%%%%%%%%%%%%%%%%%%%%%%%%%%%%%%%%%%%%%%%%%%%%%%%%%%%%%%%
\subsection{Descarga versi\'on en desarrollo}

Para los interesados en el desarrollo actual de lpmd y sus dependencias principales pueden ser descargadas con:

\begin{center}
 \begin{verbatim}
  svn co svn://www.gnm.cl/lpmd/liblpmd/unstable liblpmd-uns
  svn co svn://www.gnm.cl/lpmd/plugins/unstable plugins-uns
  svn co svn://www.gnm.cl/lpmd/lpmd/unstable lpmd-uns
 \end{verbatim}
\end{center}

\section{Instalaci\'on}
Antes de comenzar con la instalaci\'on de \lpmd es necesario instalar los 2 paquetes previos, \textbf{liblpmd} y \textbf{plugins}, a continuaci\'on se muestra una descripci\'on de como instalar cada uno de los paquetes.

\cajatx{Nota : Los que poseen la versi\'on inestable cuentan con el fichero \textbf{autogen.sh} para poder generar los Makefiles.}

%%%%%%%%%%%%%%%%%%%%%%%%%%%%%%%%%%%%%%%%%%%%%%%%%%%%%%%%%%%%%%%%%
%%%%%%%%%%%%%%%%%%%%%%%%%%%%%%%%%%%%%%%%%%%%%%%%%%%%%%%%%%%%%%%%%
\subsection{Instalando liblpmd}

En primer lugar descomprima el fichero 

\control{tar -xvzf liblpmd-X.X.X.tar.gz}

lo que le generar\'a un nuevo directorio, para instalar esta librer\'ia con todos los m\'etodos necesarios para el funcionamiento de \lpmd y la implementaci\'on de plugins ejecute :

\control{./configure \\ make}

y como administrador, 


\control{make install}


Por \textit{default} el directorio de instalaci\'on de la API programa es \verb|/usr/local/|, en caso de requerir una ubicaci\'on distinta revise las opciones con \verb|./configure --help|.

En caso de cualquier error envie un mail a alguno de los desarrolladores principales o en su defecto a \verb|gnm@gnm.cl|.

%%%%%%%%%%%%%%%%%%%%%%%%%%%%%%%%%%%%%%%%%%%%%%%%%%%%%%%%%%%%%%%%%
%%%%%%%%%%%%%%%%%%%%%%%%%%%%%%%%%%%%%%%%%%%%%%%%%%%%%%%%%%%%%%%%%
\subsection{Instalando plugins}

Una de los requerimientos b\'asicos de \lpmd es tener bien configurada la ubicaci\'on de los plugins que \lpmd requiere, es por eso que se debe utilizar la ubicacion de la instalaci\'on de la libreria lpmd \textbf{liblpmd}. Usualmente deber\'ia correrse con:

\control{./configure --with-lpmd=/usr/local \\ make}

y proceder la instalaci\'on como superusuario:

\control{make install}

Esto ubicar\'a todos los plugins incluidos en el paquete \verb|plugins| en el directorio \verb|/usr/local/lib/lpmd|. De esta forma ya estamos listos para comenzar la instalaci\'on de lpmd y realizar las primeras pruebas.

%%%%%%%%%%%%%%%%%%%%%%%%%%%%%%%%%%%%%%%%%%%%%%%%%%%%%%%%%%%%%%%%%
%%%%%%%%%%%%%%%%%%%%%%%%%%%%%%%%%%%%%%%%%%%%%%%%%%%%%%%%%%%%%%%%%
\subsection{Instalando lpmd}

Es uno de los paquetes m\'as peque\~nos y rapidos de instalar, para proceder, se hace de manera similar que los anteriores, ejecutando :

\control{./configure \\ make}

y proceder a instalar como superusuario:

\control{make install}

Esto generara un ejecutable \verb|lpmd| en \verb|/usr/local/bin/lpmd|, que puede ser ejecutado desde cualquier sitio (si se presenta algun problema, corriga su \verb|PATH|).

puede correr lpmd con

\begin{verbatim}
username@machine:~$ lpmd
Usage: lpmd [--systemfile <file.sys> | -s <file.sys>] [--version | v]
       lpmd [--help | -h ]
username@machine:~$
\end{verbatim}

Para los detalles de el fichero de sistema revise el cap\'itulo~\ref{chap:input}.

%%%%%%%%%%%%%%%%%%%%%%%%%%%%%%%%%%%%%%%%%%%%%%%%%%%%%%%%%%%%%%%%%
%%%%%%%%%%%%%%%%%%%%%%%%%%%%%%%%%%%%%%%%%%%%%%%%%%%%%%%%%%%%%%%%%
\subsection{Instalaci\'on de la API liblpmd en directorio personal}

%%%%%%%%%%%%%%%%%%%%%%%%%%%%%%%%%%%%%%%%%%%%%%%%%%%%%%%%%%%%%%%%%
%%%%%%%%%%%%%%%%%%%%%%%%%%%%%%%%%%%%%%%%%%%%%%%%%%%%%%%%%%%%%%%%%
\subsection{Instalaci\'on de lpmd en directorio personal}

%%%%%%%%%%%%%%%%%%%%%%%%%%%%%%%%%%%%%%%%%%%%%%%%%%%%%%%%%%%%%%%%%
%%%%%%%%%%%%%%%%%%%%%%%%%%%%%%%%%%%%%%%%%%%%%%%%%%%%%%%%%%%%%%%%%
\subsection{Actualizando lpmd}

\lpmd tiene m\'as de una forma de actualizarse, para eso en primer lugar debemos tener claro que versi\'on de la API posee \lpmd ya que las versiones nuevas de \lpmd o de los plugins dependen completamente de la API que tengamos.

Para ver la versi\'on que utiliza \lpmd ejecute : \verb|lpmd -h| donde mostrara una linea que contendr\'a el siguiente texto \textbf{Using liblmpd version X.X.X}.

Actualmente la API se encuentra en la versi\'on \textbf{1.0.0} esperamos que la pr\'oxima versi\'on soporte MPI para muchas fases de la din\'amica molecular.


%%%%%%%%%%%%%%%%%%%%%%%%%%%%%%%%%%%%%%%%%%%%%%%%%%%%%%%%%%%%%%%%%%%%%%%%%%%%%%%%%%%%%%%%%%%%%%%%%%%%%%%%%
%%%%%%%%%%%%%%%%%%%%%%%%%%%%%%%%%%%%%%%%%%%%%%%%%%%%%%%%%%%%%%%%%%%%%%%%%%%%%%%%%%%%%%%%%%%%%%%%%%%%%%%%%
%CAPITULO 2%%%%%%%%%%%%%%%%%%%%%%%%%%%%%%%%%%%%%%%%%%%%%%%%%%%%%%%%%%%%%%%%%%%%%%%%%%%%%%%%%%%%%%%%%%%%%%
%%%%%%%%%%%%%%%%%%%%%%%%%%%%%%%%%%%%%%%%%%%%%%%%%%%%%%%%%%%%%%%%%%%%%%%%%%%%%%%%%%%%%%%%%%%%%%%%%%%%%%%%%
%%%%%%%%%%%%%%%%%%%%%%%%%%%%%%%%%%%%%%%%%%%%%%%%%%%%%%%%%%%%%%%%%%%%%%%%%%%%%%%%%%%%%%%%%%%%%%%%%%%%%%%%%
\chapter{El fichero de Entrada}
\label{chap:input}

Una de las piezas fundamentales en \lpmd para la corrida de una simulaci\'on molecular son los ficheros iniciales de configuraci\'on del sistema. Para el correcto funcionamiento se necesita un fichero de sistema, este fichero espec\'ifica casi el 100\% de los requerimientos de la simulaci\'on y en ocaciones el 100\%. Aunque en la mayor\'ia de los casos se requiere un fichero adicional en donde se encuentran las posiciones at\'omicas de los \'atomos pertenecientes a la celda de simulaci\'on. Es por eso que en primer lugar revisaremos los arhicovs de posiciones at\'omicas antes de revisar el fichero de sistema.

\section{Fichero Con Posiciones Atomicas}

Actualmente \lpmd soporta dos formatos de ficheros para especificar las posiciones at\'omicas, uno es \textbf{xyz} y el otro \textbf{lpmd}, que son archivos ASCII manejables en cualquier editor.

\cajatx{Sin embargo, la version de prueba 0.4.X presentaba problemas con tabulaciones en un fichero de entrada, por lo que se recomienda evitarlo o bien chequear que la salida entregue las posiciones correctas.}

Veamos a continuaci\'on cada uno de estos formatos.

%%%%%%%%%%%%%%%%%%%%%%%%%%%%%%%%%%%%%%%%%%%%%%%%%%%%%%%%%%%%%%%%%
%%%%%%%%%%%%%%%%%%%%%%%%%%%%%%%%%%%%%%%%%%%%%%%%%%%%%%%%%%%%%%%%%
\subsection{Fichero .xyz}

Es el estandar de ficheros \verb|xyz| utilizado en muchos c\'odigos de simulacion computacional, es un archivo simple que cuenta con la informaci\'on de las posiciones at\'omicas del sistema. La estructura de un fichero es :
\begin{center}
\begin{tabular}{l|l}
 \verb|N| & Especif\'ica el n\'umero de \'atomos en la celda \\
 \verb|comment| & Una l\'inea adicional de comentario, titulo etc. \\
 \verb|Sym X Y Z| & S\'imbolo at\'omico y las posiciones en cordenadas cartesianas. \\
\end{tabular}
\end{center}

%%%%%%%%%%%%%%%%%%%%%%%%%%%%%%%%%%%%%%%%%%%%%%%%%%%%%%%%%%%%%%%%%
%%%%%%%%%%%%%%%%%%%%%%%%%%%%%%%%%%%%%%%%%%%%%%%%%%%%%%%%%%%%%%%%%
\subsection{Fichero .lpmd}

Es un fichero con las posiciones escalas de los \'atomos que forman la celda. el fichero es de tipo ASCCI y su estructura principal esta dada por:

\begin{center}
 \begin{tabular}{l|l}
 \verb|LPMD VERSION X.X | & Especif\'ica la version del fichero \verb|.lpmd| \\
 \verb|cell propoerties | & Propiedades de la celda, pueden ser 3 vectores o largos y angulos. \\
 \verb|Sym sx sy sz| & S\'imbolo at\'omico y las posiciones escalads en cada eje.\\
\end{tabular}
\end{center}


\section{El Fichero de configuraci\'on .control}

Es el fichero principal para realizar la simulaci\'on computacional. Es por eso que en primer lugar se har\'a una descripci\'on general y luego veremos cada uno de los flags principales.

Entre las cosas a considerar en un fichero de sistema, est\'an:

\begin{itemize}
 \item \# Es una linea de comentario.
 \item Evite los \verb|\tab| dentro del c\'odigo (se corregir\'a).
 \item Pese a que puede ser aleatorio el orden de los flaggs, se recomienda un orden.
 \item siempre defina los modulos antes de utilizarlos.
\end{itemize}

%%%%%%%%%%%%%%%%%%%%%%%%%%%%%%%%%%%%%%%%%%%%%%%%%%%%%%%%%%%%%%%%%
%%%%%%%%%%%%%%%%%%%%%%%%%%%%%%%%%%%%%%%%%%%%%%%%%%%%%%%%%%%%%%%%%
\subsection{Celda de Simulaci\'on.}

Son todas las propiedades que describen la celda de simulaci\'on. La mayor\'ia de las opciones del c\'odigo en esta parte estan descripas en \textbf{Propiedades Generales}. Sin embargo hay un par de opciones que siempre deben tenerse en cuenta en la primera l\'inea de nuestro archivo de sistema.

\subsubsection{cell}

El flag cell es utilizado para describir la celda de simulaci\'on y asignar las propiedades de \'esta. Generalmente una celda de simulaci\'on puede venir descrita ya en el formato del fichero de entrada, sin embargo hay formatos, como el \textbf{xyz}, que no poseen la descripci\'on de la celda, es por eso que es necesario utilizar siempre esta opci\'on. Si despues de dar la opci\'on \textbf{cell} se utiliza un formato de entrada descriptivo, como \textbf{lpmd}, \'este \'ultimo es el valor que toma la celda.

\cajatx{ 
\begin{tabular}{lcl}
 a & = & indica el largo de la celda en \textbf{a}.\\
 b & = & indica el largo de la celda en \textbf{b}.\\
 c & = & indica el largo de la celda en \textbf{c}.\\
 alpha & = & indica el \'angulo $\alpha$.\\
 beta & = & indica el \'angulo $\beta$.\\
 gamma & = & indica el \'angulo $\gamma$.\\
\end{tabular}
}

Formas de uso :

\control{cell a=10 b=5 c=5 alpha=45 beta=90 gamma=90}

%%%%%%%%%%%%%%%%%%%%%%%%%%%%%%%%%%%%%%%%%%%%%%%%%%%%%%%%%%%%%%%%%
%%%%%%%%%%%%%%%%%%%%%%%%%%%%%%%%%%%%%%%%%%%%%%%%%%%%%%%%%%%%%%%%%
\subsection{Entrada Salida}

\subsubsection{input}
Existen distintas formas de ingresas un fichero de configuraci\'on inicial, el fichero cuenta con las posiciones atomicas de los \'atomos que hay en la simulaci\'on, donde se debe indicar el m\'odulo que indica el tipo de fichero que lo carga. Adem\'as existe la posibilidad de cargar distintos m\'odulos de entrada como generadores de celda.

\cajatx{ 
\begin{tabular}{lcl}
 module & = & indica el m\'odulo para cargar la celda.\\
 file & = & indica el fichero que carga la celda.\\
 level & = & indica el nivel del fichero.\\
\end{tabular}
}

Formas de uso

\begin{itemize}
 \item Carga desde un fichero XYZ.
\control{input module=xyz file=fichero.xyz level=0}
 \item Inicializa una celda del tipo fcc
\control{input module=fcc a=1 nx=3 ny=3 nz=3}
 \item Inicializa con metodo skewstart
\control{input module=skewstart atoms=108 symbol=Ar}
\end{itemize}

\subsubsection{output}

Con el par\'ametro \verb|control| se especifican las opciones de salida de las configuraciones at\'omicas de nuestra simulaci\'on, los formatos de salidas son complementamente modulares y pueden ser implementados por los usuarios, sin embargo en la version 0.1 del set de plugins \verb|lpmd-plugins| ya se encuentran disponibles los m\'odulos \textbf{xy} y \textbf{lpmd}. Los argumentos requeridos por el par\'ametro son:

\cajatx{ 
\begin{tabular}{lcl}
 module & = & indica el m\'odulo (formato) de \\
&&salida de la simulacion.\\
 file & = & indica el fichero en el que graba.\\
 level & = & indica el nivel del modulo de salida.\\
 each & = & indica cada cuantos pasos la celda \\
&&es gabada en el fichero.\\
\end{tabular}
}

Formas de uso

\begin{itemize}
 \item Grabando la simulacion en un fichero XYZ (nivel 0), cada 20 steps.
\control{output module=xyz file=fichero.xyz level=0 each=20}
 \item Grabando la simulacion en fichero LPMD (nivel 1), cada 1 step.
\control{output module=lpmd file=fichero.xyz level=1 each=1}
\end{itemize}


\subsubsection{restore}

Es utilizado para restaurar una simulaci\'on a partir de un punto en que se produjo un corte de luz o cualquier otro tipo de falla. El punto de restauraci\'on es a partir de el \'ultimo dumping realizado por la simulacion, dado por la orden ``dumping''.

Actualmente no se han realizado pruebas exhaustivas de este punto, pero deber\'ia funcionar sin problemas, porfavor si encuentra un bug, reportelo.

%%%%%%%%%%%%%%%%%%%%%%%%%%%%%%%%%%%%%%%%%%%%%%%%%%%%%%%%%%%%%%%%%
%%%%%%%%%%%%%%%%%%%%%%%%%%%%%%%%%%%%%%%%%%%%%%%%%%%%%%%%%%%%%%%%%
\subsection{Propiedades Generales}
\subsubsection{repeat}
Indica el nivel de repeticion de la celda en cada eje, no utilize esto en celdas para las cuales las condiciones de borde no sean peri\'odicas, podr\'ia conseguir resultados no deseados, recomendamos esto encarecidamente.

Forma de uso

\begin{itemize}
 \item Replicando una celda 2 veces en \verb|x| e \verb|y| pero ninguna en \verb|z|
\control{repeat 2 2 1}
\end{itemize}

\subsubsection{charge}
Set de las cargas en eV para las especies at\'omicas. Estos valores de las cargas, son seteados principalmente para utilizaci\'on de potenciales interatomicos en los cuales se utilizan las cargas de los atomos involucrados.

Forma de uso

\begin{itemize}
 \item Seteando las cargas de los atomos de O y Ge.
\control{charge O=XX \\ charge Ge=XX}
\end{itemize}

\subsubsection{periodic}
Indica la periodicidad de la celda, en cada eje. Al bloquear la periodicidad en un eje, este se ve ``modificado'' en ambos lados de la celda, revise con cuidado estas opciones.

\subsubsection{steps}
N\'umero de pasos de la simulaci\'on de DM.

\subsubsection{distcache}
Utiliza cache de distancias, puede ahorrar tiempo en algunas simulaciones, recomendamos que siempre se utilize.

Forma de uso

\begin{itemize}
 \item Aplicando el usao de distance-cache para la simulaci\'on.
\control{distcache}
\end{itemize}

\subsubsection{dumping}
Genera una salida global del sistema para poder restaurar a partir de ese punto.

\subsubsection{monitor}
Cada cuantos pasos la simulacion muestra las propiedades globales.

\subsubsection{temperature}
Indica la temperatura inicial del sistema. \'Este es el m\'etodo m\'as utilizado por los progamas de \textbf{DM} para ingresar las velocidades iniciales de los \'atomos del sistema.

\subsubsection{startinfo}
Muestra ono ifnormacion inicial del sistema, celda y modulos no usados.

Forma de uso

\begin{itemize}
 \item Mostrando en el inicio las posiciones at\'omicas y los m\'odulos no usados.
\control{startinfo coords=true unused=true}
 \item Ocultando la informaci\'on de las posiciones at\'omicas iniciales y los modulos no usados.
\control{startinfo coords=false unused=false}
\end{itemize}

%%%%%%%%%%%%%%%%%%%%%%%%%%%%%%%%%%%%%%%%%%%%%%%%%%%%%%%%%%%%%%%%%
%%%%%%%%%%%%%%%%%%%%%%%%%%%%%%%%%%%%%%%%%%%%%%%%%%%%%%%%%%%%%%%%%
\subsection{M\'odulos : Propiedades del sistema}
\subsubsection{tempscaling}
Escalamiento de velocidades cl\'asico, seg\'un la temperatura indicada.
\subsubsection{berendsen}
Termostato de berendsen, mucho m\'as suave que tempscaling.

%%%%%%%%%%%%%%%%%%%%%%%%%%%%%%%%%%%%%%%%%%%%%%%%%%%%%%%%%%%%%%%%%
%%%%%%%%%%%%%%%%%%%%%%%%%%%%%%%%%%%%%%%%%%%%%%%%%%%%%%%%%%%%%%%%%
\subsection{M\'odulos : Propiedades de Celda}
\subsubsection{cellscaling}
Escala la celda cierto porcentaje, puede ser por eje o total.
\subsubsection{crystalfcc}
Genera una celda cristalina del tipo fcc.
\subsubsection{crystalsc}
Genera una celda cstalino del tipo sc.
\subsubsection{lpmd}
M\'odulo lectura/escritura de formato lpmd.
\subsubsection{xyz}
M\'odulo lectura/escrtura de formato xyz.
\subsubsection{skewstart}
Genera una celda con m\'etodo SkewStart (Refson).
\subsubsection{cellmanager}
Especifica el manejador de celda, actualmente hay dos m\'odulos manejadores de celda, minima imagen y listas linkeadas. cada m\'odulo debe declararse previemente con los parametros requeridos.

%%%%%%%%%%%%%%%%%%%%%%%%%%%%%%%%%%%%%%%%%%%%%%%%%%%%%%%%%%%%%%%%%
%%%%%%%%%%%%%%%%%%%%%%%%%%%%%%%%%%%%%%%%%%%%%%%%%%%%%%%%%%%%%%%%%
\subsection{M\'odulos : Potenciales Interat\'omicos}
\subsubsection{lennardjones}
El m\'odulo \textbf{lennardjones} hace referencia al potencial de Lennard-Jones, que es de la forma,
$$U(r_{ij}) = 4\epsilon\left(\left(\frac{\sigma}{r_{ij}}\right)^{12}-\left(\frac{\sigma}{r_{ij}}\right)^6\right)$$
En donde $r_{ij}$ es la distancia interat\'omica de los \'atomos $i$ y $j$. La implementaci\'on del m\'etodo virtual, queda entonces como :
\begin{verbatim}
double LennardJones::pairEnergy(const double & r) const
{
 double rtmp=sigma/r;
 double r6 = rtmp*rtmp*rtmp*rtmp*rtmp*rtmp;
 double r12 = r6*r6;
 return 4.0e0*epsilon*(r12 - r6);
}
\end{verbatim}

Para el c\'alculo de Fuerzas, la forma del potencial que nos interesa, es aquella fuerza que siente el \'atomo $i$ producida por el \'atomo $j$, la que debe ser implementada en el plugin, para potentiales de pares, para el caso del potencial de Lennard Jones, la fuerza esta dada por,

$$F_{ij} = \frac{-48.0\epsilon}{r_{ij}^2}\left( \left(\frac{\sigma}{r_{ij}}\right)^{12} + \frac{1}{2}\left(\frac{\sigma}{r_{ij}}\right)^6 \right) \vec{r_{ij}}$$

en donde $\vec{r_{ij}}$ es el vector distancia entre los \'atomos $i$ y $j$, y $r_{ij}$ es la distancia entre ellos. La implementaci\'on de la fuerza de pares requerida por la API es :

\begin{verbatim}
Vector LennardJones::pairForce(const Vector & r) const
{
 double rr2 = r.Mod2();
 double r6 = pow(sigma*sigma / rr2, 3.0e0);
 double r12 = r6*r6;
 double ff = -48.0e0*(epsilon/rr2)*(r12 - 0.50e0*r6);
 Vector fv = r;
 fv.Scale(ff);
 return fv;
}
\end{verbatim}

Las palbras reservadas por el plugin \textbf{lennardjones}, son :

\cajatx{
\begin{tabular}{lcl}
 epsilon & = & indica el valor de epsilon.\\
 sigma & = & indica el valor de sigma.\\
 cutoff & = & indica el cutoff del potencial.\\
\end{tabular}
}

Las unidades en que deben ser ingresados, las constantes, deben ser basadas en que las distancias estan en [\AA] y la energ\'ia debe ser adquirida en [eV].

\subsubsection{fastlj}
Utiliza Potencial de LJ tabulado.
\subsubsection{morse}
Utiliza Potencial de Morse para la interacci\'on de las especis at\'omicas.
\subsubsection{constantforce}
Mantiene una fuerza constante.
\subsubsection{harmonic}
Potencial arm\'onico entre especies at\'omicas.
\subsubsection{buckingham}
Potencial de Buckingham IMPORANTE, no incluye parte coulombiana.
\subsubsection{suttonchen}
Este potencial, se utiliza para interacciones de atomos met\'alicos, es por eso que el plugin \textbf{suttonchen} implementa los m\'etodos virtuales de \verb|metalpotential|, que cuentan con una parte de pares y otro t\'ermino de muchos cuerpos. La parte asociada al t\'ermino de pares, est\'a dado por,

$$U(r_{ij}) = \left(\frac{a}{r_{ij}}\right)^n$$

en donde $r_ij$ es la distancia entre un atomo $i$ y otro atomo $j$ del sistema. El t\'ermino de mcuhos cuerpos esta dado por

$$F(\rho_{i}) = -c\epsilon\sqrt{\rho_i}$$

en donde,

$$\rho_i = \sum_{j\neq i} \left(\frac{a}{r_{ij}}\right)^m$$

lo que corresponde a una densidad local del atomo $i$, que depende de todos los atomos $j$ cercanos a \'el, \'esta densidad local sin embargo, debe ser corregida para el caso de suttonchen (note que no todos los potenciales asociados a los metales requieren de esta correcci\'on, pero \verb|metalpotential| lo requiere, as\'i que en ocaciones debe ser cero).

Para el potencial de SuttonChen, la correcci\'on de la densidad esta dada por,

$$\delta\rho_i=\frac{4\pi\overline{\rho}a^3}{m-3}\left(\frac{a}{r_{met}}\right)^{(m-3)}$$

Esta correcci\'on de la densidad debe ser aplicada inmediatamente luego de ser calculada la densidad local. La correcci\'on de la energ\'ia para Sutton Chen, se obtiene de esta forma con :

$$\delta U_1 = \frac{2\pi N\overline{\rho}\epsilon a^3}{n-3}\left(\frac{a}{r_{met}}\right)^{n-3}$$

Hay que notar que $\delta U_2$ no es requerido si $\rho_i$ ya fue corregido, con $\delta U_2$ de la forma

$$\delta U_2 = -\frac{4\pi\overline{\rho}a^3}{m-3}\left(\frac{a}{r_{met}}\right)^{n-3}\left<\frac{Nc\epsilon}{2\sqrt{\rho_i^0}}\right>$$

La implementaci\'on de esta energ\'ia para el potencial met\'alico, 

\begin{verbatim}
double SuttonChen::pairEnergy(const double &r) const
{
 return e*pow((a/r),n);
}

double SuttonChen::rhoij(const double &r) const
{
 return pow((a/r),m);
}

double SuttonChen::F(const double &rhoi) const
{
 return -c*e*sqrt(rhoi);
}

double SuttonChen::deltarhoi(const double &rhobar) const
{
 return (4*M_PI*rhobar*a*a*a/(m-3))*pow(a/rcut,m-3);
}

double SuttonChen::deltaU1(const double &rhobar, const int &N) const
{
 double f = 2*M_PI*N*rhobar*e*a*a*a/(n-3);
 return f*pow(a/rcut,n-3);
}
\end{verbatim}


\subsubsection{nullpairpotential}
Potencial de pares nulo entre especies.
\subsubsection{nullpotential}
Potencial nulo entre especies.

%%%%%%%%%%%%%%%%%%%%%%%%%%%%%%%%%%%%%%%%%%%%%%%%%%%%%%%%%%%%%%%%%
%%%%%%%%%%%%%%%%%%%%%%%%%%%%%%%%%%%%%%%%%%%%%%%%%%%%%%%%%%%%%%%%%
\subsection{M\'odulos : Integradores}
\subsubsection{beeman}
Integrador de beeman.
\subsubsection{euler}
Integrador de Euler.
\subsubsection{nullintegrator}
Integrador nulo.
\subsubsection{velocityverlet}
Utiliza metodo velocity verlet para integrar.
\subsubsection{verlet}
Utiliza metodo verlet para integrar.

%%%%%%%%%%%%%%%%%%%%%%%%%%%%%%%%%%%%%%%%%%%%%%%%%%%%%%%%%%%%%%%%%
%%%%%%%%%%%%%%%%%%%%%%%%%%%%%%%%%%%%%%%%%%%%%%%%%%%%%%%%%%%%%%%%%
\subsection{M\'odulos : CellManager}
Estos m\'odulos son los encargados de generar las \textbf{lista inteligentes} para poder realizar simulaciones o calculos de Din\'amica Molecular, los dos plugins implementados a la fecha son:

\subsubsection{minimumimage}
Uiliza el m\'etodo de m\'inima imagen para realizar los procesos.

\subsubsection{linkedcell}
Utiliza el m\'etodo de listas linkeadas, \'este m\'etodo es mucho m\'as rapido que el m\'etodo de m\'inima imagen, y de por s\'i es el m\'as utilizado en la mayor\'ia de los c\'odigos de \textbf{DM}

%%%%%%%%%%%%%%%%%%%%%%%%%%%%%%%%%%%%%%%%%%%%%%%%%%%%%%%%%%%%%%%%%
%%%%%%%%%%%%%%%%%%%%%%%%%%%%%%%%%%%%%%%%%%%%%%%%%%%%%%%%%%%%%%%%%
\subsection{M\'odulos : Visualizadores}
\subsubsection{povray}
Genera diretorios con ficheros pov para visualizar el sistema. Este m\'odulo genera un set de archivos \verb|pov| los cuales son ubicados dentro de un directorio, para un posterior \textit{rendering} para el dise\'no de peliculas o fotograf\'ias de la simulaci\'on.

Los argumentos requeridos (no todos) por el m\'odulo povray, son los siguientes,

\cajatx{
\begin{tabular}{lcl}
 header & = & Es un nombre previo al nombre \\
&&de los ficheros \textbf{pov} que ser\'an generados.\\
 direct & = & Nombre del directorio que se crear\'a.\\
 text & = & Orden para poner texto en \\
&&diferentes posiciones.\\
 background & = & Color del fondo de la imagen. \\
 rotate & = & Orientaci\'on de la c\'amara.\\
 logo   & = & Si desea anadir una imagen. \\
 box  & = & Muestra o no la \textbf{celda} (True/False).\\
 camera & = & Posici\'on de la c\'amara \\
 &&(recomendamos default).\\
\end{tabular}
}

Dentro de cada uno de \'estos argumentos, el que m\'as cabe detallar es \textbf{text}, el formato de ingreso de textos para la vsualizaci\'on, es el siguiente,

\control{text "Titulo" <pos> <color> [size] (extra)}

Ac\'a las opciones son en el orden requerido. El t\'itulo puede ser cualquier texto, si se utiliza el s\'imbolo ``\% '' entonces el valor de \verb|extra| ser\'a reemplazado (Actualmente : Temp, Step). Las opciones \verb|<pos>| y \verb|<color>| son vectores que deben ingresarse en el formato \verb|<x,y,z>| y corresponden a la posici\'on del texto y los colores, exiten valores por defecto, tales como \verb|<green>|, \verb|<red>| o posiciones como \verb|<dl>| (abajo a la izquierda) que pueden ser utilizadas. Finalmente [size] es el tama\~no escalado del texto.

A continuaci\'on un ejemplo t\'ipico de uso de \verb|povray| en una simulaci\'on.

\begin{verbatim}
use povray
    header shoot-
    direct movie
    text "Modelacion de Ar" <dl> <green> [1] ()
    text "Step = %" <3,3,3> <red> [1] (Step)
    text "Temperatura : % [K]" <uc> <blue> [1] (Temp)
    text "http://www.gnm.cl/" <dr> <green> [0.5] ()
    background <0.2,0.1,0.4>
    rotate <0,0,0>
    logo "logo-v2.gif" 1.5 <cr>
enduse
\end{verbatim}

Luego de que los archivos son creados en el directorio ``movie'', es importante que coloque en ese directorio el logo al que los archivos hacen referencia \verb|logo-v2.gif|.

%%%%%%%%%%%%%%%%%%%%%%%%%%%%%%%%%%%%%%%%%%%%%%%%%%%%%%%%%%%%%%%%%
%%%%%%%%%%%%%%%%%%%%%%%%%%%%%%%%%%%%%%%%%%%%%%%%%%%%%%%%%%%%%%%%%
\subsection{M\'odulos : Propiedades Est\'aticas}
\subsubsection{angdist}
Calcula la distribucion angular de la celda
\subsubsection{cordnum}
Calcula el n\'umero de cordinaci\'on de la celda.
\subsubsection{cordnumfunc}
Calcula el n\'umero de cordinaci\'on de la celda.
\subsubsection{gdr}
Caulcula la funcion de distribucion de pares de la celda.

%%%%%%%%%%%%%%%%%%%%%%%%%%%%%%%%%%%%%%%%%%%%%%%%%%%%%%%%%%%%%%%%%
%%%%%%%%%%%%%%%%%%%%%%%%%%%%%%%%%%%%%%%%%%%%%%%%%%%%%%%%%%%%%%%%%
\subsection{M\'odulos : Propiedades Din\'amicas}
\subsubsection{vacf}
Calcula la funci\'on de autocorrelaci\'on de velocidades de la celda.
\subsubsection{msd}
Calcula el desplazamiento cuadratico medio del sistema.

\section{Ficheros de Salida}

%%%%%%%%%%%%%%%%%%%%%%%%%%%%%%%%%%%%%%%%%%%%%%%%%%%%%%%%%%%%%%%%%
%%%%%%%%%%%%%%%%%%%%%%%%%%%%%%%%%%%%%%%%%%%%%%%%%%%%%%%%%%%%%%%%%
\subsection{Tipos de fichero de salida}
\lpmd Tiene dos tipos de fichero de salida, uno que es generado usualmente por la opci\'on \verb|output| dentro del fichero de control y otro es la salida standard, que por defecto va a \verb|cout|, pero que nosotros recomendamos enviar a un fichero.

%%%%%%%%%%%%%%%%%%%%%%%%%%%%%%%%%%%%%%%%%%%%%%%%%%%%%%%%%%%%%%%%%
%%%%%%%%%%%%%%%%%%%%%%%%%%%%%%%%%%%%%%%%%%%%%%%%%%%%%%%%%%%%%%%%%
\subsection{Salida standard}
Esta es la salida que muestra en pantalla lpmd, la forma de enviar esta salida a un fichero, durante la ejecuci\'on de lpmd, es

\control{lpmd -s fichero.control > salida.out}

el fichero \verb|salida.out| tiene toda la informaci\'on que deb\'ia salir a pantalla utilizando lpmd, entre ella se encuentran,

\begin{itemize}
 \item Descripci\'on completa de la celda
 \item Informacion de \verb|startinfo|
 \item Infomraci\'on de m\'odulos utilizados y variables de c/u.
 \item Energias, Temperatura, Presion y Volumen, seg\'un \verb|monitor|.
\end{itemize}


%%%%%%%%%%%%%%%%%%%%%%%%%%%%%%%%%%%%%%%%%%%%%%%%%%%%%%%%%%%%%%%%%
%%%%%%%%%%%%%%%%%%%%%%%%%%%%%%%%%%%%%%%%%%%%%%%%%%%%%%%%%%%%%%%%%
\subsection{Fichero generado por output}
Este fichero se gener\'o con el nombre entregado en el fichero de control a la linea \textbf{output}, en \'el se encuentran las configuraciones atomicas de la \textbf{DM} y suelen ser los ficheros a partir de los cuales suelen crearse animaciones y an\'alisis detallados de la simulaci\'on.


%%%%%%%%%%%%%%%%%%%%%%%%%%%%%%%%%%%%%%%%%%%%%%%%%%%%%%%%%%%%%%%%%
%%%%%%%%%%%%%%%%%%%%%%%%%%%%%%%%%%%%%%%%%%%%%%%%%%%%%%%%%%%%%%%%%
\subsection{Salida de errores}
La versi\'on actual no redrecciona informacion a la salida de errores, salvo los que realmente corresponden a errores de ejecuci\'on de \lpmd.

%%%%%%%%%%%%%%%%%%%%%%%%%%%%%%%%%%%%%%%%%%%%%%%%%%%%%%%%%%%%%%%%%%%%%%%%%%%%%%%%%%%%%%%%%%%%%%%%%%%%%%%%%
%%%%%%%%%%%%%%%%%%%%%%%%%%%%%%%%%%%%%%%%%%%%%%%%%%%%%%%%%%%%%%%%%%%%%%%%%%%%%%%%%%%%%%%%%%%%%%%%%%%%%%%%%
%CAPITULO 3%%%%%%%%%%%%%%%%%%%%%%%%%%%%%%%%%%%%%%%%%%%%%%%%%%%%%%%%%%%%%%%%%%%%%%%%%%%%%%%%%%%%%%%%%%%%%%
%%%%%%%%%%%%%%%%%%%%%%%%%%%%%%%%%%%%%%%%%%%%%%%%%%%%%%%%%%%%%%%%%%%%%%%%%%%%%%%%%%%%%%%%%%%%%%%%%%%%%%%%%
%%%%%%%%%%%%%%%%%%%%%%%%%%%%%%%%%%%%%%%%%%%%%%%%%%%%%%%%%%%%%%%%%%%%%%%%%%%%%%%%%%%%%%%%%%%%%%%%%%%%%%%%%
\chapter{Desarrollando M\'odulos}
\label{chap:own}

\section{Idea Principal}

Una de las caracter\'isticas princpales de \lpmd con respecto a otros c\'odigos de Din\'amica Molecular es su gran \textit{modularidad} lo que hace que muchas propiedades de un ciclo regular de din\'amica molecular sean modificables facilmentes, por ejemplo un cilco de din\'amica molecular consta de muchas \textbf{piezas} constantes, tales como los potenciales, integradores o bien una propiedad que puede ser calculada de forma instantanea o que requiere una correlaci\'on temporal del sistema.

Consideremos por ejemplo :

Se puede observar claramente que existen \textbf{bloques} en donde la caracter\'istica principal de cada uno de ellos en \lpmd es que son modificables por diferentes tipos de \textbf{m\'odulos} que \textit{encajan} perfectamente en estos bloques, estos m\'odulos pueden ser din\'amicos lo que da una ventaja significativa a la hora de desarrollar el c\'odigo necesario para trabajar con \'el.

En este cap\'itulo se exponen las distintas piezas \textbf{modificables} de \lpmd que har\'an de este un c\'odigo mucho m\'as \'util para el desarrollo de distintas investigaciones con una misma herramienta.

\section{Desarrollando un Potencial}

Una de las piezas fundamentales en la din\'amica molecular, es la integraci\'on de un potencial interat\'omico entre las particulas que componen el sistema, es por eso que \lpmd facilita 

\section{Desarrollando una Propiedad}

Durante una simulaci\'on de din\'amica molecular una de las herramientas m\'as utilizadas  son las propiedades f\'isicas del sistema, las que son, een ocaciones, comparables con resultados experimentales provenientes del laboratorio. Sin embargo estas propiedades, no siempre pueden ser evaluadas ya que los programas no cuentan con ellas, o bien deben implementarse para resolver este problema, aprendiendo a tomar configuraciones de salida de otros programas, para nuestros fines.

Para resolver esta situaci\'on \lpmd calcula propiedades de un sistema atomico, de forma modular, es decir cada uno de nosotros puede \textbf{programar} la propidad que necesesita para su evaluacion, instantanea, o en ocaciones temporal.

\lpmd separa las propiedades de una celda de simulaci\'on en 2 tipos :

\begin{itemize}
 \item Propiedades Instantaneas.
 \item Propiedades Temporales.
\end{itemize}

En donde, las instantaneas corresponden a las propiedades que pueden calcularse en un instante de tiempo y no dependen de configuraciones previas del sistema (como funci\'on de distribuci\'on de pares), en cambio las temporales son aquellas que dependen de configuraciones previas del sistema, por ejemplo la funci\'on de autocorrelaci\'on de velocidades.

A continuaci\'on se mostrar\'a la estructura b\'asica necesaria para implementar propiedades instantaneas y temporales en el programa \lpmd y as\'i utilizarlas durante la ejecuci\'on de \lpmd o bien para trabajar con nuevas utilidades.

%%%%%%%%%%%%%%%%%%%%%%%%%%%%%%%%%%%%%%%%%%%%%%%%%%%%%%%%%%%%%%%%%
%%%%%%%%%%%%%%%%%%%%%%%%%%%%%%%%%%%%%%%%%%%%%%%%%%%%%%%%%%%%%%%%%
\subsection{Instant\'anea}

Las propiedades m\'as simples para comenzar a implementar son las instantaneas, dentro de este tipo de propiedades tenemos aquellas que retornan por valor un solo n\'umero real (energ\'ia, temperatura, etc.) y otras que retornan una matriz de numeros reales (como g(r) o distribuci\'on angular, etc.), para esto es necesario ubicarse dentro del directorio \verb|lib| de \lpmd y generar dos nuevos archivos que constan con informacion b\'asica de la propiedad.

Consideremos por ejemplo la funci\'on de distribuci\'on de pares (\verb|g(r)|) (\textbf{nota : esto ya existe en el directorio, ac\'a se muestra a modo de ejemplo.}), para ello generamos dos nuevos ficheros dentro de \verb|lib| :

\begin{center}
 \verb|touch gdr.cc gdr.h|
\end{center}

%%%%%%%%%%%%%%%%%%%%%%%%%%%%%%%%%%%%%%%%%%%%%%%%%%%%%%%%%%%%%%%%%
%%%%%%%%%%%%%%%%%%%%%%%%%%%%%%%%%%%%%%%%%%%%%%%%%%%%%%%%%%%%%%%%%
\subsection{Temporal}

Las propiedades temporales n est\'an dise\~nadas para ser evaluadas duratne a siulaci\'on, sin embargo es facil su implementacion en la API, lo que puede llevar a utilziarlas en otros c\'odigos, tales como fumody.

La idea es utilizar los archivos de configuraci\'on de salida de lpmd.

\section{Desarrollando Integrador}

Un integrador cumple la funcion de ...

\section{Desarrollando Utilidades}

La API (liblpmd) es la principal herramienta que deja lpmd, que puede ser utilizada no solo por \'el sino que por utilidades que nosotros deseamos dise\~nar.

%%%%%%%%%%%%%%%%%%%%%%%%%%%%%%%%%%%%%%%%%%%%%%%%%%%%%%%%%%%%%%%%%%%%%%%%%%%%%%%%%%%%%%%%%%%%%%%%%%%%%%%%%
%%%%%%%%%%%%%%%%%%%%%%%%%%%%%%%%%%%%%%%%%%%%%%%%%%%%%%%%%%%%%%%%%%%%%%%%%%%%%%%%%%%%%%%%%%%%%%%%%%%%%%%%%
%CAPITULO 4%%%%%%%%%%%%%%%%%%%%%%%%%%%%%%%%%%%%%%%%%%%%%%%%%%%%%%%%%%%%%%%%%%%%%%%%%%%%%%%%%%%%%%%%%%%%%%
%%%%%%%%%%%%%%%%%%%%%%%%%%%%%%%%%%%%%%%%%%%%%%%%%%%%%%%%%%%%%%%%%%%%%%%%%%%%%%%%%%%%%%%%%%%%%%%%%%%%%%%%%
%%%%%%%%%%%%%%%%%%%%%%%%%%%%%%%%%%%%%%%%%%%%%%%%%%%%%%%%%%%%%%%%%%%%%%%%%%%%%%%%%%%%%%%%%%%%%%%%%%%%%%%%%
\chapter{Ejemplos.}
\label{chap:exa}

Ac\'a encontrar\'a algunos ejemplos de simulaciones realizadas con lpmd, en su ultima versi\'on estable.

\section{Ejemplos B\'asicos}

\subsection{Dimero 1D}
Muestra un dimero 

\subsection{Mallas 2D}
Una red Triangular.

\subsection{Sistemas Cristalinos 3D}

\subsubsection{Arg\'on}

A continuaci\'on una simulaci\'on de Ar con 108 \'atomos, en la cu\'al se realizan distintos escalamientos de temperatura. El ejemplo puede descargarlo completamente de,

\cajatx{http://wwww.gnm.cl/software/lpmd/examples/ar108-1.tgz}

Veamos algunas partes fundamentales del fichero de control.


\begin{verbatim}

################################################################
# System file of Ar gas 
# using LPMD
################################################################

####################
#CELL PROPERTIES####
####################
cell crystal a=17.1191 b=17.1191 c=17.1191 alpha=90.0 beta=90.0 gamma=90.0
input module=lpmd file=argon108.lpmd level=0
output module=xyz file=output.xyz each=20 level=0
charge Ar 0.0

###################
#MD PROPERTIES#####
###################
repeat 1 1 1 
periodic true true true
steps 50000
dumping 10000 ljargon.dump
monitor 10
temperature 84

###################
##MODULES DEF######
###################

use lennardjones as lj_Ar
    sigma 3.41
    epsilon 0.0103408
    cutoff 8.5
enduse

use beeman
    dt 10.0
enduse

use tempscaling as ts0
    from 84
    to 84
enduse

use tempscaling as ts1
    from 84
    to 300
enduse

use tempscaling as ts2
    from 300
    to 300
enduse

use tempscaling as ts3
    from 300
    to 1
enduse

use tempscaling as ts4
    from 1
    to 1
enduse

use povray2
    header shoot-
    direct movie
    text "Modelacion de Ar" <dl> <green> [1] ()
    text "Step = %" <3,3,3> <red> [1] (Step)
    text "Temperatura : % [K]" <uc> <blue> [1] (Temp)
    text "http://www.gnm.cl/" <dr> <green> [0.5] ()
    background <0.2,0.1,0.4>
    rotate <0,0,0>
    logo "logo-v2.gif" 1.5 <cr>
enduse

###################
#MOD APPLICATION###
###################
potential lj_Ar Ar Ar
integrator beeman
apply ts0 start=1 end=10000 each=10
apply ts1 start=10000 end=20000 each=10
apply ts2 start=20000 end=30000 each=10
apply ts3 start=30000 end=40000 each=10
apply ts4 start=40000 end=50000 each=10
visualize povray2 start=1 end=50000 each=100
\end{verbatim}


\subsection{Ejemplos m\'as Complejos}
Pared fija bajo presion, o una caja que suelta un gas.

%%%%%%%%%%%%%%%%%%%%%%%%%%%%%%%%%%%%%%%%%%%%%%%%%%%%%%%%%%%%%%%%%%%%%%%%%%%%%%%%%%%%%%%%%%%%%%%%%%%%%%%%%
%%%%%%%%%%%%%%%%%%%%%%%%%%%%%%%%%%%%%%%%%%%%%%%%%%%%%%%%%%%%%%%%%%%%%%%%%%%%%%%%%%%%%%%%%%%%%%%%%%%%%%%%%
%CAPITULO 5%%%%%%%%%%%%%%%%%%%%%%%%%%%%%%%%%%%%%%%%%%%%%%%%%%%%%%%%%%%%%%%%%%%%%%%%%%%%%%%%%%%%%%%%%%%%%%
%%%%%%%%%%%%%%%%%%%%%%%%%%%%%%%%%%%%%%%%%%%%%%%%%%%%%%%%%%%%%%%%%%%%%%%%%%%%%%%%%%%%%%%%%%%%%%%%%%%%%%%%%
%%%%%%%%%%%%%%%%%%%%%%%%%%%%%%%%%%%%%%%%%%%%%%%%%%%%%%%%%%%%%%%%%%%%%%%%%%%%%%%%%%%%%%%%%%%%%%%%%%%%%%%%%
\chapter{Paralelizaci\'on}

Por qu\'e paralelizar, desde donde comenzar. Actualmente, se espera una versi\'on de \lpmd paralela para la versi\'on 0.6 o 0.7 del c\'odigo, sin embargo el nucleo principal de paralelizaci\'on no s encuentra en lpmd, sino que en la API \textbf{liblpmd}, por lo que la evoluci\'on de \'esta es el primer paso en la paralelizaci\'on final del c\'odigo.


%%%%%%%%%%%%%%%%%%%%%%%%%%%%%%%%%%%%%%%%%%%%%%%%%%%%%%%%%%%%%%%%%%%%%%%%%%%%%%%%%%%%%%%%%%%%%%%%%%%%%%%%%
%%%%%%%%%%%%%%%%%%%%%%%%%%%%%%%%%%%%%%%%%%%%%%%%%%%%%%%%%%%%%%%%%%%%%%%%%%%%%%%%%%%%%%%%%%%%%%%%%%%%%%%%%
%CAPITULO 6%%%%%%%%%%%%%%%%%%%%%%%%%%%%%%%%%%%%%%%%%%%%%%%%%%%%%%%%%%%%%%%%%%%%%%%%%%%%%%%%%%%%%%%%%%%%%%
%%%%%%%%%%%%%%%%%%%%%%%%%%%%%%%%%%%%%%%%%%%%%%%%%%%%%%%%%%%%%%%%%%%%%%%%%%%%%%%%%%%%%%%%%%%%%%%%%%%%%%%%%
%%%%%%%%%%%%%%%%%%%%%%%%%%%%%%%%%%%%%%%%%%%%%%%%%%%%%%%%%%%%%%%%%%%%%%%%%%%%%%%%%%%%%%%%%%%%%%%%%%%%%%%%%
\chapter{Gente}
\label{chap:auth}

Desarrollador principal \textbf{Sergio Davis} en colaboraci\'on con \textbf{Joaqu\'in Peralta} y \textbf{Claudia Loyola}, sin embargo el objetivo de lpmd es lograr atraer la atenci\'on de la gente y motivar a que sean colaboradores activos en el desarrollo del proyecto, ya sea desarrollando plugins o bien realizando sugerencias, que con mucho gusto ser\'an acogidas.

Programadores Principales:

\begin{itemize}
 \item Sergio Davis, KTH
 \item Claudia Loyola, UCHILE
 \item Joaqu\'in Peralta, UCHILE
\end{itemize}

Programadores Adicionales:

\begin{itemize}
 \item Nicolas Viaux, UCHILE
 \item Nicolas Perez, UCHILE
 \item NN, UCHILE
 \item 
\end{itemize}

Colaboradores :

\begin{itemize}
 \item Gonz\'alo Guti\'errez
 \item Eduardo Men\'endez
\end{itemize}


\begin{itemize}
 \item ... : ...
\end{itemize}


\end{document}
