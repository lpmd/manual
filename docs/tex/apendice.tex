\appendix
%\chapter{Ap\'endice}

\chapter{M\'odulos}
A continuaci\'on se muestran las listas de los tipos de m\'odulos implementados a la fecha en {\lpmd}. Adem\'as se indica la calidad del m\'odulo, segun la tabla~\ref{tab:modquality}

\begin{table}[h!]
 \begin{tabular}{|c|l|}\hline\hline
 Calidad & Descripci\'on \\\hline\hline
 A & Sin documentaci\'on, sin pruebas, alta posibilidad de fallas. \\
 B & Documentaci\'on incompleta, sin pruebas, alta posibilidad de fallas. \\
 C & Documentaci\'on completa, sin pruebas, baja posibilidad de fallas. \\
 D & Documentaci\'on incompleta, pruebas rigurosas, baja posibilidad de fallas. \\
 E & Documentaci\'on completa, pruebas rigurosas, no ha presentado fallas. \\
 F & Full support. \\\hline\hline
 \end{tabular}
 \label{tab:modquality}
 \caption{Tabla de calidad de implementaci\'on de m\'odulos, s\'olo se presta soporte para los m\'odulos incluidos en el paquete \textbf{lpmd-plugins}}
\end{table}


\section{General}
M\'odulos de {\lpmd} que manejan informaci\'on interna de la API, tales como caracter\'isticas propias de la celda, son en general todos de tipo informativo ya que no llevan dentro algoritmo de an\'alisis complejos.

\begin{table}[h!]
\centering
 \begin{tabular}{|l|c|c|p{10cm}|}\hline
 M\'odulo & Versi\'on & Calidad & Descripci\'on \\
 \hline\hline
 \texttt{energy} & 5.0 & A & M\'odulo que retorna propiedades del sistema tales como energ\'ia(cin\'etica, potencial y total), momentum en cada eje y la temperatura.\\
 \hline
 \texttt{cell} & 5.0 & A & M\'odulo que retorna propiedades del sistema, volumen, largo por eje, densidad y volumen y densdidad por \'atomo. \\
 \hline
 \texttt{pressure} & 5.0 & A & M\'odulo que retorna propiedades del sistema, presi\'on cin\'etica y virial, junto con las componentes del stress. \\
 \hline
 \end{tabular}
\label{tab:modgeneral}
\caption{Tabla con los m\'odulos generales utilizados por lpmd.}
\end{table}

\section{Manejadores de Celda}
Determinan como es la forma de interactuar entre los \'atomos pertenecientes a la celda de simulaci\'on.

\begin{table}[h!]
 \begin{tabular}{|l|c|c|p{10cm}|}\hline
 M\'odulo & Versi\'on & Calidad & Descripci\'on \\
 \hline\hline
 \texttt{minimumimage} & 5.0 & F & M\'odulo que utiliza lpmd, para manejar las listas de interacci\'on, utilizando el m\'etodo de m\'imima imagen.\\
 \hline
 \texttt{linkedcell} & 5.0 & F & M\'odulo que utiliza lpmd, para manejar las listas de interacci\'on, utilizando el m\'etodo de \textit{linked-cell}.\\
 \hline
 \end{tabular}
\label{tab:modmanager}
\caption{Tabla con los m\'odulos que manejan las interacciones at\'omicas en la din\'amica molecular.}
\end{table}

\section{Entrada Salida}
M\'odulos para manejar los ficheros de entrada/salida para las configuraci\'ones at\'omicas que se simulan.

\begin{table}[h!]
 \begin{tabular}{|l|c|c|p{10cm}|}\hline
 M\'odulo & Versi\'on & Calidad & Descripci\'on \\
 \hline\hline
 \texttt{xyz} & 5.0 & F & M\'odulo de lectura/escritura para ficheros \textbf{xyz}, soporta 3 niveles distintos para manejo.\\
 \hline
 \texttt{lpmd} & 5.0 & F & Formato propio de {\lpmd}, soporte lectura/escritura, 3 niveles distintos de manejo.\\
 \hline
 \texttt{zlp} & 5.0 & F & Formato propio de {\lpmd}, utiliza las zlib, y 3 niveles distintos de manejo, la utilizaci\'on es similar a zlp pero con ficheros de menor tama\~no.\\
 \hline
 \texttt{crystalfcc} & 5.0 & F & Generador de celdas FCC. \\
 \hline
 \texttt{crystalbcc} & 5.0 & F & Generador de celdas BCC. \\
 \hline
 \texttt{crystalhcp} & 5.0 & F & Generador de celdas HCP. \\
 \hline
 \texttt{crystalsc} & 5.0 & F & Generador de celdas SC. \\
 \hline
 \texttt{crystal2d} & 5.0 & F & Generador de celdas bidimensionales.\\
 \hline
 \texttt{skewstart} & 5.0 & F & Generador de celdas con m\'etodo skewstart, desarrollado por \textit{K. Refson}, para el programa de cin\'amica molecular \textbf{moldy}.\\
 \hline
 \texttt{dlpoly} & 5.0 & F & Lee Ficheros en formato de posiciones de \textbf{dlpoly}.\\
 \hline
 \texttt{vasp} & 5.0 & F & Lee ficheros \textbf{POSCAR} de vasp, el cu\'al posee las posiciones at\'omcias de la configuraci\'on.\\
 \hline
 \end{tabular}
\label{tab:modinout}
\caption{Tabla con los m\'odulos de entrada y salida utilizados por {\lpmd} y sus utilitarios.}
\end{table}

\section{Modificadores}
Son los m\'odulos que alteran propiedades de la celda, tales como tama\~no, forma, o bien modifican los \'atomos que se encuentran dentro de ella.
\begin{table}[h!]
 \begin{tabular}{|l|c|c|p{10cm}|}\hline
 M\'odulo & Versi\'on & Calidad & Descripci\'on \\
 \hline\hline
 \texttt{tempscaling} & 5.0 & F & Escalamiento de temperatura.\\
 \hline
 \texttt{berendsen} & 5.0 & F & Termostato de berendsen para escalar la temperatura del sistema.\\
 \hline
 \texttt{cellscaling} & 5.0 & F & Escala la celda, modificando los vectores base de la muestra.\\
 \hline
 \texttt{displace} & 5.0 & F & M\'odulo que puede modificar la posici\'on de los \'atomos, desplazandolos.\\
 \hline
 \texttt{rotate} & 5.0 & F & M\'odulo que puede modificar la posici\'on at\'omica, rotando los atomos en torno a un origen y en un cierto \'angulo.\\
 \hline
 \texttt{quenchedmd} & 5.0 & F & M\'odulo que minimiza la muestra a temperatura cero.\\
 \hline
 \end{tabular}
\label{tab:modmodify}
\caption{Tabla con los m\'odulos modificadores del sistema utilizado por {\lpmd}.}
\end{table}

\section{Visualizadores}
Utilizados para obtener imagenes de la simulaci\'on.

\begin{table}[h!]
 \begin{tabular}{|l|c|c|p{10cm}|}\hline
 M\'odulo & Versi\'on & Calidad & Descripci\'on \\
 \hline\hline
 \texttt{povray} & 5.0 & B & Generador de archivos \textbf{pov} para generar im\'agenes de alta resoluci\'on.\\
 \hline
 \end{tabular}
\label{tab:modgvisual}
\caption{Tabla con los m\'odulos visualizadores de lpmd.}
\end{table}


\section{Potenciales}
Son los que determinan como interactuan los \'atomos durante la simulaci\'on.

\begin{table}[h!]
 \begin{tabular}{|l|c|c|p{10cm}|}\hline
 M\'odulo & Versi\'on & Calidad & Descripci\'on \\
 \hline\hline
 \texttt{lennardjones} & 5.0 & F & Interacci\'on at\'omica con potencial de Lennard-Jones.\\
 \hline
 \texttt{fastlj} & 5.0 & F & Interacci\'on at\'omica con potencial de Lennard-Jones Tabulado.\\
 \hline
 \texttt{morse} & 5.0 & F & Interacci\'on at\'omica con potencial de Morse.\\
 \hline
 \texttt{constantforce} & 5.0 & F & Interacci\'on at\'omica con potencial de fuerza constante.\\
 \hline
 \texttt{harmonic} & 5.0 & F & Interacci\'on at\'omica con potencial Arm\'onico.\\
 \hline
 \texttt{buckingham} & 5.0 & F & Interacci\'on at\'omica con potencial de Buckingham.\\
 \hline
 \texttt{suttonchen} & 5.0 & F & Interacci\'on at\'omica con potencial de Sutton y Chen.\\
 \hline
 \texttt{gupta} & 5.0 & F & Interacci\'on at\'omica con potencial de Gupta.\\
 \hline
 \end{tabular}
\label{tab:modpotentials}
\caption{Tabla con los Potenciales interat\'omicos con los que cuenta {\lpmd}.}
\end{table}

\section{Integradores}
Resuelven las ecuaciones de movimiento del sistema.

\begin{table}[h!]
 \begin{tabular}{|l|c|c|p{10cm}|}\hline
 M\'odulo & Versi\'on & Calidad & Descripci\'on \\
 \hline\hline
 \texttt{beeman} & 5.0 & F & Integrador de Beeman.\\
 \hline
 \end{tabular}
\label{tab:modginteg}
\caption{Tabla con los m\'odulos generales utilizados por lpmd.}
\end{table}

\section{Propiedades}
Calculan caracteristicas propias del sistema, tanto como propiedades temporales as\'i como instantaneas.

\begin{table}[h!]
 \begin{tabular}{|l|c|c|p{10cm}|}\hline
 M\'odulo & Versi\'on & Calidad & Descripci\'on \\
 \hline
 \texttt{angdist} & 5.0 & F & Calcula la distribuci\'on angular de la muestra.\\
 \hline
 \end{tabular}
\label{tab:modproper}
\caption{Tabla con los m\'odulos generales utilizados por lpmd.}
\end{table}

\chapter{API - liblpmd}
\label{ap:API}
La \textbf{API} (Ap. Programming Interface) es una herramienta de programaci\'on que puede ser utilizada por cualquier usuario/programador que se vea beneficiado por sus caracter\'isticas.

Consideramos que la mejor forma de comprender el funcionamiento de esta \textbf{API}, es directamente con c\'odigos de ejemplo que pueden escribir los desarrolladores. A continuaci\'on se muestran 3 ejemplos de utilizaci\'on de la \textbf{API}, el primero se enmarca en un ``nano-programa'' de \textbf{DM}, el segundo es la evaluaci\'on de una propiedad est\'atica de una celda del tipo \texttt{.xyz} y la \'ultima una propiedad din\'amica de una celda.

%%%%%%%%%%%%%%%%%%%%%%%%%%%%%%%%%%%%%%%%%%%%%%%%%%%%%%%%%%%%%%%%%
%%%%%%%%%%%%%%%%%%%%%%%%%%%%%%%%%%%%%%%%%%%%%%%%%%%%%%%%%%%%%%%%%
\section{Din\'amica Molecular B\'asica}

A continuaci\'on un c\'odigo que utilza todas las caracter\'isticas de la \textbf{API}, para realizar din\'amica molecular.

\begin{verbatim}
 /*
 * Ejemplo simple de dinamica molecular usando el API de liblpmd
 */

#include <lpmd/api.h>
#include <iostream>

using namespace lpmd;

int main()
{
 MD md;            // define md como un objeto de dinamica molecular
 PluginManager pm; // define pm como un manejador de plugins

 SimulationCell cell(1, 1, 1, true, true, true); // cell es la celda de simulacion
 cell.SetVector(0, Vector(17.1191, 0.0, 0.0));   // define los vectores de la celda
 cell.SetVector(1, Vector(0.0, 17.1191, 0.0));
 cell.SetVector(2, Vector(0.0, 0.0, 17.1191));
 md.SetCell(cell);                    // asigna la celda de simulacion al objeto MD 

 // Carga de plugins con sus parametros
 pm.LoadPlugin("minimumimage", "");
 pm.LoadPlugin("crystalfcc", "symbol Ar nx 3 ny 3 nz 3");
 pm.LoadPlugin("lennardjones", "sigma 3.41 epsilon 0.0138");
 pm.LoadPlugin("velocityverlet", "dt 1.0");
 pm.LoadPlugin("temperature", "t 600.0");
 pm.LoadPlugin("energy", "");

 CellManager & cm = CastModule<CellManager>(pm["minimumimage"]);
 cell.SetCellManager(cm);            // asigna el manejador de celda

 CellGenerator & cg = CastModule<CellGenerator>(pm["crystalfcc"]);
 cg.Generate(cell);

 Potential & pot = CastModule<Potential>(pm["lennardjones"]);
 PotentialArray & potarray = md.GetPotentialArray();
 potarray.Set("Ar", "Ar", pot); // asigna lennardjones al arreglo de potenciales de MD

 Integrator & integ = CastModule<Integrator>(pm["velocityverlet"]);
 md.SetIntegrator(integ);

 InstantProperty & energ = CastModule<InstantProperty>(pm["energy"]);
 
 SystemModifier & therm = CastModule<SystemModifier>(pm["temperature"]);
 therm.Apply(cell);  // aplica el termalizador "temperature" a la celda de simulacion

 // Loop principal de la simulacion, hace 500 pasos
 md.Initialize(); 
 std::cout << "# Pasos   Temperatura" << '\n';
 for (long i=0;i<500;++i)
 {
  md.DoStep();                       // avanza el sistema un paso de simulacion
  energ.Evaluate(cell, pot);         // evalua las propiedades en el plugin energy
  double T;
  T = pm["energy"].GetProperty("temperature"); // pide valor de temp al plugin energy
  std::cout << i << "         " << T << '\n';
 }
 return 0;
}
\end{verbatim}

Para generar el ejecutable,

\control{g++ -o nanodm main.cc -llpmd -ldl -lm}

y listo, tendremos entonces un ejecutable llamado \verb|nanodm| que realizar\'a una simple corrida de din\'amica molecular.

%%%%%%%%%%%%%%%%%%%%%%%%%%%%%%%%%%%%%%%%%%%%%%%%%%%%%%%%%%%%%%%%%
%%%%%%%%%%%%%%%%%%%%%%%%%%%%%%%%%%%%%%%%%%%%%%%%%%%%%%%%%%%%%%%%%
\section{Calculo de Propiedad est\'atica}

Consideremos que tenemos una celda de simulaci\'on y queremos utiliar las ventajas de la \textbf{API} para calcular una propiedad, que sabemos existe en un m\'odulo, por ejemplo \textbf{gdr}. El c\'odigo para el c\'alculo de \textbf{gdr} de la celda nos quea as\'i,

\begin{verbatim}
 /*
 *
 *
 *
 */

#include <lpmd/api.h>

using namespace lpmd;

int main(int argc, char *argv[])
{
 if (argc < 2) 
 {
  std::cerr << "testgdr <file.xyz>" << '\n';
  exit(1);
 }
 PluginManager pm;
 pm.LoadPlugin("xyz", "file="+std::string(argv[1]));
 pm.LoadPlugin("gdr", "rcut 8.0 bins 300 average true");
 pm.LoadPlugin("nullpotential", "");
 pm.LoadPlugin("linkedcell", "nx 7 ny 7 nz 7 cutoff 8.0");

 CellReader & cread = dynamic_cast<CellReader &>(pm["xyz"]);
 InstantProperty & gdr = dynamic_cast<InstantProperty &>(pm["gdr"]); 
 ScalarTable & gdrvalue = dynamic_cast<ScalarTable &>(pm["gdr"]);
 CellManager & cm = dynamic_cast<CellManager &>(pm["linkedcell"]);
 Potential & dummy = dynamic_cast<Potential &>(pm["nullpotential"]);

 pm["gdr"].Show();

 std::vector<SimulationCell> configs;
 cread.ReadMany(std::string(argv[1]), configs);

 Cell cell(13.16, 13.16, 21.39, M_PI/2, M_PI/2, M_PI*120.0/180.0);
 Vector v1 = cell.GetVector(0);
 v1 = Vector(v1.Get(1), v1.Get(0), v1.Get(2));
 Vector v2 = cell.GetVector(1);
 v2 = Vector(v2.Get(1), v2.Get(0), v2.Get(2));
 cell.SetVector(0, v2);
 cell.SetVector(1, v1);
 for (int i=0;i<3;++i) std::cerr << cell.GetVector(i) << std::endl;

 std::cerr << "Read " << configs.size() << " configurations." << '\n';
 std::cerr << "Configuration 0 has " << configs[0].Size() << " atoms\n";

 for (unsigned long i=0;i<configs.size();++i)
 {
  configs[i].SetCell(cell);
  configs[i].SetCellManager(cm);
  gdr.Evaluate(configs[i], dummy);
  gdrvalue.AddToAverage();
 }

 std::cout << gdrvalue << '\n';

 return 0;
}
\end{verbatim}

Esto, lo compilamos de manera similar al caso anterior, obteniendo un ejecutable para calcular una propiedad est\'atica, en este caso \verb|gdr| para la celda de simulaci\'on.
