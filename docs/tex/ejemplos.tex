\chapter{Ejemplos.}
\label{chap:exa}

Ac\'a encontrar\'a algunos ejemplos de simulaciones realizadas con lpmd, en su ultima versi\'on estable.

%%%%%%%%%%%%%%%%%%%%%%%%%%%%%%%%%%%%%%%%%%%%%%%%%%%%%%%%%%%%%%%%%
%%%%%%%%%%%%%%%%%%%%%%%%%%%%%%%%%%%%%%%%%%%%%%%%%%%%%%%%%%%%%%%%%
\section{Ejemplos para LPMD}

Existen algunas l\'ineas dentro de cada ejemplo que al final poseen \verb|//|, lo que significa que la l\'inea no ha finalizado, sino que contin\'ua en la siguiente l\'inea.

\subsection{Celda de Ar de 108 \'atomos.}

A continuaci\'on una simulaci\'on de Ar con 108 \'atomos, en la cu\'al se realizan distintos escalamientos de temperatura. El ejemplo puede descargarlo completamente de,

\cajatx{http://wwww.gnm.cl/software/lpmd/examples/ar108-1.tgz}

Veamos algunas partes fundamentales del fichero de control.

% \texttt{
% \# System file of Ar gas
% \# using lpmd
% 
% \#CELL
% 
% cell crystal a\=17.1191 b\=17.1191 c\=17.1191 alpha\=90.0 beta\=90.0 gamma\=90.0
% 
% }

% \texttt{
% \# System file of Ar gas 
% \# using LPMD
% \#
% 
% \#
% \#CELL and IN/OUT
% \#
% cell crystal a=17.1191 b=17.1191 c=17.1191 alpha=90.0 beta=90.0 gamma=90.0
% 
% input module=lpmd file=argon108.lpmd level=0
% output module=xyz file=output.xyz each=20 level=0
% 
% \#
% \#GENERAL
% \#
% prepare repeat 1 1 1
% prepare temperature 84
% charge Ar 0.0
% steps 5000
% dumping file=ljargon.dump each=10000
% periodic true true true
% 
% \#Cargamos inmediatamente pressure
% \#para poder visualizar con monitor
% 
% use pressure
% enduse
% 
% monitor start=0 end=5000 each=10 properties=kinetic-energy,potential-energy, \/\/
% total-energy,pressure,volume output=monitor.dat
% 
% \#
% \#MODULES DEF
% \#
% use lennardjones as lj\_Ar
%     sigma 3.41
%     epsilon 0.0103408
%     cutoff 8.5
% enduse
% 
% use beeman
%     dt 10.0
% enduse
% \#
% \#MOD APPLICATION
% \#
% potential lj\_Ar Ar Ar
% integrator beeman
% }

Corremos la simulaci\'on con 
\begin{verbatim}
  lpmd ljargon.control > salida.out
\end{verbatim}

Podemos entonces ver algunos resultados de la simulaci\'on, por ejemplo la conservaci\'on de la energ\'ia a partir del fichero \verb|monitor.dat| que fue generado. Como se observa en la figura a continuaci\'on.

\subsection{Escalamiento de Temperatura.}

\subsection{Escalamiento de Celda.}

\subsection{Calculando Propiedades durante la Simulaci\'on.}

\subsection{Multiples corridas con bash.}

\subsection{Generando ficheros pov para crear pel\'iculas.}

\subsection{Cambiando el integrador durante la simulaci\'on.}

\subsection{Multiples Monitor}

\subsection{Multiples Output}

\subsection{Calculo de Modulo de Bulk}

%%%%%%%%%%%%%%%%%%%%%%%%%%%%%%%%%%%%%%%%%%%%%%%%%%%%%%%%%%%%%%%%%
%%%%%%%%%%%%%%%%%%%%%%%%%%%%%%%%%%%%%%%%%%%%%%%%%%%%%%%%%%%%%%%%%
\section{Ejemplos LPMD-ANALYZER}

\subsection{Calculando funci\'on radial de distribuci\'on.}

\subsection{Calculando distribucion angular}

\subsection{Dos tipos de N\'umero de Coordinaci\'on}

\subsection{Distribuci\'on de velocidades}

\subsection{Autocorrelaci\'ond e velocidades}

%%%%%%%%%%%%%%%%%%%%%%%%%%%%%%%%%%%%%%%%%%%%%%%%%%%%%%%%%%%%%%%%%
%%%%%%%%%%%%%%%%%%%%%%%%%%%%%%%%%%%%%%%%%%%%%%%%%%%%%%%%%%%%%%%%%
\section{Ejemplos LPMD-CONVERTER}

\subsection{De un formato a otro}

\subsection{Formatos \'utiles}
