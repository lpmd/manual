\chapter{Ejemplos.}
\label{chap:exa}

Ac\'a encontrar\'a algunos ejemplos de simulaciones realizadas con lpmd, en su ultima versi\'on estable.

%%%%%%%%%%%%%%%%%%%%%%%%%%%%%%%%%%%%%%%%%%%%%%%%%%%%%%%%%%%%%%%%%
%%%%%%%%%%%%%%%%%%%%%%%%%%%%%%%%%%%%%%%%%%%%%%%%%%%%%%%%%%%%%%%%%
\section{Ejemplos para LPMD}

Existen algunas l\'ineas dentro de cada ejemplo que al final poseen \verb|//|, lo que significa que la l\'inea no ha finalizado, sino que contin\'ua en la siguiente l\'inea.

\subsection{Celda de Ar de 108 \'atomos.}

A continuaci\'on una simulaci\'on de Ar con 108 \'atomos, en la cu\'al se realizan distintos escalamientos de temperatura. El ejemplo puede descargarlo completamente de,

\cajatx{http://wwww.gnm.cl/software/lpmd/examples/ar108-1.tgz}

Veamos el fichero de control.

\begin{multicols}{2}
\setlength{\columnseprule}{.5pt}
%\setlength{\columnsep}{20pt}
\begin{verbatim}
# System file of Ar gas 
# using LPMD
#
###################
#CELL and IN/OUT###
###################
cell crystal a=17.1191 b=17.1191 //
     c=17.1191 alpha=90.0 //
     beta=90.0 gamma=90.0

input module=lpmd file=Ar108.lpmd
output module=xyz file=output.xyz //
     each=20 level=0
###################
#GENERAL###########
###################
prepare replicate 1 1 1
prepare temperature 84
charge Ar 0.0
steps 5000
dumping file=ljargon.dump each=10000
periodic true true true

#Cargamos inmediatamente pressure
#para poder visualizar con monitor

use pressure
enduse

monitor start=0 end=5000 each=10 //
  properties=kinetic-energy, //
  potential-energy,total-energy, //
  pressure,volume output=monitor.dat
###################
#MODULES DEF#######
###################
use lennardjones as lj_Ar
    sigma 3.41
    epsilon 0.0103408
    cutoff 8.5
enduse

use beeman
    dt 10.0
enduse

use minimumimage
    cutoff 8.5
enduse
###################
#MOD APPLICATION###
###################
potential lj_Ar Ar Ar
integrator beeman
cellmanager minimumimage
\end{verbatim}
\end{multicols}


Corremos la simulaci\'on con 
\begin{verbatim}
  lpmd ljargon.control > salida.out
\end{verbatim}

\cajafi{ar108-1-energy.eps}{Valores de la Energ\'ia para la simulaci\'on de Argon con 108 \'atomos.}{ar1081energy}

Podemos entonces ver algunos resultados de la simulaci\'on, por ejemplo la conservaci\'on de la energ\'ia a partir del fichero \verb|monitor.dat| que fue generado. Como se observa en la figura~\ref{fig:ar1081energy}. En un equipo moderno, la simulaci\'on no deber\'ia tardar m\'as all\'a de 40 segundos, puede observar los detalls de las cargas de los m\'odulos, asi como tambi\'en toda la informaci\'on de la simulaci\'on realizada en el archivo \verb|salida.out|. Junto con la finalizaci\'on de la simulaci\'on, se han generado los siguientes ficheros :

\begin{tabular}{lcl}\\
 monitor.dat &:& Guarda toda la informaci\'ond e monitoreo solicitada por la orden \verb|monitor|\\
&& del fichero de control. \\
 output.xyz &:& Salida de las posiciones at\'omicas de la celda de simulaci\'on. \\
 restore.dump &:& En caso de corte de luz o falla, sirve para reiniciar una simulaci\'on.\\
\end{tabular}

\subsection{Escalamiento de Temperatura.}

A continuacion veremos un ejemplo de escalamiento de temperatura, utilizando el rescalamiento cl\'asico, consistira en enfriar,de manera directa una muestra de Ar. El ejemplo puede encontrarse en:

\cajatx{http://wwww.gnm.cl/software/lpmd/examples/ar108-scale.tgz}

\subsection{Escalamiento de Celda.}

Modificaremos las caracter\'isticas de la celda durante la simulaci\'on, para ello haremos uso del m\'odulo cellscaling, y luego veremos como se modifica la presi\'on del sistema, junto con la temperatura, el ejemplo esta disponible en

\cajatx{http://wwww.gnm.cl/software/lpmd/examples/ar108-temp.tgz}

\subsection{Calculando Propiedades durante la Simulaci\'on.}

A continuaci\'on, realizaremos un procedimiento simple de din\'amica molecular, pero esta vez, con una celda de Oro, sobre la cual calcularemos propiedades, durante la simulaci\'on. En este caso veremos, la \textit{funci\'on radial de distribuci\'on}, \textit{distribuci\'on angular} y \textit{n\'umero de coordinaci\'on}, un ejemplo de esto se puede encontrar en:

\cajatx{http://wwww.gnm.cl/software/lpmd/examples/au-prop.tgz}

\subsection{Multiples corridas con bash.}

Haremos a continuaci\'on un peque\~no estudio de un gas de Ar a distintas temperaturas, para ello nos respaldaremos de los flags del comando \lpmd para poder modificar variables a partir de un fichero de control, relizaremos un estudio de la \textit{funci\'on de distribuci\'on de pares} para argon bajo distintas temperaturas.

\subsection{Generando ficheros pov para crear pel\'iculas.}

Fue uno de los primeros \textit{approach} a lo que a m\'odulos de visualizaci\'on se refiere, su intenci\'on es generar, a partir de la simulaci\'on, un set de ficheros \verb|pov| para un posterior renderizado y creaci\'on de peliculas, animaciones o simplemente imagnes de alta calidad.

\subsection{Cambiando el integrador durante la simulaci\'on.}

Una caracter\'istica de \lpmd es poder cambiar el integrador, durante la simulaci\'on, esto es utilizado en un ejemplo a continuaci\'on que puede descagar en:

\cajatx{http://wwww.gnm.cl/software/lpmd/examples/int-change.tgz}

\subsection{Multiples Monitor}

La informaci\'on en \lpmd puede ser \textbf{subdividida} seg\'un los requerimientos mismos del usuario a la hora de monitorear caracter\'isiticas propias de la celda de simulaci\'on. En este ejemplo, se guarda la informaci\'on de las energ\'ias, la celda y presiones, en tres ficheros distintos, para un an\'alisis mucho mas simple.

\cajatx{http://wwww.gnm.cl/software/lpmd/examples/multimon.tgz}

\subsection{Multiples Output}

Se puden guardar m\'as de un tipo de formato de dalida, lo que da distintas funcionalidades, sin la necesidad de convertir ntre los formatos, por ejemplo, los \verb|xyz| son mas utilizados para analizar, en cambio los \verb|lpmd| son m\'as portables por su caracter\'isticas de ser posiciones fraccionarias.

\cajatx{http://wwww.gnm.cl/software/lpmd/examples/multiout.tgz}

\subsection{Calculo de Modulo de Bulk}

En este ejemplo, se calcula el m\'odulo de Bulk para Au modificando el tama\~no de la celda de forma hidrostatica durante la simulaci\'on, para as\'i obtener directamente las presiones.

%%%%%%%%%%%%%%%%%%%%%%%%%%%%%%%%%%%%%%%%%%%%%%%%%%%%%%%%%%%%%%%%%
%%%%%%%%%%%%%%%%%%%%%%%%%%%%%%%%%%%%%%%%%%%%%%%%%%%%%%%%%%%%%%%%%
\section{Ejemplos LPMD-ANALYZER}

\subsection{Calculando funci\'on radial de distribuci\'on.}

\subsection{Calculando distribucion angular}

\subsection{Dos tipos de N\'umero de Coordinaci\'on}

\subsection{Distribuci\'on de velocidades}

\subsection{Autocorrelaci\'ond e velocidades}

%%%%%%%%%%%%%%%%%%%%%%%%%%%%%%%%%%%%%%%%%%%%%%%%%%%%%%%%%%%%%%%%%
%%%%%%%%%%%%%%%%%%%%%%%%%%%%%%%%%%%%%%%%%%%%%%%%%%%%%%%%%%%%%%%%%
\section{Ejemplos LPMD-CONVERTER}

\subsection{De un formato a otro}

\subsection{Formatos \'utiles}
