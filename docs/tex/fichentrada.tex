\chapter{El Fichero de Control}
\label{chap:input}

Una de las piezas fundamentales en {\lpmd} para la corrida de una simulaci\'on molecular son los ficheros iniciales de configuraci\'on del sistema. Para el correcto funcionamiento se necesita un fichero de control, este fichero espec\'ifica casi la totalidad de los requerimientos de la simulaci\'on y en ocasiones el 100\%. Aunque en la mayor\'ia de los casos se requiere un fichero adicional en donde se encuentran las posiciones at\'omicas de los \'atomos pertenecientes a la celda de simulaci\'on. Es por eso que en primer lugar, antes de revisar el fichero de control, revisaremos de forma general los archivos de posiciones at\'omicas.

\section{Fichero con Posiciones At\'omicas}

{\lpmd} puede manejar los tipos de fichero de posiciones at\'omicas seg\'un los m\'odulos de los que se disponga; actualmente \textbf{lpmd-plugins} cuenta con varios m\'odulos de formatos de ficheros para especificar las posiciones at\'omicas, entre ellos \textbf{xyz}, \textbf{lpmd}, \textbf{dlpoly} (archivos \verb|CONFIG| o \verb|HISTORY| de dl\_poly), etc.

Esperamos que los usuarios, seg\'un su necesidad, ayuden a implementar o solicitar a los desarrolladores tipos espec\'ificos de sistemas de posiciones at\'omicas para din\'amica molecular. Veamos brevemente en que consisten algunos de ellos. Para informaci\'on espec\'ifica de cada m\'odulo, vea la secci\'on~\ref{chap:modulos:entradasalida}.

%%%%%%%%%%%%%%%%%%%%%%%%%%%%%%%%%%%%%%%%%%%%%%%%%%%%%%%%%%%%%%%%%
%%%%%%%%%%%%%%%%%%%%%%%%%%%%%%%%%%%%%%%%%%%%%%%%%%%%%%%%%%%%%%%%%
\subsection{Fichero .xyz}

Es el est\'andar de ficheros \verb|xyz| utilizado en muchos c\'odigos de simulaci\'on computacional, es un archivo simple que cuenta con la informaci\'on de las posiciones at\'omicas del sistema en cordenadas cartesianas y sus unidades en \AA. La estructura de un fichero es :
\begin{center}
\begin{tabular}{l|l}
 \verb|N| & Especifica el n\'umero de \'atomos en la celda \\
 \verb|comment| & Una l\'inea adicional de comentario, t\'itulo, etc. \\
 \verb|Sym X Y Z| & S\'imbolo at\'omico y las posiciones en coordenadas cartesianas. \\
\end{tabular}
\end{center}

Actualmente el m\'odulo \verb|xyz| soporta tres niveles distintos estos son 0,1 y 2 los que indican la informaci\'on que poseen los \'atomos an cada l\'inea. El nivel cero indica las posiciones at\'omicas, el nivel uno a\~nade las velocidades y el nivel 2 las aceleraciones. Es decir :

\begin{itemize}
\item 0 : \verb|Sym X Y Z| (valor por defecto).
\item 1 : \verb|Sym X Y Z VX VY VZ|.
\item 2 : \verb|Sym X Y Z VX VY VZ AX AY AZ|.
\end{itemize}

Cualquiera de estos niveles puede ser le\'ido como un fichero de entrada para el formato \verb|xyz|.

%%%%%%%%%%%%%%%%%%%%%%%%%%%%%%%%%%%%%%%%%%%%%%%%%%%%%%%%%%%%%%%%%
%%%%%%%%%%%%%%%%%%%%%%%%%%%%%%%%%%%%%%%%%%%%%%%%%%%%%%%%%%%%%%%%%
\subsection{Fichero .lpmd}

A partir de {\lpmd} 0.6.0, existe un nuevo formato 2.0 para los ficheros \verb|.lpmd| (compatible con la versi\'on 1.0). Este es un fichero con las posiciones escaladas de los \'atomos que forman la celda. Es de tipo ASCII y su estructura principal est\'a dada por:

\begin{center}
 \begin{tabular}{l|l}
 \verb|LPMD VERSION X.X | & Especifica la versi\'on del fichero \verb|.lpmd| \\
 \verb|HEAD X Y Z| & Especif\'ica la informaci\'on que posee cada l\'inea at\'omica. \\
 \verb|cell properties | & Propiedades de la celda, pueden ser 3 vectores o longitudes y \'angulos. \\
 \verb|Sym sx sy sz| & S\'imbolo at\'omico y las posiciones escaladas en cada eje [0,1].\\
\end{tabular}
\end{center}

Al igual que el \verb|xyz|, los ficheros de tipos \verb|lpmd| cuentan con niveles (0,1 y 2) junto con flags extras tales como color del \'atomo o alguna propiedad que se puede o no guardar, seg\'un los requerimientos del mismo usuario. Adem\'as a partir de la version 0.6.1 de {\lpmd} el plugin \verb|lpmd| cuenta con soporte de compresi\'on para leer y escribir archivos comprimidos, es decir el antiguo plugin \verb|zlp| ahora est\'a dentro del plugin lpmd.

%%%%%%%%%%%%%%%%%%%%%%%%%%%%%%%%%%%%%%%%%%%%%%%%%%%%%%%%%%%%%%%%%
%%%%%%%%%%%%%%%%%%%%%%%%%%%%%%%%%%%%%%%%%%%%%%%%%%%%%%%%%%%%%%%%%
\subsection{Otros Formatos Soportados}

Pese a que los formatos principales de entrada recomendados son \textbf{xyz} y \textbf{lpmd}, existen actualmente otros formatos que son soportados para lectura/escritura de configuraciones at\'omicas, algunos de ellos son listados en la tabla~\ref{tab:modinout}. Si tiene dudas con respecto a alg\'un formato, as\'i como sugerencias o requerimientos, no dude en contactarnos.

\section{El Fichero de configuraci\'on .control}

Es el fichero principal para realizar la simulaci\'on computacional. Es por eso que en primer lugar se har\'a una descripci\'on general de cada una de sus partes principales y luego veremos \'estas de forma mucho m\'as detallada.

Entre los puntos generales a considerar en un fichero de control, est\'an:

\begin{itemize}
 \item \# Es una l\'inea de comentario. (Evitelas dentro de un bloque \texttt{use ... enduse}).
 \item Pese a que puede ser aleatorio el orden de los flags, se recomienda llevar un orden.
 \item Recomendamos \textbf{definir los m\'odulos} antes de utilizarlos. Tampoco es obligaci\'on.
 \item Existen m\'odulos en un fichero de control que \textbf{no} pueden ser omitidos, como por ejemplo, un \textbf{cellmanager}. Sin embargo la mayor\'ia de estos casos, pueden ser entregados por l\'inea de comando.
\end{itemize}

Con estos puntos previos en mente, veamos entonces cuales son las secciones principales a tener en cuenta dentro de un \textbf{fichero de control} para utilizar con {\lpmd}, consideremos de forma general el esquema que se muestra a continuaci\'on.

\fb{ 
\texttt{
\begin{tabular}{lcl}
 \#Cell Properties & $\rightarrow$ & Propiedades de la celda.\\
 cell ... &&\\
 \#Input/Output & $\rightarrow$ & Principales ficheros de entrada y salida\\
 input ... &&\\
 output ... &&\\
 \#General & $\rightarrow$ & Setting generales de la simulaci\'on\\
 prepare ... &&\\
 steps ... &&\\
 monitor ... &&\\
 \#Filters & $\rightarrow$ & Filtros a aplicar sobre la muestra.\\
 filter ... &&\\
 \#Module Load & $\rightarrow$ & Carga de todos los modulos necesarios.\\
 use ... &&\\
 enduse ... &&\\
 \#Module Apply & $\rightarrow$ & Indica como son aplicados los modulos.\\
 apply ... &&\\
 potential ... &&\\
 integrator ... &&\\
\end{tabular}
}
}

Esta entonces es una \textbf{aproximaci\'on general} a los ficheros de entrada utilizados en din\'amica molecular por {\lpmd}. Con esta idea en mente veremos a continuaci\'on cada una de las secciones principales del fichero de control.

%%%%%%%%%%%%%%%%%%%%%%%%%%%%%%%%%%%%%%%%%%%%%%%%%%%%%%%%%%%%%%%%%
%%%%%%%%%%%%%%%%%%%%%%%%%%%%%%%%%%%%%%%%%%%%%%%%%%%%%%%%%%%%%%%%%
\subsection{Celda de Simulaci\'on.}

Es la propiedad que describe la celda de simulaci\'on. Es decir el detalle completo de cada uno de los ejes que la conforman, los que pueden ser entregados en forma detallada o en forma general. A continuaci\'on se describir\'a la forma en la cu\'al se entrega \'esta propiedad, que \textit{casi siempre} debe estar presente en el fichero .control y va al comienzo de \'este.

\subsubsection{cell}

El flag cell es utilizado para describir la celda de simulaci\'on y asignar las propiedades de \'esta. Generalmente una celda de simulaci\'on puede venir descrita ya en el formato del fichero de entrada (como es el caso de lo ficheros \texttt{lpmd} o \texttt{CONFIG}). Sin embargo, hay formatos, como el \texttt{xyz}, que no poseen la descripci\'on de la celda, es por eso que es necesario en algunos casos utilizar esta opci\'on. Si se desea dar la opci\'on \verb|cell| aunque la informaci\'on se encuentre en un archivo, \'esta predominar\'a sobre la del archivo.

\begin{itemize} 
\item{Forma 1}

Se utilizan la longitud de los lados y \'angulos de la celda, como sigue:

\control{cell a=10 b=5 c=5 alpha=45 beta=90 gamma=90}

donde,

\fb{ 
\begin{tabular}{lcl}
 a & = & indica el largo de la celda en \textbf{a}.\\
 b & = & indica el largo de la celda en \textbf{b}.\\
 c & = & indica el largo de la celda en \textbf{c}.\\
 alpha & = & indica el \'angulo $\alpha$.\\
 beta & = & indica el \'angulo $\beta$.\\
 gamma & = & indica el \'angulo $\gamma$.\\
\end{tabular}
}

\item{Forma 2}

Se utilizan los 3 vectores bases, poni\'endolos de la siguiente manera en el fichero, note que \verb|\| indican la continuaci\'on de una l\'inea \'unica.

\control{cell ax=1.0 ay=0.0 az=0.0 bx=0.0 by=1.0 bz=0.0 \textbackslash \\cx=0.0 cy=0.0 cz=1.0}

donde: a${i}$, b${i}$ y c${i}$ con $i$={$x$,$y$,$z$} son las coordenadas $x, y$ y $z$ de los vectores bases.

Las posibles formas de ingresar una descripcion de la llamada \verb|cell| pueden ser entregadas como argumentos en la ejecuci\'on de {\lpmd} y no necesitan estar dentro del fichero de \textbf{control}, lo que ayuda a la creaci\'on de \textit{scripts}.

\fb{\begin{minipage}[l]{9.5cm}\tt lpmd -L a,b,c -A alpha,beta,gamma archivo.control \\ lpmd -V ax,ay,az,bx,by,bz,cx,cy,cz archivo.control\end{minipage}}


\item{Forma 3}

Existe una forma extra para celdas cubicas que ahorran un poco la escritura completa de cada t\'ermino de la celda, por ejemplo una celda cubica de largo 5\AA, se puede asignar facilmente con :

\control{cell cubic a=5}

\end{itemize}

\subsubsection{Omitiendo cell}
Como se mencion\'o previamente, hay ocaciones en que el archivo de posiciones at\'omicas posee adem\'as la informaci\'on de la celda de simulaci\'on, para estos casos hay dos formas de \textbf{especificar} a {\lpmd} que debe leer la informaci\'on desde ese archivo.

\begin{itemize} 
\item{Forma 1}

Se utilizan opciones especiales dentro del mismo fichero de control :

\control{set replacecell true}

\item{Forma 2}

Se especifica en la ejecuci\'on misma de lpmd con el flag \verb|-r|.

\fb{\begin{minipage}[l]{9.5cm}\tt lpmd archivo.control -r\end{minipage}}

\end{itemize}

%%%%%%%%%%%%%%%%%%%%%%%%%%%%%%%%%%%%%%%%%%%%%%%%%%%%%%%%%%%%%%%%%
%%%%%%%%%%%%%%%%%%%%%%%%%%%%%%%%%%%%%%%%%%%%%%%%%%%%%%%%%%%%%%%%%
\subsection{Entrada - Salida}

\subsubsection{input}

Existen actualmente dos formas de ingreso de un sistema de entrada para la configuraci\'on at\'omica que se requiere simular; estas son, el ingreso de las posiciones at\'omicas de la celda a trav\'es de un archivo (por ejemplo \verb|.xyz| o \verb|.lpmd|) y el otro es mediante m\'odulos que generan autom\'aticamente celdas at\'omicas con ciertas propiedades, por ejemplo celdas \textbf{bcc}, \textbf{fcc}, etc.

Veamos brevemente a continuaci\'on cada uno de ellos,

\begin{itemize}
 \item{Con Fichero}

Para cargar un fichero con configuraciones at\'omicas es necesario la existencia del m\'odulo que reconozca el tipo de fichero, por ejemplo para cargar un fichero del tipo \verb|.xyz|, es necesario utilizar el m\'odulo \verb|xyz| para poder leer sin problemas el archivo, ya que es el m\'odulo el que \textit{entiende} el archivo de ese tipo. 
  \item{Generadores de celda}

A diferencia con el m\'etodo anterior, este m\'etodo no requiere de un fichero con posiciones at\'omicas, en lugar de ello se requiere un m\'odulo que genera automaticamente una celda con \'atomos, seg\'un los requerimientos propios del m\'odulo. Por ejemplo existen m\'odulos actualmente para generar celdas del tipo \textbf{sc}, \textbf{bcc}, \textbf{fcc}, etc. utilizando el plugin \verb|crystal3d|, tambi\'en hay generadores de redes bidimensionales (\verb|crystal2d|) y finalmente se pueden utilizar m\'etodos m\'as sofisticados como \textbf{skewstart}.

\end{itemize}

La forma general de la orden \verb|input| requiere de argumentos para un funcionamiento adecuado. Para ver m\'as informaci\'on sobre el m\'odulo revise la secci\'on ~\ref{chap:modulos:entradasalida}. Estos son ejemplos de algunos argumentos:

\fb{ 
\begin{tabular}{lcl}
 module & = & indica el m\'odulo con el que cargar la celda.\\ 
 file & = & indica el fichero con posiciones at\'omicas.\\
 level & = & indica el nivel del fichero.\\ 
\end{tabular}
}

Estos son los argumentos m\'as standard ya que cada m\'odulo posee sus propios argumentos, por lo que se hace necesario ver cada uno seg\'un el inter\'es.

Veamos algunas formas de uso para la orden \verb|input| :

\begin{itemize}
\item Carga posiciones at\'omicas desde un fichero XYZ.
\control{input module=xyz file=fichero.xyz level=0}
\item Carga posiciones y velocidades desde un fichero XYZ (level 1).
\control{input module=xyz file=fichero.xyz level=1}
\item Inicializa una celda del tipo fcc con átomos de Au.
\control{input module=crystal3d type=fcc nx=3 ny=3 nz=3 symbol=Au}
\item Inicializa una celda del tipo sc con átomos de Na
\control{input module=crystal3d type=sc nx=5 ny=5 nz=5 symbol=Na}
\item Inicializa con metodo skewstart para 108 \'atomos de arg\'on.
\control{input module=skewstart atoms=108 symbol=Ar}
\end{itemize}

La lista de los m\'odulos soportados a la fecha para lectura/generaci\'on de configuraciones, se pueden observar en la tabla~\ref{tab:modinout}.

\subsubsection{output}

Con el par\'ametro \verb|output| se especifican las opciones de salida de las configuraciones at\'omicas de nuestra simulaci\'on, los formatos de salidas son complementamente modulares y pueden ser implementados por los usuarios, sin embargo a partir de la version 0.5.2 del set de plugins \verb|lpmd-plugins| ya se encuentran disponibles muchos m\'odulos, pese a esto es \textit{importante} notar que \textbf{cada m\'odulo posee configuraciones independientes}, por ejemplo \verb|level| es utilizado por m\'odulos como \textbf{xyz}, \textbf{dlpoly} o \textbf{lpmd}, sin embargo no es requerido para \textbf{mol2}, para m\'as informaci\'on refierase a la secci\'on ~\ref{chap:modulos:entradasalida}. Los argumentos generales principales del par\'ametro \verb|output| son:

\fb{
\begin{tabular}{lcl}
 module & = & indica el m\'odulo (formato) de \\
&&salida de la simulaci\'on.\\
 file & = & indica el fichero en el que graba.\\
 level & = & indica el nivel del modulo de salida.\\
 each & = & indica cada cuantos pasos la celda \\
&&es gabada en el fichero.\\
\end{tabular}
}

Al igual que antes, existen m\'as par\'ametros que son independientes de cada m\'odulo.

Algunas formas de uso,

\begin{itemize}
 \item Grabando la simulaci\'on en un fichero XYZ (nivel 0), cada 20 steps.
\control{output module=xyz file=fichero.xyz level=0 each=20}
 \item Grabando la simulaci\'on en fichero LPMD (nivel 1), cada 1 step.
\control{output module=lpmd file=fichero.lpmd level=1 each=1}
 \item Grabando las posiciones at\'omicas en formato lpmd nivel 2 y con colores de los \'atomos.
\control{output module=lpd file=saved.lpmd level=2 each=5 extra=rgb}
\end{itemize}

La lista de los m\'odulos soportados a la fecha para escritura de configuraciones, pude verse en la tabla~\ref{tab:modinout}.

\subsubsection{restore}

Es utilizado para restaurar una simulaci\'on a partir de un punto en que se produjo un corte de energ\'ia el\'ectrica o cualquier otro tipo de falla f\'isica en un centro de c\'alculo. El punto de restauraci\'on es a partir de el \'ultimo dumping realizado por la simulacion, dado por la orden ``dumping'' dentro del fichero de control, indicando el nombre del archivo \verb|dump|. Es recomendable que antes de reiniciar una corrida, respalde los datos en otro directorio, o efectue la \textit{reiniciaci\'on} de la corrida en un directorio distinto, para evitar da\~nar, perder o sobreescribir los datos previos de la simulaci\'on.

Actualmente no se han realizado pruebas exhaustivas de este punto, pero deber\'ia funcionar sin problemas, por favor si encuentra alg\'un bug, reportelo y \textbf{recuerde usar esta opci\'on con precauci\'on}.

%%%%%%%%%%%%%%%%%%%%%%%%%%%%%%%%%%%%%%%%%%%%%%%%%%%%%%%%%%%%%%%%%
%%%%%%%%%%%%%%%%%%%%%%%%%%%%%%%%%%%%%%%%%%%%%%%%%%%%%%%%%%%%%%%%%
\subsection{Propiedades Generales}
\subsubsection{prepare}
Esta opci\'on es utilizada para \textit{setear} valores y caracter\'isticas de la simulaci\'on, que son brindadas a trav\'es de plugins o de la misma API, tenemos por ejemplo :

\begin{itemize}
 \item \textbf{temperature}
Para dar una temperatura inicial al sistema, se prepara la celda con :
\control{prepare temperature t=300}
de esta forma el sistema asigna velocidades iniciales a las part\'iculas para que la temperatura de nuestro sistema corresponda a 300K.
 \item \textbf{replicate}
Para replicar nuestra celda en las distintas direcciones de los vectores bases, la forma de hacerlo para una celda antes de comenzar la simulaci\'on es:
\control{prepare replicate nx=2 ny=2 nz=2}
De esta froma, la celda que se ley\'o en \verb|input| es replicada 2 veces por cada eje, alcanzando 8 veces el n\'umero inicial de part\'iculas. Esta opci\'on es v\'alida s\'olo cuando se desactiva la optimizaci\'on previa utilizando \verb|set|.
\end{itemize}

\subsubsection{set}
Utilizado para setear valores de la simulaci\'on, principalmente para algunas variables globales del sistema. A continuaci\'on los m\'as utilizados :

\begin{itemize}
 \item Desactivando la optimizaci\'on de celda previa a la simulaci\'on.
\control{set optimize-simulation false}
 \item Se asigna que la informaci\'on necesaria para \texttt{cell} est\'a en el fichero de entrada.
\control{set replacecell true}
 \item Seteando la variable \texttt{delay} usada en visualizaci\'on.
\control{set delay 0.1}
\end{itemize}

\subsubsection{charge}
Set de las cargas en \verb|eV| para las especies at\'omicas. Estos valores de las cargas, son seteados principalmente para utilizaci\'on de potenciales interat\'omicos en los cuales se utilizan las cargas de los atomos involucrados.


Forma de uso

\begin{itemize}
 \item Seteando las cargas de los atomos de O y Ge.
\control{charge O XX \\ charge Ge XX}
\end{itemize}

\subsubsection{mass}
Seting de la masa en \verb|a.u.| para las especies at\'omicas. Estos valores, son seteados principalmente para utilizaci\'on de potenciales interat\'omicos en los cuales se desea modificar la masa de los \'atomos involucrados.


Forma de uso

\begin{itemize}
 \item Seteando las cargas de los atomos de O y Ge.
\control{mass O XX \\ mass Ge XX}
\end{itemize}


\subsubsection{periodic}
Indica la periodicidad de la celda, en cada eje. Al bloquear la periodicidad en un eje, este se ve ``modificado'' en ambos lados de la celda, revise con cuidado estas opciones.

\control{periodic false false true}

En \'este caso s\'olo tenemos periodicidad en el eje \verb|z|. En las versiones posteriores a 0.6.0 no se han hecho pruebas rigurosas de las opciones de periodicidad as\'i que \textbf{\'usela con precacuci\'on}.

\subsubsection{steps}
N\'umero de pasos de la simulaci\'on de DM.

\control{steps 10000}

Ac\'a se indica que la simulaci\'on se realizar\'a con 10000 pasos.

\subsubsection{dumping}
Genera una salida global del sistema para poder restaurar a partir de ese punto.

\control{dumping file=rescue.dump each=10000}

Generamos un fichero de volcado cada 10000 pasos de la simulaci\'on, en \'el se graba toda la informaci\'on necesaria, para reiniciar una corrida.

\subsubsection{monitor}
Cada cuantos pasos la simulaci\'on muestra las propiedades globales. Estas propiedades, pueden ser asignadas por el mismo usuario, haciendo la salida lo m\'as configurables seg\'un los propios requerimientos.

Entre las opciones de monitor, cuentan :

\fb{
\begin{tabular}{lcl}
 step & = & muestra el \texttt{step} actual de la simulaci\'on\\
 start & = & indica el valor de epsilon.\\
 end & = & indica el valor de sigma.\\
 each & = & indica el cutoff del potencial.\\
 properties & = & indica que valores se desea monitorear. \\
 output & = & archivo de salida para guardar los valores, \\
 & & si no, el \textit{standard output} es utilziado.\\
\end{tabular}
}

Si queremos ir chequeando, los valores de la energ\'ia durante la simulaci\'on, cada 10 pasos, utilizamos la l\'inea, (note que \verb|\\| indica que la l\'inea contin\'ua)

\begin{verbatim}
monitor start=0 end=1000 each=10 properties=step,kinetik-energy,\\
        potential-energy,total-energy output=salida.out
\end{verbatim}

en \'este caso, no es necesario cargar los modulos \verb|energy| y \verb|cell| ya que son cargados por \textit{default}, sin embargo para cuando necesitemos ver la presi\'on durante la simulaci\'on, es necesario ingresar en el fichero de control el uso del m\'odulo \verb|pressure| ya que \'el es el encargado de mostrar y calcular la presi\'on.

\begin{verbatim}
use pressure
enduse
monitor start=0 end=1000 each=10 properties=kinetik-energy, \\
        potential-energy,total-energy,total-pressure output=salida.out
\end{verbatim}

Algunas de las opciones soportadas por \textbf{properties} son:

\begin{itemize}
 \item kinetic-energy : Muestra la energ\'ia cin\'etica.
 \item potential-energy : Muestra la energ\'ia potencial.
 \item total-energy : Muestra la energ\'ia total.
 \item temperature : Muestra la temperatura del sistema.
 \item virial-pressure : Aporte del t\'ermino del virial a la presi\'on.
 \item kinetic-pressure : T\'ermino de la presi\'on asociado.
 \item pressure : Presi\'on total del sistema.
 \item volume : Volumen de la celda de simulaci\'on.
 \item cell-x : x=a,b,c son los largos de la celda en cada eje.
 \item sij : i,j=x,y,z entrega los valores para el tensor de stress.
\end{itemize}

Una ventaja es la utilizaci\'on de multiples \verb|monitor| para as\'i ir almacenando la informaci\'on \textit{a medida}. Para revisar todas las opciones de salida, refirase a la secci\'on~\ref{chap:modulos}.

\subsection{Filtros}

Aparecen en versiones posteriores a 0.6.0, son m\'odulos cuya caracter\'istica principal es \textit{filtrar} los \'atomos de una simulaci\'on acorde a ciertos tipos de requerimientos, por ejemplo :

\vspace{1cm}
\begin{center}
\begin{tabular}{lcl}
index &:& Filtrado de atomos seg\'un sus \'indices.\\
box &:& Filtrado de \'atomos pernetecientes a paralelep\'ipedo.\\
sphere &:& Filtrado de \'atomos seg\'un esfera.\\  
element &:& Filtrado de \'atomos seg\'un simbolo at\'omico.\\
\end{tabular}
\end{center}
\vspace{1cm}

La ventaja de estos filtros es poder generar nuevas configuraciones a partir de resultados previos, as\'i como tambi\'en an\'alisis mucho m\'as detallados para los \textit{\'atomos filtrados}.

\subsection{Carga de M\'odulos}

Ac\'a mostraremos c\'omo se utilizan en general la carga de m\'odulos dentro de un fichero de control. Los m\'odulos o plugins pueden ser cargados en cualquier secci\'on del fichero de control, sin embargo recomendamos hacerlo de forma ordenada como veremos en los ejemplos posteriores.

\subsubsection{C\'omo cargar un m\'odulo}

Los m\'odulos son de distintos tipos en general, y los de un tipo en com\'un comparten ``ciertas'' caracter\'isticas ya que cada uno posee sus propias ventajas. Una visi\'on general de c\'omo se han distribuido los m\'odulos es: 

\begin{tabular}{lcl}
 Generadores de celda & : & Tales como \verb|xyz|, \verb|pdb|, \verb|crystalfcc|, etc. \\
 Manejadores de celda & : &\verb|minimumimage|, \verb|linkedcell|, etc. \\
 Modificadores de celda & : &\verb|cellscale|, \verb|tempscalling|, etc.\\
 Filtros & : &\verb|sphere|, \verb|box|, etc.\\
 Calculadores de Propiedades & : & \verb|gdr|, \verb|angdist|, \verb|msd|, etc. \\
 Integradores & : & \verb|verlet|, \verb|euler|, etc. \\
 Potenciales & : & \verb|LennardJones|, \verb|SuttonChen|, \verb|Morse|, etc. \\
 Informativos & : & \verb|cell|, \verb|pressure|, \verb|Energy|, etc.
\end{tabular}

En general, siempre a un m\'odulo se le puede asignar un \textbf{alias} para su posterior llamado, por ejemplo.

\control{use MODULO as ALIAS \\ ... \\enduse}

De esta forma el m\'odulo \texttt{MODULO} puede ser llamado en forma posterior con el nombre \texttt{ALIAS}, lo que da una ventaja para reutilizar, combinar y simplificar la utilizaci\'on de un m\'odulo en m\'as ocaciones.

Entonces dentro de un archivo de \textbf{control}, debemos cargar los m\'odulos necesarios para un posterior llamado.

\subsection{Aplicaci\'on de m\'odulos}

Los m\'odulos son llamados en la parte final de nuestro fichero de \textbf{control}, como existen distintas ``especies'' de m\'odulos, estos deben ser llamados de diferentes maneras, aunque su forma es muy general.

\begin{itemize}
 \item \textbf{Modificadores}
  M\'odulos llamados para modificar alguna propiedad de la celda de simulaci\'on. Se aplican con:
  \control{apply module-alias start=0 end=1000 each=20}
 \item \textbf{Filtros}
  M\'odulos que modifican la celda de simulaci\'on filtrando \'atomos seg\'un ciertas condiciones, pueden ser
  utilizados tanto con la instrucci\'on \verb|over| de \textbf{apply} como con \textbf{filter}.
  \control{apply <apply-options> over <filter-options>}
  \control{filter module-alias <options>}
 \item \textbf{Calculadores de Propiedades}
  Estos siempre son llamados a ser evaluados cada cierto tiempo entre un rango de intervalos. A partir de la versi\'on 0.6.0 de {\lpmd} se pueden adem\'as calcular propiedades sobre un \textit{set} de \'atomos filtrados con \verb|over|.
  \control{property module-alias start=0 end=1000 each=10}
  \control{property gdr start=0 end=-1 each=5 over <filter-options>}
 \item \textbf{Integradores}
  Estos son llamados facilmente con:
  \control{integrator module-alias}
 \item \textbf{Potenciales}
  Estos definen una interaccion entre dos \'atomos (por el momento no hay de tres cuerpos). Se crea un potencial para cada interacci\'on especificando los valores y llamando luego con
  \control{potential module-alias Pt Au}
 \item \textbf{Manejadores de celda}
  Llaman al manejador de celda que se utilizara en la simulaci\'on, en nuestro caso hay dsiponibilidad de dos manejadores, \verb|linkdecell| y \verb|minimumimage|.
  \control{cellmanager linkedcell}
\end{itemize}

\section{Ficheros de Salida}

%%%%%%%%%%%%%%%%%%%%%%%%%%%%%%%%%%%%%%%%%%%%%%%%%%%%%%%%%%%%%%%%%
%%%%%%%%%%%%%%%%%%%%%%%%%%%%%%%%%%%%%%%%%%%%%%%%%%%%%%%%%%%%%%%%%
\subsection{Ficheros de salida}
{\lpmd} Tiene variados tipos de ficheros de salida, uno que es generado usualmente por la opci\'on \verb|output| dentro del fichero de control, otro es la salida standard, que por defecto va a \verb|cout|, pero que nosotros recomendamos enviar (o redireccionar) a un fichero adem\'as estan los que generan cada uno de los m\'odulos en forma independiente cuando estos requieren de aquello.

%%%%%%%%%%%%%%%%%%%%%%%%%%%%%%%%%%%%%%%%%%%%%%%%%%%%%%%%%%%%%%%%%
%%%%%%%%%%%%%%%%%%%%%%%%%%%%%%%%%%%%%%%%%%%%%%%%%%%%%%%%%%%%%%%%%
\subsubsection{Salida standard}
Esta es la salida que muestra en pantalla lpmd, la forma de enviar esta salida a un fichero, durante la ejecuci\'on de lpmd, es

\control{lpmd fichero.control > salida.out}

o si desea verla en pantalla y enviar a un fichero:

\control{lpmd fichero.control | tee salida.out}

tambi\'en puede redireccionar una salida standard y los mensajes de error de forma independiente utilizando:

\control{lpmd fichero.control 1\&> salida.out 2\&> salida.err}

el fichero \verb|salida.out| tiene toda la informaci\'on que deb\'ia salir a pantalla utilizando lpmd, entre ella se encuentran,

\begin{itemize}
 \item Descripci\'on completa de la celda
 \item Informacion de \verb|startinfo|
 \item Informaci\'on de m\'odulos utilizados y variables de cada uno de ellos.
 \item Energ\'ias, Temperatura, Presi\'on y Vol\'umen, seg\'un \verb|monitor|. En caso de que monitor no utilize un valor propio para el valor de la variable \verb|output|.
 \item \textit{Debugger Information} : Informaci\'on sobre aplicaciones y otras cosas que se realizan durante la simulaci\'on usualmente dirigida a \verb|std::cer|.
\end{itemize}


%%%%%%%%%%%%%%%%%%%%%%%%%%%%%%%%%%%%%%%%%%%%%%%%%%%%%%%%%%%%%%%%%
%%%%%%%%%%%%%%%%%%%%%%%%%%%%%%%%%%%%%%%%%%%%%%%%%%%%%%%%%%%%%%%%%
\subsubsection{Fichero generado por output}
Este fichero se gener\'o con el nombre entregado en el fichero de control a la linea \textbf{output}, en \'el se encuentran las configuraciones atomicas de la \textbf{DM} y suelen ser los ficheros a partir de los cuales suelen crearse animaciones y an\'alisis detallados de la simulaci\'on.

Todas las propiedades instantaneas pueden calcularse \textit{\textbf{durante la simulaci\'on misma}}, sin embargo {\lpmd} tambi\'en cuenta con herramientas propias de an\'alisis como se puede ver en la secci\'on~\ref{chap:utilidades}.

Es importante definir un buen formato de salida ya que es la clave para simplificar o dificultar los an\'alisis posteriores. Por ejemplo si se adquiere un archivo \verb|xyz| con \verb|level=0| no es el ideal para calcular por ejemplo la \textit{Velocity autocorrelation Function} (\verb|vacf|) ya que en caso de que el nivel sea menor a uno algunos plugins determinar\'an la velocidad utilizando configuraciones at\'omicas sucesivas de forma autom\'atica lo que reducira un poco el tiempo de an\'alisis, que pudo aver sido aprovechado durante la simulaci\'on grabando con \verb|level=1|.

\subsubsection{Ficheros generados por m\'odulos}
Muchos m\'odulos de lpmd que realizan an\'alisis generan ficheros de informaci\'on de manera independiente, además los plugins en general manejan dos \textit{est\'andares} de salidas asignados con el flag \verb|average|. Por ejemplo si deseamos calcular la \textit{Funci\'on de distribuci\'on de pares} sobre nuestras configuraciones, podemos calcular \'esta sobre cada una de las muestras e imprimirlas o bien generar un promedio sobre todas las configuraciones.

\begin{itemize}
\item \verb|average true|
Especif\'ca que luego de realizar los correspondientes an\'alisis sobre las configuraciones especificadas del fichero, estas deben promediarse antes de imprimir en el archivo de salida.
\item \verb|average false|
Especif\'ica que cada vez que realiza un an\'alisis sobre las configuraciones pertenecientes al archivo, estas deben ser impresas una a una en el archivo de salida. 
\end{itemize}
