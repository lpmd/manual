\chapter{The Control File} % Translation by F. Gonzalez
\label{chap:input}

One of the fundamental pieces in {\lpmd} to run a molecular dynamics simulation
is the \textbf{control file} of the system. This file specifies all the
requirements that the simulation needs to be executed, including the name and
type of file where the positions and velocities of the atoms are, if these are
needed. In fact usually we are going to work with two kind of files :

\begin{itemize}
 \item input-file : Atomic positions and cell details. This file have
principally the atomic structure of our simulation.
 \item control-file : Specify the simulation procedure details, some times the
atomic structures are included too in this file.
\end{itemize}


Now, before we continue, we are going to give a short description of the
files that have the information about the atomic positions and cell
specifications: the input files. And the different types that {\lpmd} can
handle.

\section{The file with the atomic positions, the input-file}

{\lpmd} has many modules (see chapter~\ref{chap:modulos:entradasalida}), and the
 input/output files are one of them. The modules are plugins of the program
that, in this case, manage the reading and writing of atomic positions and/or
velocities and/or accelerations. Some available formats (modules) for doing this
are availables in {\lpmd} \textbf{xyz}, \textbf{lpmd}, \textbf{dlpoly}
(\verb|CONFIG| or \verb|HISTORY| files types from dl\_poly), etc.

We expect that the users, depending on their needs, help us to implement (or
ask the developers to do it) new input/output modules. Now let's see breafly
what they consist of.

%%%%%%%%%%%%%%%%%%%%%%%%%%%%%%%%%%%%%%%%%%%%%%%%%%%%%%%%%%%%%%%%%
%%%%%%%%%%%%%%%%%%%%%%%%%%%%%%%%%%%%%%%%%%%%%%%%%%%%%%%%%%%%%%%%%
\subsection{The file type .xyz}
\label{subsec:xyz}\index{xyz}

This is one of the more standard format used in a lot of different molecular
dynamics codes. It is an ASCII file that contains the position (in cartesian
coordinates) and atomic symbol of each atom of the simulation. The structure of
the file is basically:

\begin{center}
\begin{tabular}{l|l}
 \verb|N| & Specifies the number of atoms in the simulation cell.\\
 \verb|comment| & A line for comments, title, etc. This line is usually left
in blank.\\
 \verb|Sym X Y Z| & Atomic symbol, X coordinate, Y coordinate and Z coordinate.
  \\
\end{tabular}
\end{center}

Currently, the module \verb|xyz| has three different levels (0,1,2) which 
indicate the amount of information for each atom. The 0 (zero) level indicates
that the file contains (or will contain) the symbol and the 3-dimensional
position of each atom only in \AA units. The level 1 indicates that the file
contains not only the symbol and position, but the velocity (7 columns
file) in \AA/fs units. The level 2 is used when the user also needs the
aceleration of each atom (3 aditional columns) in \AA/fs$^2$ units.

\begin{itemize}
\item 0 : \verb|Sym X Y Z| (default value).
\item 1 : \verb|Sym X Y Z VX VY VZ|.
\item 2 : \verb|Sym X Y Z VX VY VZ AX AY AZ|.
\end{itemize}

%%%%%%%%%%%%%%%%%%%%%%%%%%%%%%%%%%%%%%%%%%%%%%%%%%%%%%%%%%%%%%%%%
%%%%%%%%%%%%%%%%%%%%%%%%%%%%%%%%%%%%%%%%%%%%%%%%%%%%%%%%%%%%%%%%%
\subsection{The file type .lpmd and .zlp}

The \verb|.lpmd| and \verb|.zlp| file format are a special type of format in
{\lpmd}. This files are plane ASCII format (lpmd) and a compress format (zlp).
The principal structure for this kind of files is given by :

\begin{center}
 \begin{tabular}{l|l}
 \verb|LPMD X.X C| & Header line with information about the version of the file
(X.X) and a special char (C) that indicate if is a compress file or not. \\
 \verb|HDR SYM X Y Z| & Show the information that every atomic line will have.\\
 \verb|cell properties | & A line with cell structure information. \\
 \verb|Sym sx sy sz| & Atomic information, depends of the HDR line.\\
\end{tabular}
\end{center}

In a similar way that for the case of \verb|xyz| file types, the \verb|lpmd|
file types have different levels (0, 1 and 2), but on the other hand they have
available different additional information about the atoms, information like
the atom color, or atom tags.

%%%%%%%%%%%%%%%%%%%%%%%%%%%%%%%%%%%%%%%%%%%%%%%%%%%%%%%%%%%%%%%%%
%%%%%%%%%%%%%%%%%%%%%%%%%%%%%%%%%%%%%%%%%%%%%%%%%%%%%%%%%%%%%%%%%
\subsection{Other supported formats}

The two principall supported formats are \verb|xyz| and \verb|lpmd| (compress
and not compress variants). We have implemented another input plugins in order
to read or write atomic configurations from the file. All the availables to the
date are listed in the tables~\ref{tab:modinout} and~\ref{tab:cellgen}. If you
have questions or suggestions respect to the supported formats, please do not
hesitate to contact us.

We do not give a detailed information about this plugins here, because is not
deeply necessary.

\section{The control-file}


This is the principall file in order to realize a computational simulation. For
this reason we will give you a description of each part of this file and then
we will explain each one separately and deeply.

First of all, a general considerations about the control files:

\begin{itemize}
 \item \# Is a comment line. (Avoid this lines in a \texttt{use ... enduse}
section).
 \item The \verb|/| symbol indicate that the linea continue in the next line.
 \item Altought they may be random, we suggest that you have a order with
the plugins load.
 \item You have to load a plugin, before use it.
 \item Some plugins are mandatory, for example \textbf{cellmanager}.
However many of them can be applied in command line execution.
\end{itemize}

With all this point in mind, we will see the principal sections to consider in
a control file in order to use {\lpmd}. A general scheme is show next.

\fb{ 
\texttt{
\begin{tabular}{lcl}
 \#Cell Properties & $\rightarrow$ & Properties about the CELL.\\
 cell ... &&\\
 \#Input/Output & $\rightarrow$ & Input/Output atomic files\\
 input ... &&\\
 output ... &&\\
 \#General & $\rightarrow$ & General Simulation settings\\
 prepare ... &&\\
 steps ... &&\\
 monitor ... &&\\
 \#Filters & $\rightarrow$ & Filters in the atomic configuration.\\
 filter ... &&\\
 \#Module Load & $\rightarrow$ & Load modules.\\
 use ... &&\\
 enduse ... &&\\
 \#Module Apply & $\rightarrow$ & Applying modules.\\
 apply ... &&\\
 potential ... &&\\
 integrator ... &&\\
\end{tabular}
}
}

This is a general approximation to the control files used in molecular dynamics
by {\lpmd}. With this idea in mind we will analyse each of these section
separately and deeply.

%%%%%%%%%%%%%%%%%%%%%%%%%%%%%%%%%%%%%%%%%%%%%%%%%%%%%%%%%%%%%%%%%
%%%%%%%%%%%%%%%%%%%%%%%%%%%%%%%%%%%%%%%%%%%%%%%%%%%%%%%%%%%%%%%%%
\subsection{Simulation Cell.}

This is the property that describe the simulation cell. This mean the full
detail of each axis that describe the cell, this information can be given in
detailes and in general way. Next we will describe the way in that this
property is set in the control file, this is the first description in the
control file.


\subsubsection{cell}

The \verb|cell| option is used to describe the simulationcell and set the
properties of these. Usually a simulation cell can be described previously in a
input-file together all the atomic positions (like the case of the
\texttt{lpmd} or \texttt{CONFIG} files). However, some formats like the
\texttt{xyz}, that not have description of the cell and for this reason is
necessary set this information in the control file. If you give the information
in the control-file and the inormation is available too in the input-file, so
the information in the control-file will be considered.

\fb{All the longs and settings of this variable have to be given in \AA units.}

To set the description of the simulationcell in teh control file, you have
three different styles, all of these will be dscribed bellow.

\begin{itemize} 
\item{Style 1}

You will put the long of all the axis and the cell angles, like this:

\control{cell crystal a=10 b=5 c=5 alpha=45 beta=90 gamma=90}
\noindent
where,

\fb{ 
\begin{tabular}{lcl}
 a & = & Is the long of the \textbf{a} vector.\\
 b & = & Is the long of the \textbf{b} vector.\\
 c & = & Is the long of the \textbf{c} vector.\\
 alpha & = & Set the $\alpha$ angle.\\
 beta & = & Set the $\beta$ angle.\\
 gamma & = & Set the $\gamma$ angle.\\
\end{tabular}
}

\item{Style 2}

This is the more precisely way to set the simulation cell, in this case you
will put the 3 vectors that generate the simulation cell. Remember that the
symbol / indicate that the line continue.

\control{cell vector ax=1.0 ay=0.0 az=0.0 bx=0.0 by=1.0 bz=0.0 / \\cx=0.0 cy=0.0
cz=1.0}

\noindent
where: a${i}$, b${i}$ y c${i}$ with $i$={$x$,$y$,$z$} are the coordinates $x, y$
and $z$ of the cell vectors.

For the two previous ways to set the simulation cell information in the control
file, you can apply theses directly in the execution, this mean that is not
necessary put this explicitely in the control file.

\fb{\begin{minipage}[l]{9.5cm}\tt lpmd -L a,b,c -A alpha,beta,gamma
archivo.control \\ lpmd -V ax,ay,az,bx,by,bz,cx,cy,cz
archivo.control\end{minipage}}

\item{Style 3}

This is an additional style to set cubic cells in a easy way. For example for a
cubic cell of 5\AA, you can set this easily with :

\control{cell cubic a=5.0}

\end{itemize}

\subsubsection{Omitting cell}

Like we saw avobe, sometimes the input-file with atomic positions have the
information about the cell, for this cases, we have two options to
\textbf{specify} to the program that must read these information from the file.

\begin{itemize} 
\item{Option 1}

We add in the control file the information that the cell is in the input-file
using the line:

\control{set replacecell true}

\item{Option 2}

We specify the information during the execution of the program using the flag
\verb|-r|.

\fb{\begin{minipage}[l]{9.5cm}\tt lpmd archivo.control -r\end{minipage}}

\end{itemize}

%%%%%%%%%%%%%%%%%%%%%%%%%%%%%%%%%%%%%%%%%%%%%%%%%%%%%%%%%%%%%%%%%
%%%%%%%%%%%%%%%%%%%%%%%%%%%%%%%%%%%%%%%%%%%%%%%%%%%%%%%%%%%%%%%%%
\subsection{Input/Output Section}

\subsubsection{Input}
\index{input file}

Nowadays we have two differnet ways to set the atomic configurations of the
simulation, there are:a) The input of the atomic position by means of a input
file (ex: \verb|.xyz|, \verb|.lpmd|, etc) and b) using specialized plugins that
generate automatically atomic cells with specific properties, like
\textbf{bcc}, \textbf{hcp}, etc.

Now a brief description of each method is showed below.

\begin{itemize}
\item{Using a input-file}

In order to load a file with atomic configurations, you will need a plugin that
can read this filetype, for example in order to read a \verb|.xyz| file you
will need to use the \verb|.xyz| plugin. A wide variety of supported types of
files are availables in the section~\ref{...}.
  
\item{Generadores de celda}
\index{cell generators}

In a different way that the previous method, this methodology do not requeried
a input-file, in place of this, we will need a plugin that can generate
automatically a cell with atoms. For this case {\lpmd} have a wide range of
possibilities, generating \textbf{bcc}, \textbf{fcc}, etc. using the
\verb|crystal3d| plugin for example.
\end{itemize}

The \verb|input| option require some specific arguments for a correct read of
the input-file. The requested information depend onf the module that you will
use, for the specific information about each module, take a look to the
section~\ref{chap:modulos:entradasalida}. 

The more standard arguments required in the input section are :

\fb{ 
\begin{tabular}{lcl}
 module & = & Indicate the plugin that will load the cell.\\ 
 file & = & Indicate the name of the file.\\
 level & = & Indicate the level of the file.\\ 
\end{tabular}
}

Take a look of some examples of the \verb|input| option :

\begin{itemize}
\item Load atomic positions from a XYZ file.
\control{input module=xyz file=fichero.xyz level=0}
\item Load atomic positions and velocities from a XYZ file.
\control{input module=xyz file=fichero.xyz level=1}
\item Generate a FCC cell with Au atoms.
\control{input module=crystal3d type=fcc nx=3 ny=3 nz=3\textbackslash\\ symbol=Au}
\item Generate a SC cell with Na atoms.
\control{input module=crystal3d type=sc nx=5 ny=5 nz=5\textbackslash\\ symbol=Na}
\item Generate a Argon cell with 108 using the skewstart method.
\control{input module=skewstart atoms=108 symbol=Ar}
\end{itemize}

The full list of supported plugins for reading and generation of atomic
structures can be found in the tables~\ref{tab:modinout} and~\ref{tab:cellgen}.

\subsubsection{Output}

With the option \verb|output| you will set the output options of the atomic
configurations during the simulation process. In similarity with the
\verb|input| option the additional arguments of the \verb|output| option
depends completely of the plugin that you choice for write the atomic
configurations. For example \verb|level| is an argument that can be used in
plugins like \textbf{xyz}, \textbf{dlpoly} or \textbf{lpmd}, however this can
not be used in \textbf{mol2} plugins, for a deeper information about this take
a look to the section~\ref{chap:modulos:entradasalida}. 

The more general arguments of the \verb|output| option are :

\fb{
\begin{tabular}{lcl}
 module & = & Indicate the output plugin for the simulation.\\
 file & = & Indicate the name of the output file.\\
 level & = & Indicate the output level of the file.\\
 each & = & Indicate each many steps the atomic positions \\
&&are stored in the file.\\
 start & = & Indicate when the configuration start to be saved.\\
 end & =& Indicate when finish to save to the file.\\
\end{tabular}
}

Some examples using the \verb|output| option are :

\begin{itemize}
 \item Save the simulation in a XYZ file each 20 steps.
\control{output module=xyz file=fichero.xyz level=0 each=20}
 \item Save the simulation in a level 1 XYZ file each 1 step.
\control{output module=lpmd file=fichero.lpmd level=1 each=1}
 \item Save the simulation in a level 2 lpmd file with atom colors.
\control{output module=lpd file=saved.lpmd level=2 each=5\textbackslash\\ extra=rgb}
\end{itemize}

The full-list of plugins used to write atomic simulations can be found in the
table~\ref{tab:modinout}.

\subsubsection{dumping/restore}
\index{dumping}\index{restore}

This option is used to restore a simulation from a break pint caused by
electrical cutoff, computer explotion or similar problems. The restoring point
is determined for the last dumping realized by the simulation, given by the
\verb|dumping| option.

In order to make a simulation using the \verb|dumping| option you will have to
specify the dumping filename and the frequency of the write procuder over this
file. For this you will only need an additional line like this :

\control{dumping file=restore.dump each=50000}
\noindent
with this you will dump all the simulation information each 50K steps in the
file \verb|restore.dump|.

If you want to restore a simulation process from this dumping file, at first we
suggest that you copy all the previous information to a new folder, principally
to avoid lose, damage or overwrite important data. On the other hand has been
not deeply proved, so use this tool carefuly.

Now if you want to start a new simulation from this restoring point you will
have to add a new line to the control file.

\control{...\\ dumping file=restauracion-2.dat each=50000 \\ restore
file=restauracion.dat \\...}

In this way the simulation restart from the last saved simulation in the file
\verb|restore.dump|.
%%%%%%%%%%%%%%%%%%%%%%%%%%%%%%%%%%%%%%%%%%%%%%%%%%%%%%%%%%%%%%%%%
%%%%%%%%%%%%%%%%%%%%%%%%%%%%%%%%%%%%%%%%%%%%%%%%%%%%%%%%%%%%%%%%%
\subsection{General properties}

In this section we will detail some specific options available in a control
file. This option are highly related to the preparation and tune-up of the
simulation.

\subsubsection{prepare}

This option is used to set values and characteristics of the simulation, that
can be given by specific plugins or internally from the API, some examples are :

\begin{itemize}
 \item \textbf{temperature}
In order to set a initial temperature to the system we will use :
\control{prepare temperature t=300}
In this way the system use the \verb|temperature| plugin in order to set a
initial velocity to the atoms of the simulation corresponding to 300K.
 \item \textbf{replicate}
We can replicate our simulation cell in different directions of the cell
vectors, the way to do this is :
\control{prepare replicate nx=2 ny=2 nz=2}
For this case we used the \verb|replicate| plugin in order to increase the cell
2 times for side, this mean 8 times the original number of atoms. This option
can be used only if you deactivate the optimization flag using the
option \verb|set optimize-simulation false|.
\end{itemize}

\subsubsection{set}

This option is used to assign specific values to the simulation, particular for
some global variables of the system, the more used are :

\begin{itemize}
 \item Disable optimization previous to the simulation.
\control{set optimize-simulation false}
 \item Indicate that the information about the cell is specified in the
input-file with atomic coordinates.
\control{set replacecell true}
 \item Set the \texttt{delay} variable used in lpvisual plugin.
\control{set delay 0.1}
\end{itemize}

\subsubsection{charge}

Set the atomic charge by species in \verb|eV| units. This values are seted
principally for the use of interatomic potential where the atomic charges are
involved.

How to use it :

\begin{itemize}
 \item Seting the charges of the O and Ge atoms. The values have to be numbers.
\control{charge O XX \\ charge Ge XX}
\end{itemize}

\subsubsection{mass}

Set the mass of the atomic species in \verb|a.u.| units. These values are seted
principally to be used with interatomic potentials where you need change the
atomic mass, because the normal mass are internally controlled by the
\verb|API| of the {\lpmd}.

How to use it :

\begin{itemize}
 \item Set the mass of the Ge and O atoms. The values have to be numbers.
\control{mass O XX \\ mass Ge XX}
\end{itemize}

\subsubsection{periodic}

This option specify how is the cell periodicity during the simulation. If you
change the periodicity in one axis, this will be seted in both sides of the
cell, check carefuly this options. A standard way to use is :

\control{periodic false false true}

In this previous case we will have periodicity only in the \verb|z| axis. The
order of the periodicity flags are \verb|x|, verb|y| and \verb|z|.

\subsubsection{steps}
This option set the number of simulations steps of the molecular
dynamics run. This have to be a integer number.

\control{steps 10000}

With this we indicate that the number of steps will be 10K.

\subsubsection{monitor/average}
The \verb|monitor| and the \verb|average| options show general information
about the simulation. This properties can be showed instantly (\verb|monitor|)
or average on a range (\verb|average|). The properties that you are interested
can be set by the user, these make the output highly configurable under your
own requirements.

Between the option of monitor and average you will find :

\fb{
\begin{tabular}{lcl}
 start & = & The initial time step in order to show the information.\\
 end & = & The final time step to finish show the information.\\
 each & = & Each how many steps the information is displayed.\\
 properties & = & Indicate the properties that you want look. \\
 output & = & Ouput file to store the values, if is not used the \\
 & & \textit{standard output} of the system will be used.\\
 interval & = & Only for average plugin, this indicate the average \\
 & & interval during the simulation. \\
\end{tabular}
}

For example if we want check the energy values during the simulation every 10
timesteps, we will use :

\begin{verbatim}
monitor start=0 end=1000 each=10 properties=step,kinetik-energy,\
        potential-energy,total-energy output=salida.out
\end{verbatim}

Some of the \textbf{properties} options availables are :

\begin{itemize}
 \item step : Show the timestep of the simulation.
 \item kinetic-energy : Show the kinetic energy.
 \item potential-energy : Show the potential energy.
 \item total-energy : Show the total energy.
 \item temperature : Show the system temperature.
 \item virial-pressure : Show the virial term of the pressure.
 \item kinetic-pressure : Show the pressure kinetic value.
 \item pressure : Show the total pressure of the system.
 \item volume : Shwo the simulation volume.
 \item cell-x : x=a,b,c Cell longs in each axis.
 \item sij : i,j=x,y,z Stress tensor of the system.
 \item volume-per-atom : Show the volume per atom of the cell.
 \item particle-density : Show the density by particle of the cell.
 \item density : Show the density of the simulation cell.
 \item momentum : Show the momentum on the cell.
 \item px,py,pz : Show each component of the momentum of the system.
\end{itemize}

\subsection{Filters}

This kind of plugins are used to \textit{filter} atoms in a simulation cell
with some specific requirements, some of the filters plugins availables are :

\vspace{1cm}
\begin{center}
\begin{tabular}{lcl}
box &:& Filter atoms that are located in a paralellepiped.\\
cylinder &:& Filter atoms located inside a cylinder.\\
element &:& Filter atoms by atomic elements.\\
external &:& Filter atoms by an external file with some information.\\
index &:& Filter atoms by index.\\
random &:& Filter random atoms.\\
sphere &:& Filter atoms located in a sphere.\\  
tag &:& Filter atoms by a specific tag.\\
\end{tabular}
\end{center}
\vspace{1cm}

The principall advantage of these filters is that we can generate new
configurations from previous results, and more specific and detailed analysis
on the atomic configurations.

\subsection{Plugins load}

In this section we will show how is in general the plugin load procedure in a
control file. The plugins can be loaded in any section of the control file,
however we suggest that you do this in a order way, like we eill se in some
examples later.

\subsubsection{How to load a plugin}

The plugin are from different types, and if a couple of plugins have the same
origin so they share some specific characteristics. A general idea of how the
plugins are splitted in the code is :

\begin{tabular}{lcl}
 Cell generators & : &Like \verb|xyz|, \verb|pdb|, \verb|crystalfcc|,
etc. \\
 Cell managers & : &Like \verb|minimumimage|, \verb|linkedcell|, etc. \\
 Cell Modifiers & : &Like \verb|cellscale|, \verb|tempscalling|, etc.\\
 Filters & : &Like \verb|sphere|, \verb|box|, etc.\\
 Properties Evaluators & : &Like \verb|gdr|, \verb|angdist|, \verb|msd|, etc.
\\
 Integrators & : &Like \verb|verlet|, \verb|euler|, etc. \\
 Potentials & : &Like \verb|LennardJones|, \verb|SuttonChen|, \verb|Morse|,
etc. \\
\end{tabular}

In order to load a plugin in a control file, you will need use the \verb|use|
and\verb|enduse| statement in the file. For example if we want load a
lennardjones potential plugin in the control file we will use :

\control{use lennardjones \\ cutoff 8.0 \\ epsilon 1.5 \\ sigma 0.6 \\enduse}

In general, always a plugin can be seted by an \textbf{alias} for a posterior
applying procedure.

\control{use PLUGINS as ALIAS \\ ... \\enduse}

In this way the plugin \texttt{PLUGIN} can be called later with the name of
\texttt{ALIAS}, this is an advantage because we can combine and reuse the same
plugin multiples times.

\subsection{How to apply a plugin}

The plugin applying process is located in the final section of the control
file, because we have different kind of plugins, each kind is called in a
different way, however is a very easy way for each.

\begin{itemize}
 \item \textbf{Modifiers}
  This plugins are used to modify some property of the simulation cell. This
must be applied using
  \control{apply module-alias start=0 end=1000 each=20}
 \item \textbf{Filters}
  This plugins are used to modify the simulationcell by filter atoms with some
specific conditions, can be used neither with \verb|over| of \textbf{apply} or
with \textbf{filter}.
  \control{apply mod-alias-apply <apply-options> over mod-alias-filter
  <filter-options>}
  \control{filter module-alias <options>}
 \item \textbf{Properties evaluation}
  This plugin determine properties and can be called each some specific
timesteps during a steps interval. You can use too in order to evaluate
properties over a specific set of atoms using filters with \verb|over|
instruction.
  \control{property module-alias start=0 end=1000 each=10}
  \control{property gdr start=0 end=-1 each=5\textbackslash over <filter-options>}
 \item \textbf{Integrators}
 This plugin are the integrators of the movement equation during the
simulation. You can change the integrator during the simulation using start and
end options in the option.
  \control{integrator module-alias start=0 end=-1}
 \item \textbf{Potentials}
 This plugins define the atomic interaction between two atomic kinds (nowadays
are not three bodies available). You will set the potential between two species
using :
  \control{potential module-alias Pt Au}
 \item \textbf{Cell Managers}
 This is one of the more imporant plugins simulations and analysis procedures,
this indicate how the cell is analyzed during the simulation process. In
molecular dynamics the more used are, \verb|minimumimage| and \verb|linkedcell|.
\verb|minimumimage|.
  \control{cellmanager linkedcell}
\end{itemize}

%%%%%%%%%%%%%%%%%%%%%%%%%%%%%%%%%%%%%%%%%%%%%%%%%%%%%%%%%%%%%%%%%
%%%%%%%%%%%%%%%%%%%%%%%%%%%%%%%%%%%%%%%%%%%%%%%%%%%%%%%%%%%%%%%%%
\section{Output files}
\index{output}

{\lpmd} can handle different kind of output files, the principall one is the
output file generated by the \verb|output| option in the control file,
additional output and error information are displayed in the standard output
and standard error while the code is executed. The standard output/error could
be redirected to different files. Another important ouput file that you can
generate with {\lpmd} are the ouput files that can handle the plugins. Some
plugins, for example the analysis plugins, have additional output information
that can be assigned to a different output file. 

In the following sections we will describe briefly each one of this ouput files.

%%%%%%%%%%%%%%%%%%%%%%%%%%%%%%%%%%%%%%%%%%%%%%%%%%%%%%%%%%%%%%%%%
%%%%%%%%%%%%%%%%%%%%%%%%%%%%%%%%%%%%%%%%%%%%%%%%%%%%%%%%%%%%%%%%%
\subsection{Standard output/Standard error}
The standard ouput and error are the information that {\lpmd} show in the
screen during the simulation process. USually this information is just a
recheck infromation about the cell and the simulation settings, however
sometimes a couple of warnings could be displayed during this process. In order
to save this information in additional files you can redirect the output using :

\control{lpmd fichero.control >~ salida.out}

or see in the screen and sent to a file:

\control{lpmd fichero.control | tee salida.out}

Finally the standard output and error could be both send to files, executing
{\lpmd} with :

\control{lpmd fichero.control 1\&> std.out 2\&> std.err}
\noindent
in this case the output file \verb|std.err| will have all the error messages
that lpmd will show and the \verb|std.out| will have all the information that
{\lpmd} usually sent to the screen, between this information you will found :

\begin{itemize}
 \item Cell complete description
 \item Information of \verb|startinfo|
 \item USed plugins and variables load.
 \item Properties by \verb|monitor| in case that you dont use the \verb|ouptut|
option in monitor.
 \item \textit{Debugger Information} : Additional information if you use the
\verb|-v| flag during the execution.
\end{itemize}

%%%%%%%%%%%%%%%%%%%%%%%%%%%%%%%%%%%%%%%%%%%%%%%%%%%%%%%%%%%%%%%%%
%%%%%%%%%%%%%%%%%%%%%%%%%%%%%%%%%%%%%%%%%%%%%%%%%%%%%%%%%%%%%%%%%
\subsection{File generated by output option}

This file is generated with the name given in the control file in the
\verb|output| option set at the first section. In this file you will find the
atomic configurations during the Molecular Dynamics simulation. This kind of
files are usually used to generate videos and more detailed analysis of the
simulation.

Is important define a good output format, because is the key in order to
simplify or make more difficult a later analysis. For example if you save a
single \verb|xyz| file with \verb|level=0| this is not the ideal case in order
to calculate the \textit{Valocity autocorrelation function} (\verb|vacf|)
because in hte case that the \verb|level=0| will obligate to the plugin to
evaluate the velocity of the system (instead of read from the file), that will
consume more time during the analysis. This time savings could be avoided in
this case save the xyz file with a \verb|level=1|. 

On the other hand you have to be clear about what will be the later analysisi
to realize after the simulations, because save in more higher level the output
file some times take more time too. You will have to find a equilibrium between
output file size, posterior analysis and times.

\subsection{Files generated by plugins}

Many plugins of {\lpmd} realize analysis that generate output files with
information in a particular form. In particular the instantaneous properties
can handle two different ways to write in a output file using the
\verb|average| argument. For example if we want to evaluate the \textit{Pair
distribution function} over a set of configurations, we can evaluate this one
by one (\verb|average=false|) or instead realize an average over all the
availables results (\verb|average=true|).

On the other hand, the instantaneous properties can be evaluated during the
simulation process, however {\lpmd} have tools to realize later analysis liek
you will see in the section~\ref{chap:utilidades}. For these cases you can use
too the \verb|average| option.

\begin{itemize}
\item \verb|average true| :
Say that after realize the analysus over all the configuration, the results
must be aveare and save in the output file.

\item \verb|average false| :
Say that after realize tha analysis over each configuration, these results must
be saved in the file directly.
\end{itemize}
