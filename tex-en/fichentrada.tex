\chapter{The Control File} % Translation by F. Gonzalez
\label{chap:input}

One of the fundamental pieces in {\lpmd} to run a molecular dynamics simulation
is the \textbf{control file} of the system. This file specifies all the
requirements that the simulation needs to be executed, including the name and
type of file where the positions and velocities of the atoms are, if these are
needed. In fact usually we are going to work with two kind of files :

\begin{itemize}
 \item input-file : Atomic positions and cell details. This file have
principally the atomic structure of our simulation.
 \item control-file : Specify the simulation procedure details, some times the
atomic structures are included too in this file.
\end{itemize}


Now, before we continue, we are going to give a short description of the
files that have the information about the atomic positions and cell
specifications: the input files. And the different types that {\lpmd} can
handle.

\section{The file with the atomic positions, the input-file}

{\lpmd} has many modules (see chapter~\ref{chap:modulos:entradasalida}), and the
 input/output files are one of them. The modules are plugins of the program
that, in this case, manage the reading and writing of atomic positions and/or
velocities and/or accelerations. Some available formats (modules) for doing this
are availables in {\lpmd} \textbf{xyz}, \textbf{lpmd}, \textbf{dlpoly}
(\verb|CONFIG| or \verb|HISTORY| files types from dl\_poly), etc.

We expect that the users, depending on their needs, help us to implement (or
ask the developers to do it) new input/output modules. Now let's see breafly
what they consist of.

%%%%%%%%%%%%%%%%%%%%%%%%%%%%%%%%%%%%%%%%%%%%%%%%%%%%%%%%%%%%%%%%%
%%%%%%%%%%%%%%%%%%%%%%%%%%%%%%%%%%%%%%%%%%%%%%%%%%%%%%%%%%%%%%%%%
\subsection{The file type .xyz}
\label{subsec:xyz}\index{xyz}

This is one of the more standard format used in a lot of different molecular
dynamics codes. It is an ASCII file that contains the position (in cartesian
coordinates) and atomic symbol of each atom of the simulation. The structure of
the file is basically:

\begin{center}
\begin{tabular}{l|l}
 \verb|N| & Specifies the number of atoms in the simulation cell.\\
 \verb|comment| & A line for comments, title, etc. This line is usually left
in blank.\\
 \verb|Sym X Y Z| & Atomic symbol, X coordinate, Y coordinate and Z coordinate.
  \\
\end{tabular}
\end{center}

Currently, the module \verb|xyz| has three different levels (0,1,2) which 
indicate the amount of information for each atom. The 0 (zero) level indicates
that the file contains (or will contain) the symbol and the 3-dimensional
position of each atom only in \AA units. The level 1 indicates that the file
contains not only the symbol and position, but the velocity (7 columns
file) in \AA/fs units. The level 2 is used when the user also needs the
aceleration of each atom (3 aditional columns) in \AA/fs$^2$ units.

\begin{itemize}
\item 0 : \verb|Sym X Y Z| (default value).
\item 1 : \verb|Sym X Y Z VX VY VZ|.
\item 2 : \verb|Sym X Y Z VX VY VZ AX AY AZ|.
\end{itemize}

%%%%%%%%%%%%%%%%%%%%%%%%%%%%%%%%%%%%%%%%%%%%%%%%%%%%%%%%%%%%%%%%%
%%%%%%%%%%%%%%%%%%%%%%%%%%%%%%%%%%%%%%%%%%%%%%%%%%%%%%%%%%%%%%%%%
\subsection{The file type .lpmd and .zlp}

The \verb|.lpmd| and \verb|.zlp| file format are a special type of format in
{\lpmd}. This files are plane ASCII format (lpmd) and a compress format (zlp).
The principal structure for this kind of files is given by :

\begin{center}
 \begin{tabular}{l|l}
 \verb|LPMD X.X C| & Header line with information about the version of the file
(X.X) and a special char (C) that indicate if is a compress file or not. \\
 \verb|HDR SYM X Y Z| & Show the information that every atomic line will have.\\
 \verb|cell properties | & A line with cell structure information. \\
 \verb|Sym sx sy sz| & Atomic information, depends of the HDR line.\\
\end{tabular}
\end{center}

In a similar way that for the case of \verb|xyz| file types, the \verb|lpmd|
file types have different levels (0, 1 and 2), but on the other hand they have
available different additional information about the atoms, information like
the atom color, or atom tags.

%%%%%%%%%%%%%%%%%%%%%%%%%%%%%%%%%%%%%%%%%%%%%%%%%%%%%%%%%%%%%%%%%
%%%%%%%%%%%%%%%%%%%%%%%%%%%%%%%%%%%%%%%%%%%%%%%%%%%%%%%%%%%%%%%%%
\subsection{Otros Formatos Soportados}

The two principall supported formats are \verb|xyz| and \verb|lpmd| (compress
and not compress variants). We have implemented another input plugins in order
to read or write atomic configurations from the file. All the availables to the
date are listed in the tables~\ref{tab:modinout} and~\ref{tab:cellgen}. If you
have questions or suggestions respect to the supported formats, please do not
hesitate to contact us.

We do not give a detailed information about this plugins here, because is not
deeply necessary.

\section{The control-file}


This is the principall file in order to realize a computational simulation. For
this reason we will give you a description of each part of this file and then
we will explain each one separately and deeply.

First of all, a general considerations about the control files:

\begin{itemize}
 \item \# Is a comment line. (Avoid this lines in a \texttt{use ... enduse}
section).
 \item The \verb|/| symbol indicate that the linea continue in the next line.
 \item Altought they may be random, we suggest that you have a order with
the plugins load.
 \item You have to load a plugin, before use it.
 \item Some plugins are mandatory, for example \textbf{cellmanager}.
However many of them can be applied in command line execution.
\end{itemize}

With all this point in mind, we will see the principal sections to consider in
a control file in order to use {\lpmd}. A general scheme is show next.

\fb{ 
\texttt{
\begin{tabular}{lcl}
 \#Cell Properties & $\rightarrow$ & Properties about the CELL.\\
 cell ... &&\\
 \#Input/Output & $\rightarrow$ & Input/Output atomic files\\
 input ... &&\\
 output ... &&\\
 \#General & $\rightarrow$ & General Simulation settings\\
 prepare ... &&\\
 steps ... &&\\
 monitor ... &&\\
 \#Filters & $\rightarrow$ & Filters in the atomic configuration.\\
 filter ... &&\\
 \#Module Load & $\rightarrow$ & Load modules.\\
 use ... &&\\
 enduse ... &&\\
 \#Module Apply & $\rightarrow$ & Applying modules.\\
 apply ... &&\\
 potential ... &&\\
 integrator ... &&\\
\end{tabular}
}
}

This is a general approximation to the control files used in molecular dynamics
by {\lpmd}. With this idea in mind we will analyse each of these section
separately and deeply.

%%%%%%%%%%%%%%%%%%%%%%%%%%%%%%%%%%%%%%%%%%%%%%%%%%%%%%%%%%%%%%%%%
%%%%%%%%%%%%%%%%%%%%%%%%%%%%%%%%%%%%%%%%%%%%%%%%%%%%%%%%%%%%%%%%%
\subsection{Simulation Cell.}

Es la propiedad que describe la celda de simulaci\'on. Es decir el detalle
completo de cada uno de los ejes que la conforman, los que pueden ser entregados
en forma detallada o en forma general. A continuaci\'on se describir\'a la forma
en la cu\'al se entrega \'esta propiedad, que \textit{casi siempre} debe estar
presente en el fichero .control y va al comienzo de \'este.

\subsubsection{cell}

El flag cell es utilizado para describir la celda de simulaci\'on y asignar las
propiedades de \'esta. Generalmente una celda de simulaci\'on puede venir
descrita ya en el formato del fichero de entrada (como es el caso de lo ficheros
\texttt{lpmd} o \texttt{CONFIG}). Sin embargo, hay formatos, como el
\texttt{xyz}, que no poseen la descripci\'on de la celda, es por eso que es
necesario en algunos casos utilizar esta opci\'on. Si se desea dar la opci\'on
\verb|cell| aunque la informaci\'on se encuentre en un archivo, \'esta
predominar\'a sobre la del archivo.

\begin{itemize} 
\item{Forma 1}

Se utilizan la longitud de los lados y \'angulos de la celda, como sigue:

\control{cell a=10 b=5 c=5 alpha=45 beta=90 gamma=90}

donde,

\fb{ 
\begin{tabular}{lcl}
 a & = & indica el largo de la celda en \textbf{a}.\\
 b & = & indica el largo de la celda en \textbf{b}.\\
 c & = & indica el largo de la celda en \textbf{c}.\\
 alpha & = & indica el \'angulo $\alpha$.\\
 beta & = & indica el \'angulo $\beta$.\\
 gamma & = & indica el \'angulo $\gamma$.\\
\end{tabular}
}

\item{Forma 2}

Se utilizan los 3 vectores bases, poni\'endolos de la siguiente manera en el
fichero, note que / indica la continuaci\'on de una l\'inea \'unica.

\control{cell ax=1.0 ay=0.0 az=0.0 bx=0.0 by=1.0 bz=0.0 / \\cx=0.0 cy=0.0 cz=1.0}

donde: a${i}$, b${i}$ y c${i}$ con $i$={$x$,$y$,$z$} son las coordenadas $x, y$
y $z$ de los vectores bases.

Las posibles formas de ingresar una descripcion de la llamada \verb|cell| pueden
ser entregadas como argumentos en la ejecuci\'on de {\lpmd} y no necesitan estar
dentro del fichero de \textbf{control}, lo que ayuda a la creaci\'on de
\textit{scripts}.

\fb{\begin{minipage}[l]{9.5cm}\tt lpmd -L a,b,c -A alpha,beta,gamma
archivo.control \\ lpmd -V ax,ay,az,bx,by,bz,cx,cy,cz
archivo.control\end{minipage}}


\item{Forma 3}

Existe una forma extra para celdas cubicas que ahorran un poco la escritura
completa de cada t\'ermino de la celda, por ejemplo una celda cubica de largo
5\AA, se puede asignar facilmente con :

\control{cell cubic a=5}

\end{itemize}

\subsubsection{Omitiendo cell}

Como se mencion\'o previamente, hay ocaciones en que el archivo de posiciones
at\'omicas posee adem\'as la informaci\'on de la celda de simulaci\'on, para
estos casos hay dos formas de \textbf{especificar} a {\lpmd} que debe leer la
informaci\'on desde ese archivo.

\begin{itemize} 
\item{Forma 1}

Se utilizan opciones especiales dentro del mismo fichero de control :

\control{set replacecell true}

\item{Forma 2}

Se especifica en la ejecuci\'on misma de lpmd con el flag \verb|-r|.

\fb{\begin{minipage}[l]{9.5cm}\tt lpmd archivo.control -r\end{minipage}}

\end{itemize}

%%%%%%%%%%%%%%%%%%%%%%%%%%%%%%%%%%%%%%%%%%%%%%%%%%%%%%%%%%%%%%%%%
%%%%%%%%%%%%%%%%%%%%%%%%%%%%%%%%%%%%%%%%%%%%%%%%%%%%%%%%%%%%%%%%%
\subsection{Entrada - Salida}

\subsubsection{input}

Existen actualmente dos formas de ingreso de un sistema de entrada para la
configuraci\'on at\'omica que se requiere simular; estas son, el ingreso de las
posiciones at\'omicas de la celda a trav\'es de un archivo (por ejemplo
\verb|.xyz| o \verb|.lpmd|) y el otro es mediante m\'odulos que generan
autom\'aticamente celdas at\'omicas con ciertas propiedades, por ejemplo celdas
\textbf{bcc}, \textbf{fcc}, etc.

Veamos brevemente a continuaci\'on cada uno de ellos,

\begin{itemize}
 \item{Con Fichero}

Para cargar un fichero con configuraciones at\'omicas es necesario la existencia
del m\'odulo que reconozca el tipo de fichero, por ejemplo para cargar un
fichero del tipo \verb|.xyz|, es necesario utilizar el m\'odulo \verb|xyz| para
poder leer sin problemas el archivo, ya que es el m\'odulo el que
\textit{entiende} el archivo de ese tipo. 
  \item{Generadores de celda}

A diferencia con el m\'etodo anterior, este m\'etodo no requiere de un fichero
con posiciones at\'omicas, en lugar de ello se requiere un m\'odulo que genera
automaticamente una celda con \'atomos, seg\'un los requerimientos propios del
m\'odulo. Por ejemplo existen m\'odulos actualmente para generar celdas del tipo
\textbf{sc}, \textbf{bcc}, \textbf{fcc}, etc. utilizando el plugin
\verb|crystal3d|, tambi\'en hay generadores de redes bidimensionales
(\verb|crystal2d|) y finalmente se pueden utilizar m\'etodos m\'as sofisticados
como \textbf{skewstart}.

\end{itemize}

La forma general de la orden \verb|input| requiere de argumentos para un
funcionamiento adecuado. Para ver m\'as informaci\'on sobre el m\'odulo revise
la secci\'on ~\ref{chap:modulos:entradasalida}. Estos son ejemplos de algunos
argumentos:

\fb{ 
\begin{tabular}{lcl}
 module & = & indica el m\'odulo con el que cargar la celda.\\ 
 file & = & indica el fichero con posiciones at\'omicas.\\
 level & = & indica el nivel del fichero.\\ 
\end{tabular}
}

Estos son los argumentos m\'as standard ya que cada m\'odulo posee sus propios
argumentos, por lo que se hace necesario ver cada uno seg\'un el inter\'es.

Veamos algunas formas de uso para la orden \verb|input| :

\begin{itemize}
\item Carga posiciones at\'omicas desde un fichero XYZ.
\control{input module=xyz file=fichero.xyz level=0}
\item Carga posiciones y velocidades desde un fichero XYZ (level 1).
\control{input module=xyz file=fichero.xyz level=1}
\item Inicializa una celda del tipo fcc con átomos de Au.
\control{input module=crystal3d type=fcc nx=3 ny=3 nz=3\textbackslash\\ symbol=Au}
\item Inicializa una celda del tipo sc con átomos de Na
\control{input module=crystal3d type=sc nx=5 ny=5 nz=5\textbackslash\\ symbol=Na}
\item Inicializa con metodo skewstart para 108 \'atomos de arg\'on.
\control{input module=skewstart atoms=108 symbol=Ar}
\end{itemize}

La lista de los m\'odulos soportados a la fecha para lectura/generaci\'on de
configuraciones, se pueden observar en la tabla~\ref{tab:modinout}
y~\ref{tab:cellgen}.

\subsubsection{output}

Con el par\'ametro \verb|output| se especifican las opciones de salida de las
configuraciones at\'omicas de nuestra simulaci\'on, los formatos de salidas son
complementamente modulares y pueden ser implementados por los usuarios, sin
embargo a partir de la version 0.5.2 del set de plugins \verb|lpmd-plugins| ya
se encuentran disponibles muchos m\'odulos, pese a esto es \textit{importante}
notar que \textbf{cada m\'odulo posee configuraciones independientes}, por
ejemplo \verb|level| es utilizado por m\'odulos como \textbf{xyz},
\textbf{dlpoly} o \textbf{lpmd}, sin embargo no es requerido para \textbf{mol2},
para m\'as informaci\'on refierase a la secci\'on
~\ref{chap:modulos:entradasalida}. Los argumentos generales m\'as utilizados del
par\'ametro \verb|output| son:

\fb{
\begin{tabular}{lcl}
 module & = & indica el m\'odulo (formato) de \\
&&salida de la simulaci\'on.\\
 file & = & indica el fichero en el que graba.\\
 level & = & indica el nivel del modulo de salida.\\
 each & = & indica cada cuantos pasos la celda \\
&&es gabada en el fichero.\\
\end{tabular}
}

Al igual que antes, existen m\'as par\'ametros que son independientes de cada m\'odulo.

Algunas formas de uso,

\begin{itemize}
 \item Grabando la simulaci\'on en un fichero XYZ (nivel 0), cada 20 steps.
\control{output module=xyz file=fichero.xyz level=0 each=20}
 \item Grabando la simulaci\'on en fichero LPMD (nivel 1), cada 1 step.
\control{output module=lpmd file=fichero.lpmd level=1 each=1}
 \item Grabando las posiciones at\'omicas en formato lpmd nivel 2 y con colores de los \'atomos.
\control{output module=lpd file=saved.lpmd level=2 each=5\textbackslash\\ extra=rgb}
\end{itemize}

La lista de los m\'odulos soportados a la fecha para escritura de
configuraciones, pude verse en la tabla~\ref{tab:modinout}.

\subsubsection{restore}\label{fich:restauracion}

Es utilizado para restaurar una simulaci\'on a partir de un punto en que se
produjo un corte de energ\'ia el\'ectrica o cualquier otro tipo de falla
f\'isica en un centro de c\'alculo. El punto de restauraci\'on es a partir de el
\'ultimo dumping realizado por la simulacion, dado por la orden ``dumping''
dentro del fichero de control, indicando el nombre del archivo \verb|dump|. Es
recomendable que antes de reiniciar una corrida, respalde los datos en otro
directorio, o efectue la \textit{reiniciaci\'on} de la corrida en un directorio
distinto, para evitar da\~nar, perder o sobreescribir los datos previos de la
simulaci\'on.

Actualmente no se han realizado pruebas exhaustivas de este punto, pero
deber\'ia funcionar sin problemas, por favor si encuentra alg\'un bug, reportelo
y \textbf{recuerde usar esta opci\'on con precauci\'on}.

Consideremos una corrida estandard de din\'amica molecular en donde el fichero
de control posee la l\'inea :

\control{dumping file=restauracion.dat each=50000}

\noindent
de esta forma el sistema guardar\'a una configuracion de restauraci\'on cada 50
mil pasos en el fichero \verb|restauracion.dat|. Ahora si ocurri\'o alguna falla
durante el calculo, podemos reiniciar la corrida con {\lpmd} para ello
recomendamos copie todos los archivos en otro directorio y luego añada las
siguientes l\'ineas al fichero de control

\control{...\\ dumping file=restauracion-2.dat each=50000 \\  restore file=restauracion.dat \\...}

De esta forma finalmente el sistema correra a partir del paso (m\'ultiplo de 50
mil) en el cu\'al se produjo el error.

%%%%%%%%%%%%%%%%%%%%%%%%%%%%%%%%%%%%%%%%%%%%%%%%%%%%%%%%%%%%%%%%%
%%%%%%%%%%%%%%%%%%%%%%%%%%%%%%%%%%%%%%%%%%%%%%%%%%%%%%%%%%%%%%%%%
\subsection{Propiedades Generales}
\subsubsection{prepare}

Esta opci\'on es utilizada para \textit{setear} valores y caracter\'isticas de
la simulaci\'on, que son brindadas a trav\'es de \verb|plugins| o de la misma
\verb|API|, tenemos por ejemplo :

\begin{itemize}
 \item \textbf{temperature}
Para dar una temperatura inicial al sistema, se prepara la celda con :
\control{prepare temperature t=300}
de esta forma el sistema asigna velocidades iniciales a las part\'iculas para
que la temperatura de nuestro sistema corresponda a 300K en el instante de
tiempo inicial.
 \item \textbf{replicate}
Para replicar nuestra celda en las distintas direcciones de los vectores bases,
la forma de hacerlo para una celda antes de comenzar la simulaci\'on es:
\control{prepare replicate nx=2 ny=2 nz=2}
De esta froma, la celda que se ley\'o en \verb|input| es replicada 2 veces por
cada eje, alcanzando 8 veces el n\'umero inicial de part\'iculas. Esta opci\'on
es v\'alida s\'olo cuando se desactiva la optimizaci\'on previa utilizando
\verb|set|.
\end{itemize}

\subsubsection{set}
Utilizado para setear valores de la simulaci\'on, principalmente para algunas
variables globales del sistema. A continuaci\'on algunos de los m\'as utilizados
:

\begin{itemize}
 \item Desactivando la optimizaci\'on de celda previa a la simulaci\'on.
\control{set optimize-simulation false}
 \item Se asigna que la informaci\'on necesaria para \texttt{cell} est\'a en el
fichero de entrada.
\control{set replacecell true}
 \item Seteando la variable \texttt{delay} usada en visualizaci\'on.
\control{set delay 0.1}
\end{itemize}

\subsubsection{charge}

Asignaci\'on de las cargas en \verb|eV| para las especies at\'omicas. Estos
valores de las cargas, son seteados principalmente para utilizaci\'on de
potenciales interat\'omicos en los cuales se utilizan las cargas de los atomos
involucrados.

Forma de uso

\begin{itemize}
 \item Seteando las cargas de los atomos de O y Ge.
\control{charge O XX \\ charge Ge XX}
\end{itemize}

\subsubsection{mass}

Asignaci\'on de la masa en \verb|a.u.| para las especies at\'omicas. Estos
valores, son seteados principalmente para utilizaci\'on de potenciales
interat\'omicos en los cuales se desea modificar la masa de los \'atomos
involucrados.


Forma de uso

\begin{itemize}
 \item Seteando las cargas de los atomos de O y Ge.
\control{mass O XX \\ mass Ge XX}
\end{itemize}


\subsubsection{periodic}

Indica la periodicidad de la celda, en cada eje. Al bloquear la periodicidad en
un eje, este se ve ``modificado'' en ambos lados de la celda, revise con cuidado
estas opciones.

\control{periodic false false true}

En \'este caso s\'olo tenemos periodicidad en el eje \verb|z|. El orden de la
periodicidad es \verb|x|, \verb|y| y \verb|z|.

\subsubsection{steps}
N\'umero de pasos de la simulaci\'on de din\'amica molecular.

\control{steps 10000}

Ac\'a se indica que la simulaci\'on se realizar\'a con 10000 pasos.

\subsubsection{dumping}
Genera una salida global del sistema para poder restaurar a partir de ese punto.

\control{dumping file=rescue.dump each=10000}

Generamos un fichero de volcado cada 10000 pasos de la simulaci\'on, en \'el se
graba toda la informaci\'on necesaria, para reiniciar una corrida. Para m\'as
detalle sobre el uso de los ficheros de restauraci\'on refi\'erase a
\ref{fich:restauracion}.

\subsubsection{monitor}
La orden \verb|monitor| indica cada cuantos pasos la simulaci\'on muestra las
propiedades globales. Estas propiedades, pueden ser asignadas por el mismo
usuario, haciendo la salida lo m\'as configurables seg\'un los propios
requerimientos.

Entre las opciones de monitor, cuentan :

\fb{
\begin{tabular}{lcl}
 start & = & indica el valor de epsilon.\\
 end & = & indica el valor de sigma.\\
 each & = & indica el cutoff del potencial.\\
 properties & = & indica que valores se desea monitorear. \\
 output & = & archivo de salida para guardar los valores, \\
 & & si no, el \textit{standard output} es utilziado.\\
\end{tabular}
}

Si queremos ir chequeando, los valores de la energ\'ia durante la simulaci\'on,
cada 10 pasos, utilizamos la l\'inea,

\begin{verbatim}
monitor start=0 end=1000 each=10 properties=step,kinetik-energy,\
        potential-energy,total-energy output=salida.out
\end{verbatim}

Algunas de las opciones soportadas por \textbf{properties} son:

\begin{itemize}
 \item step : Muestra el paso actual de la simulaci\'on.
 \item kinetic-energy : Muestra la energ\'ia cin\'etica.
 \item potential-energy : Muestra la energ\'ia potencial.
 \item total-energy : Muestra la energ\'ia total.
 \item temperature : Muestra la temperatura del sistema.
 \item virial-pressure : Aporte del t\'ermino del virial a la presi\'on.
 \item kinetic-pressure : T\'ermino de la presi\'on asociado.
 \item pressure : Presi\'on total del sistema.
 \item volume : Volumen de la celda de simulaci\'on.
 \item cell-x : x=a,b,c son los largos de la celda en cada eje.
 \item sij : i,j=x,y,z entrega los valores para el tensor de stress.
\end{itemize}

\subsection{Filtros}

Aparecen en versiones posteriores a 0.6.0, son m\'odulos cuya caracter\'istica
principal es \textit{filtrar} los \'atomos de una simulaci\'on acorde a ciertos
tipos de requerimientos, por ejemplo :

\vspace{1cm}
\begin{center}
\begin{tabular}{lcl}
index &:& Filtrado de atomos seg\'un sus \'indices.\\
box &:& Filtrado de \'atomos pernetecientes a paralelep\'ipedo.\\
sphere &:& Filtrado de \'atomos seg\'un esfera.\\  
element &:& Filtrado de \'atomos seg\'un simbolo at\'omico.\\
\end{tabular}
\end{center}
\vspace{1cm}

La ventaja de estos filtros es poder generar nuevas configuraciones a partir de
resultados previos, as\'i como tambi\'en an\'alisis mucho m\'as detallados para
los \textit{\'atomos filtrados}.

\subsection{Carga de M\'odulos}

Ac\'a mostraremos c\'omo se utilizan en general la carga de m\'odulos dentro de
un fichero de control. Los m\'odulos o plugins pueden ser cargados en cualquier
secci\'on del fichero de control, sin embargo recomendamos hacerlo de forma
ordenada como veremos en los ejemplos posteriores.

\subsubsection{C\'omo cargar un m\'odulo}

Los m\'odulos son de distintos tipos en general, y los de un tipo en com\'un
comparten ``ciertas'' caracter\'isticas ya que cada uno posee sus propias
ventajas. Una visi\'on general de c\'omo se han distribuido los m\'odulos es: 

\begin{tabular}{lcl}
 Generadores de celda & : & Tales como \verb|xyz|, \verb|pdb|, \verb|crystalfcc|, etc. \\
 Manejadores de celda & : &\verb|minimumimage|, \verb|linkedcell|, etc. \\
 Modificadores de celda & : &\verb|cellscale|, \verb|tempscalling|, etc.\\
 Filtros & : &\verb|sphere|, \verb|box|, etc.\\
 Calculadores de Propiedades & : & \verb|gdr|, \verb|angdist|, \verb|msd|, etc. \\
 Integradores & : & \verb|verlet|, \verb|euler|, etc. \\
 Potenciales & : & \verb|LennardJones|, \verb|SuttonChen|, \verb|Morse|, etc. \\
\end{tabular}

En general, siempre a un m\'odulo se le puede asignar un \textbf{alias} para su
posterior llamado, por ejemplo.

\control{use MODULO as ALIAS \\ ... \\enduse}

De esta forma el m\'odulo \texttt{MODULO} puede ser llamado en forma posterior
con el nombre \texttt{ALIAS}, lo que da una ventaja para combinar y simplificar
la utilizaci\'on de un m\'odulo en m\'as ocaciones.

Entonces dentro de un archivo de \textbf{control}, debemos cargar los m\'odulos
necesarios para un posterior llamado.

\subsection{Aplicaci\'on de m\'odulos}

Los m\'odulos son llamados en la parte final de nuestro fichero de
\textbf{control}, como existen distintas ``especies'' de m\'odulos, estos deben
ser llamados de diferentes maneras, aunque su forma es muy general.

\begin{itemize}
 \item \textbf{Modificadores}
  M\'odulos llamados para modificar alguna propiedad de la celda de
simulaci\'on. Se aplican con:
  \control{apply module-alias start=0 end=1000 each=20}
 \item \textbf{Filtros}
  M\'odulos que modifican la celda de simulaci\'on filtrando \'atomos seg\'un
ciertas condiciones, pueden ser  utilizados tanto con la instrucci\'on
\verb|over| de \textbf{apply} como con \textbf{filter}.
  \control{apply mod-alias-apply <apply-options> over mod-alias-filter
  <filter-options>}
  \control{filter module-alias <options>}
 \item \textbf{Calculadores de Propiedades}
  Estos siempre son llamados a ser evaluados cada cierto tiempo entre un rango
de intervalos. A partir de la versi\'on 0.6.0 de {\lpmd} se pueden adem\'as
calcular propiedades sobre un \textit{set} de \'atomos filtrados con
\verb|over|.
  \control{property module-alias start=0 end=1000 each=10}
  \control{property gdr start=0 end=-1 each=5\textbackslash over <filter-options>}
 \item \textbf{Integradores}
 Estos son llamados facilmente con:
  \control{integrator module-alias}
 \item \textbf{Potenciales}
 Estos definen una interaccion entre dos \'atomos (por el momento no hay de tres
cuerpos). Se crea un potencial para cada interacci\'on especificando los valores
y llamando luego con
  \control{potential module-alias Pt Au}
 \item \textbf{Manejadores de celda}
 Llaman al manejador de celda que se utilizara en la simulaci\'on, en nuestro
caso hay dsiponibilidad de dos manejadores, \verb|linkdecell| y
\verb|minimumimage|.
  \control{cellmanager linkedcell}
\end{itemize}

\section{Ficheros de Salida}

%%%%%%%%%%%%%%%%%%%%%%%%%%%%%%%%%%%%%%%%%%%%%%%%%%%%%%%%%%%%%%%%%
%%%%%%%%%%%%%%%%%%%%%%%%%%%%%%%%%%%%%%%%%%%%%%%%%%%%%%%%%%%%%%%%%
\subsection{Ficheros de salida}

{\lpmd} Tiene variados tipos de ficheros de salida, uno que es generado
usualmente por la opci\'on \verb|output| dentro del fichero de control, otro es
la salida standard, que por defecto va a pantalla, pero que nosotros
recomendamos enviar (o redireccionar) a un fichero adem\'as estan los que
generan cada uno de los m\'odulos en forma independiente cuando estos requieren
de aquello.

%%%%%%%%%%%%%%%%%%%%%%%%%%%%%%%%%%%%%%%%%%%%%%%%%%%%%%%%%%%%%%%%%
%%%%%%%%%%%%%%%%%%%%%%%%%%%%%%%%%%%%%%%%%%%%%%%%%%%%%%%%%%%%%%%%%
\subsubsection{Salida standard}
Esta es la salida que muestra en pantalla lpmd, la forma de enviar esta salida a
un fichero, durante la ejecuci\'on de lpmd, es

\control{lpmd fichero.control >~ salida.out}

o si desea verla en pantalla y enviar a un fichero:

\control{lpmd fichero.control | tee salida.out}

tambi\'en puede redireccionar una salida standard y los mensajes de error de
forma independiente utilizando:

\control{lpmd fichero.control 1\&> salida.out 2\&> salida.err}

el fichero \verb|salida.out| tendr\'a toda la informaci\'on que debi\'o salir a
pantalla utilizando lpmd, entre ella se encuentran,

\begin{itemize}
 \item Descripci\'on completa de la celda
 \item Informacion de \verb|startinfo|
 \item Informaci\'on de m\'odulos utilizados y variables de cada uno de ellos.
 \item Energ\'ias, Temperatura, Presi\'on y Vol\'umen, seg\'un \verb|monitor|.
En caso de que monitor no utilize un valor propio para el valor de la variable
\verb|output|.
 \item \textit{Debugger Information} : Informaci\'on sobre aplicaciones y otras
cosas que se realizan durante la simulaci\'on usualmente dirigida a
\verb|std::cerr|.
\end{itemize}


%%%%%%%%%%%%%%%%%%%%%%%%%%%%%%%%%%%%%%%%%%%%%%%%%%%%%%%%%%%%%%%%%
%%%%%%%%%%%%%%%%%%%%%%%%%%%%%%%%%%%%%%%%%%%%%%%%%%%%%%%%%%%%%%%%%
\subsubsection{Fichero generado por output}

Este fichero se gener\'o con el nombre entregado en el fichero de control a la
linea \textbf{output}, en \'el se encuentran las configuraciones atomicas de la
\textbf{DM} y suelen ser los ficheros a partir de los cuales suelen crearse
animaciones y an\'alisis detallados de la simulaci\'on.

Todas las propiedades instantaneas pueden calcularse \textit{\textbf{durante la
simulaci\'on misma}}, sin embargo {\lpmd} tambi\'en cuenta con herramientas
propias de an\'alisis como se puede ver en la secci\'on~\ref{chap:utilidades}.

Es importante definir un buen formato de salida ya que es la clave para
simplificar o dificultar los an\'alisis posteriores. Por ejemplo si se adquiere
un archivo \verb|xyz| con \verb|level=0| no es el ideal para calcular por
ejemplo la \textit{Velocity autocorrelation Function} (\verb|vacf|) ya que en
caso de que el nivel sea menor a uno algunos plugins determinar\'an la velocidad
utilizando configuraciones at\'omicas sucesivas de forma autom\'atica lo que
aumentar\'a un poco el tiempo de an\'alisis, que pudo aver sido aprovechado
durante la simulaci\'on grabando con \verb|level=1| nuestro fichero de salida.

\subsubsection{Ficheros generados por m\'odulos}
Muchos m\'odulos de lpmd que realizan an\'alisis generan ficheros de
informaci\'on de manera independiente, adem\'as los plugins en general manejan
dos \textit{est\'andares} de salidas asignados con el flag \verb|average|. Por
ejemplo si deseamos calcular la \textit{Funci\'on de distribuci\'on de pares}
sobre nuestras configuraciones, podemos calcular \'esta sobre cada una de las
muestras y guardarlas o bien generar un promedio sobre todas las
configuraciones.

\begin{itemize}
\item \verb|average true| :

Especif\'ca que luego de realizar los correspondientes an\'alisis sobre las
configuraciones especificadas del fichero, estas deben promediarse antes de
guardar en el archivo de salida.
\item \verb|average false| :

Especif\'ica que cada vez que realiza un an\'alisis sobre las configuraciones
pertenecientes al archivo, estas deben ser guardadas una a una en el archivo de
salida. 
\end{itemize}
